%\newpage

\section{Summary and future work}
\label{summary}

Barrier synchronization is a well-studied topic, and an important
one in parallel programming. In this paper, we studied the performance
characteristics of several alternative implementations of barrier
synchronization on modern, complex shared memory multiprocessors.

We have implemented several different barrier schemes and measured their performance on the shared memory POWER3 system and the distributed shared
memory POWER4 system. We analyzed barrier performance data through
timing and hardware performance counters. In the future, we wish to study
the impact of different barrier implementations on other shared memory
platforms particularly looking at issues of non-uniform memory access and
scalability.

Through the introduction of the \emph{distributed counters with local sensor}
implementation, we have demonstrated a dramatic 79\% reduction in overhead on a
32-way POWER4 system when comparing to a fetch-and-add implementation and a 33\%
improvement on the same system over the next best technique, distributed
counters with padding.  While these experiments were done using the explicit 
barrier test in EPCC microbenchmarks,
we get similar improvements in the performance of the implied barriers
terminating work-sharing constructs and parallel regions.  The availability of
a low overhead barrier also provides more freedom to the compiler to automatically
introduce parallelism in serial code.

% how about irregular

%Understanding more about the ccNUMA system

%We treat the system more like a distributed message-passing one.

%memory affinity processor affinity.

