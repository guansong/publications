\section{Introduction}
\label{introduction}

As several research papers pointed out, OpenMP 
%\cite{Ope00} \cite{Ope02}
\cite{Ope05} is due to an update \cite{Gon00}, \cite{Cha05}. The set of
features in the existing specification of the API provides essential
functionality that were mostly selected from previous shared memory parallel
APIs. Recently, the shared memory architectures have interesting developments,
from multi-core processors to hyperthread and SMT, from accelerate boards to
CELL broadband Engine\cite{Cel05}. The emerging architectures force us to
explore new features for the OpenMP standard. 

%Several proposals have already
%start the work\cite{Cha05}.

\begin{figure}[h!]
  \begin{center}
    \includegraphics[angle=0, width=0.45\textwidth]{arches.eps}
    \caption{\footnotesize Emerging architectures}
    \label{fig:arches}
  \end{center}
\end{figure}

Figure \ref{fig:arches} is an example of an extreme case of the latest development
of shared memory architectures. The picture includes dual core CPUs,
hyperthreads or SMT processors. Although the system is not modeled on any real
machine~\footnote{Circle \emph{a} and \emph{b} may represent two different
computers, or they are combined together to create an even more complicated
structure. This is purely imaginary. We used the figure just to illustrate the
complexity of the problem.}, it can illustrate the potential problems a
programmer may face in the real world.

Programming on such complicated systems is going to require difficult
compromise. The following items are elements that an ideal solution framework
should address, quoted directly from OpenMP language committee discussion:

\begin{itemize}

  \item \emph{Modularity}: Modern software engineering makes heavy use of
  modular software.  OpenMP's existing model, however, does not support
  compartmentalization of parallelism.  For example, MPI defines a
  communicator.  This can be passed to a library and the library is then
  restricted to the constraints defined by that communicator.  Furthermore, the
  library developer can rename the communicator so the interactions between
  components of the library are not exposed outside the library. 
  %We have no way to do this in OpenMP.

  \item \emph{Multi-level machines}: Shared memory machines will become
  increasingly hierarchical.  The combination of SMT and NUMA all come together
  to create a nightmare for the existing ``flat earth'' model of OpenMP. The
  scalar \texttt{OMP\_NUM\_THREADS} has to expand to encompass a multi-level
  abstraction of some kind. 

  \item \emph{Mapping OpenMP threads onto processors}: Different machines will
  have different hierarchies of processors.  A program must be able to query
  the system about the processor hierarchy and then adapt to it  by controlling
  how the OpenMP threads map onto processors.

  \item \emph{Worksharing between subteams}: To broaden the range of
  applications appropriate for OpenMP, we need to extend OpenMP so programmers
  can specify worksharing restricted to subteams.  

\end{itemize}

This is a very ambitious goal. We are not sure if such a solution exist to fit
all the requirements. Even if it does exist, the impact could be so large as to
make all the previous OpenMP programs obsolete; or make the implementation of
the standard so difficult that the drawbacks outweight the advantage it brings
to the standard.

In this paper we will describe our attempt to solve this problem. It may not be
the answer we are looking for, a lot of things need to be further improved.

To avoid a dramatic change in
the OpenMP standard, we propose an incremental approach, to consider
\emph{thread mapping} as the first step and address the bigger issue in the
second step. We will discuss these topics in the following sections.

