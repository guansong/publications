\section{Summary}

Through the simple examples in the report, we can see the programming
language presented here has the flexibility of a SPMD program, and the
convenience of HPF. The language encourages programmers to express
parallel algorithms in a more explicit way. We suggest it will help
programmers to solve real application problems more easily, compared with
using communication packages such as MPI directly, and allow the
compiler writer to implement the language compiler without the
difficulties met in the HPF compilation.

The Java binding is only an introduction of the programming style.
(A Fortran binding is being developed.) It can be used as a software
tool for teaching parallel programming.  As Java for scientific
computation become more mature, it will be a practical programming
language to solve real application problems in parallel and distribute
environments.

% how about irregular
