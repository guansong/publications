\section{Java language Binding} 

{\tt String} is a class in Java, but there is language syntax, including
construction and concatenation operations, to support it.  In
HPJava, we add several similar \emph{built-in} classes.

\subsection{Basic concepts}

Key concepts in the programming model are built around {\em process
groups}, used to describe program execution control in a parallel
program.

\paragraph{Process group.}
\emph{Group} is a class representing a process group, typically
with a grid structure and an associated set of {\em process
dimensions}.  It has its subclasses that represent different grid
dimensionalities, such as \texttt{Procs1}, \texttt{Procs2}, etc. For example,
\small
\begin{verbatim}
Procs2 p = new Procs2(2,4);
\end{verbatim}
\normalsize
An HPJava program will be executed in parallel across the processes
of a grid. 

\paragraph{Distributed dimension and index with position.}

The elements of an ordinary array can be represented by an array name
and an integer sequence.  Here, we have two concepts reflected by
\texttt{int} values: an index to access each array element and a range
that index can be chosen from.  In describing a distributed
array, we use two new built-in classes in HPJava to represent the
analogous concepts:
\begin{itemize}
\item A {\em range} maps an integer interval into a process dimension
  according to certain distribution format. Ranges describe the extent
  and mapping of array dimensions.

\item A {\em location}, or slot, is an abstract element of a range. A
  range can be regarded as a set of locations, actually it is a
  one-to-one mapping between the global index and locations.
\end{itemize}
For example,
\small
\begin{verbatim}
Range x = new BlockRange(100, p.dim(0)) ;
Range y = new CyclicRange(200, p.dim(1)) ;
\end{verbatim}
\normalsize
creates two ranges on the different process dimensions of the
group \texttt{p}.  One is block distributed, the other is cyclic
distributed.  There are 100 different \texttt{Location}
items mapped by the range \texttt{x} from integers, for example,
the first one is
\small
\begin{verbatim}
Location i = x[0];
\end{verbatim}
\normalsize

\paragraph{Subgroup and Subrange.}

A {\em subgroup} is some slice of a process array, formed by
restricting the process coordinates in one or more dimensions to
single values. 

Suppose {\tt i} is a location in a range distributed over a dimension
of group {\tt p}. The expression
\small
\begin{verbatim}
p / i
\end{verbatim}
\normalsize
represents a smaller group---the slice of {\tt p} to which location
{\tt i} is mapped.  

Similarly, a {\em subrange} is a section of a range, parameterized by
a global index {\em triplet}. Logically, it represents a subset of the
locations of the original range.

The syntax for a subrange expression is
\small
\begin{verbatim}
x [ 1 : 49 ]
\end{verbatim}
\normalsize
The symbol ``\texttt{:}'' is a special separator. It is used to compose
a \emph{triplet} expression with optional \texttt{int} expressions to
represent an integer subset. The default initial and final values are
respectively zero and the extent of the range.  The default stride size
is 1.

\paragraph{Structured SPMD programming.}
\label{SPMD}

When a process group is defined, a set of ranges and
locations are also implicitly defined, as shown in figure \ref{fig:group}.
\begin{figure}[htbp]
  \begin{center}
    \leavevmode
%    \putfig{group.eps}{1.00}
    \epsfig{figure=group.eps}
    \caption{Structured processes group}
    \label{fig:group}
  \end{center}
\end{figure}
The two (primitive) ranges associated with dimensions of the group
\texttt{p} are,
\small
\begin{verbatim}
Range u=p.dim(0);  
Range v=p.dim(1);
\end{verbatim}
\normalsize
\texttt{dim()} is a member function that returns a range reference,
directly representing a processor dimension.

We can obtain a location in range \texttt{v}, and use it
to create a new group,
\small
\begin{verbatim}
Location j = v[1];
Group q = p/j;
\end{verbatim}
\normalsize
As shown in the figure, group \texttt{p} is highly \emph{structured},
The notions introduced around it contribute to program execution
control in the new programming language.

In a traditional SPMD program, execution control is based on
\texttt{if} statements and process id or rank numbers. In the new
programming language, switching execution control is based on the
structured process group.  For example, it is not difficult to guess
that the following code:
\small
\begin{verbatim}
on(p) {
  ...
}
\end{verbatim}
\normalsize
will restrict the execution control inside the bracket to processes in
group \texttt{p}.

The language also provided well-defined constructs to split
execution control across processes according to data items we want to
access.  This will be discussed later.

\subsection{Global variables}

When an SPMD program starts on a group of $n$ processes, there will be
$n$ control threads mapped to $n$ physical processors.
In each control thread, the program can define variables in the same
way as in a sequential program.  The variables created in this way are
\emph{local variables}.  Their {\em names} may be replicated across
processes, but they will be accessed individually (their scope is
local to a process).

Besides local variables, HPJava allows a program to define \emph{global
variables}, explicitly mapped to a process group.  A global variable
will be treated by the process group that creating it as a single
entity.
The language has special syntax for the definition of global data.
Global variables are all defined by using the \texttt{new}
operator from free storage.  When a global variable is created, a
\emph{data descriptor} is also allocated to describe where the data are
held.


\paragraph{Data descriptor and global data.}

The concept of data descriptor is not new. It exists in the Java
language itself.  For example, the field \texttt{length} in the Java
array reflects the fact that an array is accessed through a data
descriptor.

On a single processor, an array variable might be parametrized by a
simple record containing a memory address and an \texttt{int} value
for the the length.  On a multi-processor, a more complicated structure
is needed to describe a distributed array.  The data descriptor specifies
where the data is created, and how are they are distributed.  The
logical structure of a descriptor is shown in figure
\ref{fig:descriptor}.

\begin{figure}[htbp]
  \begin{center}
    \leavevmode
%    \putfig{descriptor.eps}{1.00}
    \epsfig{figure=descriptor.eps}
    \caption{Descriptor}
    \label{fig:descriptor}
  \end{center}
\end{figure}

New syntax is added in HPJava to define data with descriptors.
\small
\begin{verbatim}
on(p)
  int # s = new int #;
\end{verbatim}
\normalsize
creates a global scalar on the current executing process group. In the
statement, \texttt{s} is a data descriptor handle, in HPJava term, a
\emph{global scalar reference}.  The scalar contains an integer value.
Global scalar references can be defined for any primitive type (or,
in principle, class type) of Java.
The symbol \texttt{\#} in the right hand side of the assignment
indicates a data descriptor is allocated as the scalar is created.

For a scalar variable, a field \texttt{value} is used to retrieve the
value.
\small
\begin{verbatim}
on(p) {
  int # s = new int #;
  s.value = 100;
}
\end{verbatim}
\normalsize
Figure \ref{fig:mapping} shows a possible memory mapping for this
scalar on different processes. 
\begin{figure}[htbp]
  \begin{center}
    \leavevmode
%    \putfig{map.eps}{1.00}
    \epsfig{figure=map.eps}
    \caption{Memory mapping}
    \label{fig:mapping}
  \end{center}
\end{figure}
Note, the value field of \texttt{s} is identical in each process in the
current executing processes.   Replicated {\em value} variables are
different from local variables with replicated {\em names}.  The
associated descriptors can be used to ensure the value is maintained
identically in each process, throughout program execution.

The group inside a descriptor is called the \emph{data owner group}, it
defines where the global values are held.
\small
\begin{verbatim}
on(p)
  int # s = new int # on q;
\end{verbatim}
\normalsize
will set data owner field in the descriptor to group \texttt{q}.  In
general this may be a subset of the default, \texttt{p} (the whole
of the current active process group).

%Note, the \texttt{on} clause only change the data owner, it does not
%change the \emph{scope} of the variable. The above code is similar but
%different from the following one.
%\small
%\begin{verbatim}
%on(p)
%  on(q) 
%    int # s = new int #;
%\end{verbatim}
%\normalsize

When defining a global array, it is not necessary to allocate a data
descriptor for each array element.  So the syntax for defining a global
array is not derived directly from the one for a scalar.  An array
can defined with different kinds of ranges introduced earlier.
Suppose we still have
\small
\begin{verbatim}
Range x = new BlockRange(100, p.dim(0)) ;
\end{verbatim}
\normalsize
and the process group defined in figure \ref{fig:group}, then
\small
\begin{verbatim}
on(q) 
  float [[ ]] a = new float [[x]];
\end{verbatim}
\normalsize
will create a global array with range \texttt{y} on group \texttt{q}.
Here \texttt{a} is a descriptor handle describing a one-dimensional
array of \texttt{float}. It is block distributed on group
\texttt{q}\footnote{The \texttt{on} clause restrict the data owner
group of the array to \texttt{q}. If group \texttt{p} is used instead,
the one dimenstion array will be replicated in the first dimenstion of
the group, and block distributed over the second dimension.}.  In
HPJava term, \texttt{a} is also called a \emph{global or distributed
array reference}.

A distributed array range can also be \emph{collapsed} (or
\emph{sequential}).  An integer range is specified, eg
\small
\begin{verbatim}
on(p)
  float [[*]] b = new float [[100]];
\end{verbatim}
\normalsize
When defining an array with collapsed dimensions, \texttt{*} can
optionally be added in a type signatures to mark these dimensions.

The typical method of accessing global array elements is not exactly
the same as for local array elements, or for global scalar references.
Since global arrays may have position information in their
dimensions, we often use locations as their subscripts:
\small
\begin{verbatim}
Location i=x[3];
at(i)
  a[i]=3;
\end{verbatim}
\normalsize
Here the fourth element of array \texttt{a} is assigned the value 3.
We will leave discussion of the \texttt{at} construct to section
\ref{control}, and give a simpler example here:
if a global array is defined with a collapsed dimension, accessing
its elements can be modelled on local arrays.  For example:
\small
\begin{verbatim}
for(int i=0; i<100; i++)
  b[i]=i;
\end{verbatim}
\normalsize
assigns the loop index to each corresponding element in the array.

When defining a multi-dimensional global array, a single descriptor
parametrizes a rectangular array of any dimensions.
\small
\begin{verbatim}
Range x = new BlockRange(100, p.dim(0)) ;  
Range y = new CyclicRange(100, p.dim(1)) ;
float [[,]] c = new float [[x, y]];
\end{verbatim}
\normalsize
This creates a two-dimension global array with the first dimension
block distributed and the second cyclic distributed.  Now \texttt{c} is a
global array reference.  Its elements can be accessed using
single brackets with two suitable locations inside.

The global array introduced here is a Fortran-style multi-dimensional
array, {\em not} a Java-like array-of-arrays.
%Hence the memory for the multi-dimension array is a continuous block,
%different from Java's memory scheme for arrays.
Java-style arrays-of-arrays are still useful. For example, one can
define a local array of distributed arrays:
\small
\begin{verbatim}
int[] size = {100, 200, 400};
float [[,]] d[] = 
        new float [size.length][[,]] ;
Range x[];
Range y[];
for (int l = 0; l < size.length; l++) {
  const int n = size [l] ;
  x[l] = new BlockRange(n, p.dim(0)) ;  
  y[l] = new BlockRange(n, p.dim(1)) ;  
  d[l] = new float [[x[l], y[l]]];
}
\end{verbatim}
\normalsize
creates an array shown in figure \ref{fig:layer}.

\begin{figure}[htbp]
  \begin{center}
    \leavevmode
%    \putfig{layer.eps}{1.00}
    \epsfig{figure=layer.eps}
    \caption{Array of distributed array}
    \label{fig:layer}
  \end{center}
\end{figure}

\paragraph{Array sections and type signatures.}

HPJava allows to construct {\em sections} of global arrays.  The syntax
of section subscripting uses double brackets.  The subscripts can be
scalar (integers or locations) or triplets.

Suppose we still have array \texttt{a} and \texttt{c} defined as
above, then, \texttt{a[[i]]}, \texttt{c}, \texttt{c[[i, 1::2]]},
and \texttt{c[[i, :]]} are all array sections.  Here {\tt i} is a
location in the first range of {\tt a} and {\tt c} (it could also
be an integer in the appropriate interval).
Both the expressions \texttt{c[[i, 1::2]]} and \texttt{c[[i, :]]}
represent one-dimensional distributed arrays, providing aliases for
subsets of the elements in \texttt{c}.  The expression \texttt{a[[i]]}
contains a single element of \texttt{a}, but the result is a global
scalar reference (unlike the expression \texttt{a[i]} which is a simple
variable).

Array section expressions are often used as arguments in function
calls\footnote{When used in method calls, the collapsed dimension
array is a \emph{subtype} of the ordinary one. i.e. an argument of
\texttt{float[[*,*]]}, \texttt{float[[*,]]} and \texttt{float[[,*]]}
type can all be passed to a dummy of type \texttt{float[[,]]}.
The converse is not true.}.  Table \ref{tab:section} shows the type
signatures of global data with different dimensions.
\begin{table}[htbp]
  \begin{center}
    \leavevmode
    \small
    \begin{tabular}[c]{l|ll} 
      \textbf{global var}& \textbf{array section} & \textbf{type} \\ 
      \hline
      2-dimension & \texttt{c} & \\
      & \texttt{c[[:,:]]} & \texttt{float [[,]]} \\ 
      \hline
      1-dimension & \texttt{c[[i,:]]} & \\
      & \texttt{c[[i,1::2]]} & \texttt{float [[ ]]} \\ 
      \hline
      scalar(0-dim) & \texttt{c[[i,j]]} & \texttt{float \#} \\ 
      \hline
    \end{tabular}
    \normalsize
    \caption{Section expression and type signature}
    \label{tab:section}
  \end{center}
\end{table}
In the table, both \texttt{i} and \texttt{j} are location
references.

\paragraph{Inquiry fields and functions}

The size of an array in Java can be had from its
\texttt{length} field.  Similarly, in HPJava, information like data
owner group and distributed dimensions can be accessed from the
following fields,
\small
\begin{verbatim}
Group group;   //data owner group
Range range[]; //dimension array
\end{verbatim}
\normalsize
Further inquiry functions on {\tt Range} yield values such as
extents and distribution formats.

%For an array with collapsed dimensions, collapsed ranges will be
%created for the corresponding dimensions.

\subsection{Program execution control}
\label{control}

HPJava has all the Java statements for execution control
within a single process.  It introduces three new control
constructs, \texttt{on}, \texttt{at} and \texttt{overall} for execution
control across processes.
A new concept, the \emph{active process group}, is introduced. It is the
set of processes sharing the current thread of control.

In a traditional SPMD program, switching the active
process group is effectively implemented by \texttt{if} statements
such as:
\small
\begin{verbatim}
if(myid>=0 && myid<4) {
  ...
}
\end{verbatim}
\normalsize
Inside the braces, only processes numbered 0 to 3
share the control thread.
In HPJava, this effect is expressed using a \texttt{Group}.
When a HPJava program starts, the active process group has a
system-defined value.  During the execution, the active process group
can be changed explicitly through an \texttt{on} construct in the
program.

In a shared memory program, accessing the value of a variable is
straightforward. In a message passing system, only the process which
holds data can read and write the data.  We sometimes call this {\em
SPMD constraint}.  A traditional SPMD program respects this constraint
by using an idiom like
\small
\begin{verbatim}
if(myid==1) 
  my_data=3;
\end{verbatim}
\normalsize
The \texttt{if} statement makes sure that only \texttt{my\_data}
on process 1 is assigned to.

In the language we present here similar constraints must be respected.
Besides \texttt{on} construct introduced earlier, there is a
convenient way to change the active process group to access a required
array element, namely \texttt{at} construct.
Suppose array \texttt{a} is defined as in the previous section, then:
\small
\begin{verbatim}
on (q) {
  Location i=x[1];
  at(i)
    a[i]=3; //correct

  a[i]=3;   //error
}
\end{verbatim}
\normalsize
The assignment statement guarded by an \texttt{at} construct is correct;
the one without it may cause run-time error.

A more powerful construct called \texttt{overall}
combines switching of the active process group with a loop:
\small
\begin{verbatim}
on(q)
  overall(i= x|0:3)
    a[i]=3;
\end{verbatim}
\normalsize
is essentially equivalent to\footnote{A compiler can implement
\texttt{overall} construct in a more efficient way, using linearized
address calculation. For detail definition on \texttt{overall}
construct, please refer to \cite{Car98}}
\small
\begin{verbatim}
on(q) 
  for(int n=0;n<4;n++)
    at(i=x[n])
      a[i]=3;
\end{verbatim}
\normalsize
In each iteration, the active process group is changed to
\texttt{q/i}.
In section \ref{examples}, we will illustrate with further programs how
\texttt{at} and \texttt{overall} constructs conveniently allow one to
keep the active process group equal to the data owner group for the
assigned data.

\subsection{Communication library functions}
\label{communication}

When accessing data on another process, HPJava needs explicit
communication, as in a ordinary SPMD program. 
Communication libraries are provided as packages in HPJava. Detailed
function specifications will be introduced in other papers. Here we
will only introduce a small number of top level collective
communication functions.

In the current design, the collective communications are member
functions of a static class \texttt{Adlib}.
\texttt{Adlib.remap} will copy the corresponding element from one to
another, regardless of their distribution format. \texttt{Adlib.shift}
will shift certain amount in a specific dimension of the array in
either cyclic or edge-off mode. \texttt{Adlib.writeHalo} is used to
support ghost regions.

It is possible to integrate other communication library as
communication packages of the language.  We have already implemented
a Java MPI interface. Currently CHAOS \cite{Das94} and GA \cite{Nie96}
are being considered as ``add-on'' packages.

