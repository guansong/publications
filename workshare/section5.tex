%\newpage

\section{Summary and future work}


In this paper, we listed important data structures needed in our
runtime library to support workshares in OpenMP standard, and the
corresponding algorithms to reduce the runtime overhead.

We explained that by carefully choosing the length of the workshare
control block queue, and reusing the queue whenever a barrier
synchronization occurred, we do not need to map control block
structures to memory for each workshare instance.  Also by preparing
the next workshare control block in advance, we do not need a separate
locking phase to initialize a control block, neither do we need to
initialize the whole queue at startup, which saves the parallel
region overhead for most of the real application cases. Although these
are just ``small techniques'', they do have significant impact on
performance benchmarks for runtime libraries.

We also introduced our view on OpenMP workshare, and algorithms to
implement barrier synchronization, which was considered as a special case of the
workshare. All these are commercial available in our XL Fortran and
VisualAge\textregistered\  C/C++ compilers. We are actually further improving the
barrier performance by taking advantage of more accurate cacheline
alignment of our internal data structures.


% how about irregular
