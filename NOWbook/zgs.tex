%Postscript encapsulation macro(s)
\input epsf
%\putfig{filespec.eps}{scale}
\def\putfig#1#2{\mbox{ \def\epsfsize##1##2{#2##1} \epsfbox{#1}}}

\chapter[The HPspmd Model and its Java Binding]{The {\em HPspmd} Model and its Java Binding}
\label{zgs:chap}
\author[Guansong Zhang, Bryan Carpenter, Geoffrey Fox,\\
  Xinying Li, and Yuhong Wen]{Guansong Zhang, Bryan Carpenter, Geoffrey Fox,\\
  Xinying Li, and Yuhong Wen}
\affil{NPAC at Syracuse University\\
        Syracuse, NY, 13 244 \\
	\vspace{0.1in}
        Email: {\em \{zgs,dbc,gcf,xli,wen\}@npac.syr.edu}}

%\noindent
%{\bf Abstract
%{\small\em
%This chapter introduces a new language, \emph{HPJava}, for parallel
%programming on message passing systems, including workstation
%clusters.  The language provides a high level SPMD programming model,
%called \emph{HPspmd}.  Through examples and performance results, the
%features of the new programming style, and its implementation, are
%illustrated.  
%} 
%}

%\vspace{0.5cm}

%\noindent
%{\it Keywords:}
%{\small Java, data parallel programming, SPMD}


%--------------------


\section{Introduction}

In this chapter we introduce the \inxx{HPJava} \emph{HPJava}
language, a programming language that extends Java for parallel
programming on message passing systems --- from multiprocessor systems to
workstation clusters.

HPJava owes much to High Performanace Fortran (HPF)
\cite{HPFStandard}.  Its model of data distribution is adapted directly
from the HPF model.  The heritage of HPF can be traced back to Fortran
dialects that were implemented most successfully on SIMD and other
tightly-coupled MPP architectures.  While it was always a goal of the
HPF designers that the language should be efficiently implementable on
the more loosely coupled MIMD clusters that dominate today, the
complexity of the language --- and notably the design goal of emulating
exactly the semantics of a sequential Fortran program --- have made
efficient implementation on today's architectures quite hard.

HPJava, in contrast, starts from the assumption that the target
hardware is a set of interacting MIMD processors, and exposes that
assumption explictly in its programming model.  This greatly
simplifies the task of the compiler, and increases the chance of
obtaining efficient implementations on architectures including PC and
workstation clusters.  Instead of the HPF programming model, the
language introduces a high-level structured SPMD programming
style --- the \inxx {HPspmd} HPspmd model.  A program written in
this class of language explicitly coordinates well-defined process
groups.  These cooperate in a loosely synchronous manner, sharing
logical threads of control.  As in a conventional distributed-memory
SPMD program, only a process owning a data item such as an array
element is allowed to access the item directly.  The language provides
special constructs that allow programmers to meet this constraint
conveniently.

Besides the normal variables of the sequential base language, the
language model introduces classes of global variables that are stored
collectively across process groups.  Primarily, these are
distributed arrays.  They provide a global name space in the form of
globally subscripted arrays, with assorted distribution patterns.  This
helps to relieve programmers of error-prone activities such as the
local-to-global, global-to-local subscript translations which occur in
data parallel applications.

In addition to special data types, the language provides special
constructs to facilitate both data parallel and task parallel
programming.  Through these constructs different processors can work either
simultaneously on globally addressed data, or independently to execute
complex procedures on locally held data.  The conversion between these
phases is seamless.

In the traditional SPMD mold, the language itself does not provide
implicit data movement semantics.  This greatly simplifies the task of
the compiler, and should encourage programmers to use algorithms that
exploit locality.  Data on remote processors is accessed exclusively
through explicit library calls.  In particular, the initial HPJava
implementation relies on a library of collective communication
routines originally developed as part of an HPF run-time library.
Other distributed-array-oriented communication libraries may be
bound to the language later.  Due to the explicit SPMD programming
model, low level MPI communication is always available as a fallback.
The language itself only provides basic concepts to organize data
arrays and process groups.  Different communication patterns are
implemented as library functions.  This allows the possibility that if
a new communication pattern is needed, it is relatively easily
integrated through new libraries.

In our earlier work on HPF compilation \cite{Zha97} the role of run-time
support was emphasized.  Difficulties in compiling HPF efficiently
suggested to make the run-time communication library directly visible in
the programming model.  Since Java is a simple, elegant language,
we are implementing our prototype based upon this language.

Section \ref{sec:javaBinding} reviews the HPspmd model in the context
of the HPJava language.  Section \ref{sec:package} describes the class
library packages used in code generated by the HPJava translator, and
thus exposes many of the implementation issues.  Some examples of simple
algorithms expressed in HPJava are given in Section
\ref{sec:examples}.  Then Section \ref{sec:design} discusses the
rationale of various design decisions in the language.  The status of
the project and future goals are summarized in Sections
\ref{sec:progress} and \ref{summary.tex}.



\section{Java Language Binding\label{sec:javaBinding}} 

This section introduces the HPJava language.
HPJava contains the whole of standard Java as a subset.  It adds
various built-in classes for describing process groups and index ranges,
new global data types, and some syntax for accessing distributed data
and specifying which processes execute particular statements.

\subsection{Basic Concepts}

Key concepts in the programming model are built around the 
\inxx{group} 
\inxx{group, process groups}
process groups used to describe program execution control in a parallel
program.
\texttt{Group} is a class representing a process group, typically with
a grid structure and an associated set of process dimensions.
It has its subclasses that represent different grid dimensionalities,
such as \texttt{Procs1}, \texttt{Procs2}, etc. For example,
\small
\begin{verbatim}
  Procs2 p = new Procs2(2,4);
\end{verbatim}
\normalsize
{\tt p} is a 2-dimensional, 2 x 4 grid of processes.

The second category of concepts is associated with 
\inxx{range}
\inxx{range, distributed ranges}
distributed ranges.
The elements of an ordinary array can be represented by an array name and
an integer sequence.  There are two parameters associated with this
sequence: an index to access each array element, and the extent of the
range this index can be chosen from.  In describing a distributed
array, HPJava introduces two new kinds of entity to represent the
analogous concepts.  A {\em range} maps an integer interval into a
process dimension according to certain distribution format. Ranges
describe the extent and the mapping of array dimensions.  A {\em
location}, or slot, is an abstract element of a range.
For example,
\small
\begin{verbatim}
  Range x = new BlockRange(100, p.dim(0)) ;
  Range y = new CyclicRange(200, p.dim(1)) ;
\end{verbatim}
\normalsize
creates two ranges distributed over the two process dimensions of the
group \texttt{p}.  One is block distributed, the other is cyclic
distributed.  There are 100 different locations in the range \texttt{x}.
The first one, for example, is
\small
\begin{verbatim}
  x [0]
\end{verbatim}
\normalsize

Additional related concepts are 
\inxx{group, subgroups}
subgroups
and 
\inxx{range, subranges}
subranges.  A subgroup is some slice of a
process array, formed by restricting the process coordinates in one or
more dimensions to single values.  Suppose {\tt i} is a location in a
range distributed over a dimension of group {\tt p}. The expression
\small
\begin{verbatim}
  p / i
\end{verbatim}
\normalsize
represents a smaller group---the slice of {\tt p} to which location
{\tt i} is mapped.  
Similarly, a subrange is a section of a range, parameterized by
a global index triplet. Logically, it represents a subset of the
locations of the original range.
The syntax for a subrange expression is
\small
\begin{verbatim}
  x [1 : 49]
\end{verbatim}
\normalsize
The symbol ``\texttt{:}'' is a special separator.  It is used to
construct a Fortran-90 style triplet, with lower- and upper-bound
expressions defining an integer subset.  The optional third member of a
triplet is a stride.

When a process grid is defined, certain ranges and locations are
also implicitly defined.  As shown in Figure \ref{fig:group},
two primitive ranges are associated with dimensions of the group
\texttt{p}:
\small
\begin{verbatim}
  Range u = p.dim(0);  
  Range v = p.dim(1);
\end{verbatim}
\normalsize
\texttt{dim()} is a member function that returns a range reference,
directly representing a processor dimension.
We can obtain a location in range \texttt{v}, and use it
to create a new group,
\small
\begin{verbatim}
  Group q = p / v [1] ;
\end{verbatim}
\normalsize
\begin{figure}[tbp]
  \begin{center}
    \leavevmode
    \psfig{figure=./zgs/group.eps,height=1.5in}
    \caption{Structured process group.}
    \label{fig:group}
  \end{center}
\end{figure}

In a traditional SPMD program, execution control is based on
{\em if} statements and process ID or rank numbers. In the new
programming language, switching execution control is based on the
structured process group.  For example, it is not difficult to guess
that the following code:
\small
\begin{verbatim}
  on(p) {
    ...
  }
\end{verbatim}
\normalsize
will restrict the execution control inside the bracket to processes in
group \texttt{p}.
The language also provided well-defined constructs to split
execution control across processes according to data items we want to
access.  This will be discussed later.

\subsection{Global Variables}

When an SPMD program starts on a group of $n$ processes, there will be
$n$ control threads mapped to $n$ physical processors.
In each control thread, the program can define variables in the same
way as in a sequential program.  The variables created in this way are
local variables.  Their names may be common to all
processes, but they will be accessed individually (their scope is
local to a process).

Besides local variables, HPJava allows a program to define
global variables, explicitly mapped to a process group.  A
global variable will be treated by the process group that created it
as a single entity. The language has special syntax for the definition
of global data.  Global variables are all defined by using the
\texttt{new} operator from free storage.  When a global variable is
created, a
\inxx{data descriptor} data descriptor 
is also allocated to describe where the data are
held.  On a single processor, an array variable might be parametrized
by a simple record containing a memory address and an \texttt{int}
value for its length.  On a multiprocessor, a more complicated
structure is needed to describe a distributed array.  The data
descriptor specifies where the data is created, and how they are
distributed.  The logical structure of a descriptor is shown in Figure
\ref{fig:descriptor}.

\begin{figure}[tbp]
  \begin{center}
    \leavevmode
    \psfig{figure=./zgs/descriptor.eps}
    \caption{Descriptor.}
    \label{fig:descriptor}
  \end{center}
\end{figure}

\begin{figure}[bp]
  \begin{center}
    \leavevmode
    \psfig{figure=./zgs/map.eps}
    \caption{Memory mapping.}
    \label{fig:mapping}
  \end{center}
\end{figure}

HPJava has special syntax to define global data.  The statement
\small
\begin{verbatim}
  int # s = new int # on p ;
\end{verbatim}
\normalsize
creates a global scalar replicated over process group {\tt p}. In the
statement, \texttt{s} is a data descriptor handle --- a
\inxx{global scalar reference} global scalar reference.  
The scalar contains an integer value.
Global scalar references can be defined for any primitive type (or,
in principle, class type) of Java.
The symbol \texttt{\#} in the type signature
distinguishes a global scalar from a primitive integer.
For a global scalar, a field \texttt{value} is used to access the
value:
\small
\begin{verbatim}
  on(p) {
    int # s = new int # ;
    s.value = 100 ;
  }
\end{verbatim}
\normalsize
Note how the \texttt{on} clause can be omitted from the constructor:
the whole of the active process group is the default distribution group.
Figure \ref{fig:mapping} shows a possible memory mapping for this
scalar on different processes. 
Note, the value field of \texttt{s} is identical in each process in the
distribution group.   Replicated value variables are
different from local variables with identical names.  The
associated descriptors can be used to ensure the value is maintained
identically in each process, throughout program execution.

When defining a global array, it is not necessary to allocate a data
descriptor for each array element---one descriptor suffices for the whole
array.
An array can defined with various kinds of range, introduced earlier.
Suppose we have, as before,
\small
\begin{verbatim}
  Range x = new BlockRange(100, p.dim(0)) ;
\end{verbatim}
\normalsize
and the process group defined in Figure \ref{fig:group}, then
\small
\begin{verbatim}
  float [[]] a = new float [[x]] on q ;
\end{verbatim}
\normalsize
will create a global array with range \texttt{x} on group \texttt{q}.
Here \texttt{a} is a descriptor handle describing a one-dimensional
array of \texttt{float}. It is block distributed on group
\texttt{q}.\footnote{The \texttt{on} clause restricts the data owner
group of the array to \texttt{q}. If group \texttt{p} is used instead,
the one-dimenstional array will be replicated in the first dimenstion of
the group, and block distributed over the second dimension.} In
HPJava \texttt{a} is called a 
\inxx{global array reference} global or distributed
array reference.

A distributed array range can also be collapsed (or
sequential).  An integer range is specified, eg
\small
\begin{verbatim}
  float [[*]] b = new float [[100]] ;
\end{verbatim}
\normalsize
When defining an array with collapsed dimensions an asterisk is
normally added in the type signatures to mark the collapsed
dimensions.

The typical method of accessing global array elements is not exactly
the same as for local array elements, or for global scalar references.
In distributed dimensions of arrays
we must use named locations as subscripts, for example
\small
\begin{verbatim}
  at(i = x [3])
    a [i] = 3 ;
\end{verbatim}
\normalsize
We will leave discussion of the {\em at} construct to Section
\ref{control}, and give a simpler example here:
if a global array is defined with a collapsed dimension, accessing
its elements is modelled on local arrays.  For example:
\small
\begin{verbatim}
  for(int i = 0 ; i < 100 ; i++)
    b [i] = i ;
\end{verbatim}
\normalsize
assigns the loop index to each corresponding element in the array.

When defining a multidimensional global array, a single descriptor
parametrizes a rectangular array of any dimension:
\small
\begin{verbatim}
  Range x = new BlockRange(100, p.dim(0)) ;  
  Range y = new CyclicRange(100, p.dim(1)) ;
  float [[,]] c = new float [[x, y]];
\end{verbatim}
\normalsize
This creates a two-dimension global array with the first dimension
block distributed and the second cyclic distributed.  Now \texttt{c} is a
global array reference.  Its elements can be accessed using
single brackets with two suitable locations inside.

The global array introduced here is a true multidimensional
array, not a Java-like array-of-arrays.
Java-style arrays-of-arrays are still useful. For example, one can
define a local array of distributed arrays:
\small
\begin{verbatim}
  int[] size = {100, 200, 400};
  float [[,]] d[] = new float [size.length][[,]] ;
  Range x[], y[];
  for (int l = 0; l < size.length; l++) {
    const int n = size [l] ;
    x[l] = new BlockRange(n, p.dim(0)) ;  
    y[l] = new BlockRange(n, p.dim(1)) ;  
    d[l] = new float [[x[l], y[l]]];
  }
\end{verbatim}
\normalsize
This creates the stack of distributed arrays shown in Figure \ref{fig:layer}.

\begin{figure}[tbp]
  \begin{center}
    \leavevmode
    \psfig{figure=./zgs/layer.eps}
    \caption{Array of distributed arrays.}
    \label{fig:layer}
  \end{center}
\end{figure}


Like Fortran 90, HPJava allows construction of sections of global
arrays.  The syntax of section subscripting uses double brackets.  The
subscripts can be scalar (integers or locations) or triplets.

Suppose we have array \texttt{a} and \texttt{c} defined as
above.  Then \texttt{a[[i]]}, \texttt{c}, \texttt{c[[i, 1::2]]},
and \texttt{c[[i, :]]} are all array sections.  Here, {\tt i} is an
integer in the appropriate interval (it could also be a
location in the first range of {\tt a} and {\tt c}).
Both the expressions \texttt{c[[i, 1::2]]} and \texttt{c[[i, :]]}
represent one-dimensional distributed arrays, providing aliases for
subsets of the elements in \texttt{c}.  The expression \texttt{a[[i]]}
contains a single element of \texttt{a}, but the result is a global
scalar reference (unlike the expression \texttt{a[i]} which is a simple
variable).

Array section expressions are often used as arguments in function
calls.\footnote{When used in method calls, the collapsed dimension
array is a \emph{subtype} of the ordinary one, i.e., an argument of
\texttt{float[[*,*]]}, \texttt{float[[*,]]} and \texttt{float[[,*]]}
type can all be passed to a dummy of type \texttt{float[[,]]}.
The converse is not true.}  Table \ref{tab:section} shows the type
signatures of global data with different dimensions.

\begin{table}[tbp]
  \begin{center}
    \caption{Section Expression and Type Signature}
    \vspace{0.1in}
    \leavevmode
    \small
    \begin{tabular}[c]{|l|ll|} 
      \hline
      \textbf{global var}& \textbf{array section} & \textbf{type} \\ 
      \hline
      2-dimension & \texttt{c} & \texttt{float [[,]]} \\ 
      & \texttt{c[[:,:]]} & \texttt{float [[,]]} \\ 
      \hline
      1-dimension & \texttt{c[[i,:]]} & \texttt{float [[]]} \\ 
      & \texttt{c[[i,1::2]]} & \texttt{float [[]]} \\ 
      \hline
      scalar(0-dim) & \texttt{c[[i,j]]} & \texttt{float \#} \\ 
      \hline
    \end{tabular}
    \normalsize
    \label{tab:section}
  \end{center}
\end{table}

The size of an array in Java can be had from its
\texttt{length} field.  In HPJava, information like the
distributed group and distributed dimensions can be accessed from the
following inquiries, available on all global array types:
\small
\begin{verbatim}
  Group grp()       // distribution group

  Range rng(int d)  // d'th range
\end{verbatim}
\normalsize
Further inquiry functions on {\tt Range} yield values such as
extents and distribution formats.

\subsection{Program Execution Control}
\label{control}

HPJava has all the conventional Java statements for execution control
within a single process.  It introduces three new control
constructs: {\em on}, {\em at}, and {\em overall} for execution
control across processes.
A new concept, the active process group, is introduced. It is the
set of processes sharing the current thread of control.

In a traditional SPMD program, switching the active
process group is effectively implemented by {\em if} statements
such as:
\small
\begin{verbatim}
  if(myid >= 0 && myid < 4) {
    ...
  }
\end{verbatim}
\normalsize
Inside the braces, only processes numbered 0 to 3
share the control thread.
In HPJava, this effect is expressed using a \texttt{Group}.
When a HPJava program starts, the active process group has a
system-defined value.  During the execution, the active process group
can be changed explicitly through an {\em on} construct in the
program.

In a shared memory program, accessing the value of a variable is
straightforward. In a message passing system, only the process which
holds data can read and write the data.  We sometimes call this 
\inxx{SPMD constraint} SPMD constraint.  
A traditional SPMD program respects this constraint by using an idiom
like
\small
\begin{verbatim}
  if(myid == 1) 
    my_data = 3 ;
\end{verbatim}
\normalsize
The {\em if} statement makes sure that only \texttt{my\_data}
on process 1 is assigned as 3.

In the language we present here, similar constraints must be respected.
Besides \emph{on} construct introduced earlier, there is a
convenient way to change the active process group to access a required
array element, namely, the {\em at} construct.
Suppose array \texttt{a} is defined as in the previous section, then:
\small
\begin{verbatim}
  on(q) {
    a [1] = 3 ;    // error

    at(j = x [1])
      a [j] = 3 ;  // correct
  }
\end{verbatim}
\normalsize
The assignment statement guarded by an {\em at} construct is
correct; the one without is likely to imply access to an element
not held locally.  Formally it is illegal because, in a simple
subscripting operation, an integer expression cannot
be used to subscript a distributed dimension.  The
{\em at} construct introduces a new variable \texttt{j}, a {\em named
location}, with scope only inside the block controlled by the
{\em at}.  Named locations are the only legal element subscripts 
in distributed dimensions.

A more powerful construct called {\em overall}
combines restriction of the active process group with a loop:
\small
\begin{verbatim}
  on(q)
    overall(i = x | 0 : 3)
      a [i] = 3 ;
\end{verbatim}
\normalsize
is essentially equivalent to\footnote{A compiler can implement
overall construct in a more efficient way, using linearized
address calculation.  For detailed translation schemes for the
overall construct, please refer to \cite{Car98}.}
\small
\begin{verbatim}
  on(q) 
    for(int n = 0 ; n < 4 ; n++)
      at(i = x [n])
        a [i] = 3;
\end{verbatim}
\normalsize
In each iteration, the active process group is changed to \texttt{q /
i}.  In Section \ref{sec:examples}, we will illustrate with further
programs how {\em at} and {\em overall} constructs conveniently
allow one to keep the active process group equal to the data owner
group for the assigned data.

\subsection{Communication Library Functions}
\label{communication}

When accessing data on another process, HPJava needs explicit
communication, as in a normal SPMD program. 
Communication libraries are provided as packages in HPJava. Detailed
function specifications are given elsewhere.  The next
section will introduce a small number of top level collective
communication functions.


\section{Java Packages for HPspmd Programming}
\label{sec:package}

The implementation of the HPJava compiler is based on a run-time system.
It is actually a source-to-source translator converting an HPJava
program to a Java node program, with function calls to the run-time
library, called \emph{adJava}.

The run-time interface consists of several Java packages. The most
important one is the HPspmd runtime proper.  It includes the classes
needed to translate language constructs.  Other packages provide
communication and some simple I/O functions.  Important classes in the
first package include distributed array ``container classes'' and
related classes describing process groups and index ranges.  These
classes correspond directly to HPJava built-in classes.

The first hierarchy is based on \texttt{Group}.  A group, or
process group, defines some subset of the processes executing the SPMD
program.  They can be used to describe how program variables,
such as arrays, are distributed or replicated across the process pool,
or to specify which subset of processes executes a particular code
fragment.  Important members of adJava {\tt Group} class include the
pair {\tt on()}, {\tt no()} used to translate the {\em on} construct.

The most common way to create a group object is through the
constructor for one of the subclasses representing a process
grid.  The subclass \texttt{Procs} represents a grid of processes
and carries information on process dimensions: in particular an
inquiry function {\tt dim(r)} returns a range object describing the
$r$-th process dimension.  {\tt Procs} is further subclassed by
\texttt{Procs0}, \texttt{Procs1}, \texttt{Procs2}, \ldots which
provide simpler constructors for fixed dimensionality process grids.

The second hierarchy in the package is based on \texttt{Range}.
A range is a map from the integer interval $0,\ldots,n-1$ into
some process dimension (i.e., some dimension of a process grid).
Ranges are used to parametrize distributed arrays and the
{\em overall} distributed loop.

The most common way to create a range object is to use the constructor
for one of the subclasses representing ranges with specific
distribution formats.  
Simple block distribution format is implemented by
{\tt BlockRange}, while \texttt{CyclicRange} and \texttt{BlockCyclicRange}
represent other standard distribution formats of HPF.  The subclass
\texttt{CollapsedRange} represents a sequential (undistributed range).
Finally, a \texttt{DimRange} is associated with each process dimension
and represents the range of coordinates for the process dimension
itself---just one element is mapped to each process.

The related adJava class \texttt{Location} represents an individual
location in a particular distributed range.  Important members of the
adJava {\tt Range} class include the function {\tt location(i)}, which
returns the $i$th location in a range, and its inverse, {\tt idx(l)},
which returns the global subscript associated with a given location.
Important members of the {\tt Location} class include {\tt at()} and
{\tt ta()}, used in the implementation of the HPJava that {\em at}
construct.

Finally, we have the rather complex hierarchy of classes representing
distributed arrays.  HPJava global arrays declared using
\texttt{[[]]} are represented by Java objects belonging to classes
such as:
\begin{small}
\begin{verbatim}
   Array1dI, Array1cI,
   Array2ddI, Array2dcI, Array2cdI, Array2ccI,
   ...
   Array1dF, Array1cF,
   Array2ddF, Array2dcF, Array2cdF, Array2ccF,
   ...
\end{verbatim}
\end{small}
Generally speaking, the class
``\texttt{Array}\textit{n}\texttt{c|d}\textit{{\ldots}T}''
represents $n$-dimensional distributed arrays with elements of type
{\em T}---currently one of {\tt I}, {\tt F}, \ldots, meaning {\tt int},
{\tt float}, \ldots. \footnote{In the inital implementation, the element
  type is restricted to the Java primitive types.}  The penultimate
part of the class name is a string of $n$ ``c''s and ``d''s specifying
whether each dimension is collapsed or normally distributed.  These
correlate with presence or absence of an asterisk in slots of the
HPJava type signature.  The concrete {\tt Array\ldots} classes
implement a series of abstract interfaces.  These follow a similar
naming convention, but the root of their names is {\tt Section} rather
than {\tt Array} (so {\tt Array2dcI}, for example, implements {\tt
Section2dcI}).  The hierarchy of {\tt Section} interfaces is
illustrated in Figure \ref{fig:arrays}.
The need to introduce the {\tt Section} interfaces should be evident
from the hierarchy diagram.  The type hierarchy of HPJava involves a
kind of multiple inheritance.  The array type {\tt int [[*, *]]}, for
example, is a specialization of both the types {\tt int [[*, ]]}
and {\tt int [[, *]]}.  Java allows ``multiple inheritance'' only from
interfaces, not classes.

\begin{figure}[tbp]
  \begin{center}
    \leavevmode
    \psfig{figure=./zgs/arrays.eps,height=2.4in} 
    \caption{The adJava {\tt Section} hierarchy}
    \label{fig:arrays}
  \end{center}
\end{figure}

Important members of the {\tt Section} interfaces include
inquiry functions {\tt dat()}, which returns an ordinary one
dimensional Java array used to store the locally held elements of the
distributed array, and the member {\tt pos(i, ...)}, which takes $n$
subscript arguments and returns the local offset of the element implied by
those subscripts.  Each argument of {\tt pos} is a location or an
integer (only allowed if the corresponding dimension is collapsed).
These functions are used to implement elemental subscripting.
The inquiry {\tt grp()} returns the group over which
elements of the array are distributed.  The inquiry {\tt rng(d)}
returns the $d$th range of the array.

Another package in adJava is the communication library.
The adJava communication package includes classes corresponding to the
various collective communication schedules provided in the NPAC PCRC
kernel.  Most of them provide a constructor to establish a schedule,
and an \texttt{execute} method, which carries out the data movement
specified by the schedule.  Different
communication models may eventually be added through further packages.

The collective communication schedules can be used directly by
the programmer or invoked through certain wrapper functions.
A class named \texttt{Adlib} is defined
with static members that create and execute
communication schedules and perform simple I/O functions. 
This class includes, for example, the following
methods, each implemented by
constructing the appropriate schedule and then executing it:
\begin{small}
\begin{verbatim}
  static void remap(Section dst, Section src)
  static void shift(Section dst, Section src, int shift, int dim, int mode)
  static void copy(Section dst, Section src)
  static void writeHalo(Section src, int [] wlo, int [] whi, int [] mode)
\end{verbatim}
\end{small}
\texttt{Adlib.remap} will copy the
corresponding elements from one array to another, regardless of their
respective distribution format.  \texttt{Adlib.shift} will shift data
by a certain amount in a specific dimension of the array, in either
cyclic or edge-off mode.  \texttt{Adlib.writeHalo} is used to update
ghost regions.

Given the classes described above, one can program in the HPspmd style
in pure Java program.  Then the idea of the HPJava compiler is just to
translate the HPJava program onto this interface, in the meanwhile, to use
optimized address calculation instead of function calls to access
elements of arrays. (For a detailed translation scheme, please refer to
\cite{Zha98}.)


\section{Programming Examples}
\label{sec:examples}

\begin{figure}[tbp]
  \begin{center}
\begin{small}
\begin{verbatim}
  Procs1 p = new Procs1(4);
  on(p) {
    Range x = new CyclicRange(n, p.dim(0));
    float a[[*,]] = new float [[n, x]];
    ... some code to initialise `a' ...
    float b[[*]] = new float [[n]]; // buffer

    for(int k = 0 ; k < N - 1 ; k++) {
      at(l = x[k]) {
        float d =  Math.sqrt(a[k,l]) ;
        a[k,l] = d ;
        for(int s = k + 1 ; s < N ; s++)
          a[s,l] /= d ;
      }
      Adlib.remap(b[[k+1:]], a[[k+1:,  k]]);

      overall(m = x | k + 1 : )
        for(int i = x.idx(m) ; i < N ; i++)
          a[i,m] -= b[i] * b[x.idx(m)] ;
    }
    at(l = x [N - 1])
      a[N - 1,l] = Math.sqrt(a[N-1,l]) ;
  }
\end{verbatim}
\end{small}
    \caption{Choleski decomposition.}
    \label{fig:choleski}
  \end{center}
\end{figure}

In this section we give two example programs to show the new
language features.  The first example is the Choleski decomposition, seen in
Figure \ref{fig:choleski}.
Here, \texttt{remap} is used to broadcast one updated column to each
process. The function \texttt{idx} gets the global index of location
\texttt{m} relative to the parent range \texttt{x}.
The second example is the Jacobi iteration, seen in Figure \ref{fig:Jacobi}.
In the displayed code there is only one iteration, but it demonstrates
how to define range references with ghost areas, how to use
the \texttt{writeHalo} function, and how to use shifted locations
as subscripts. 


\begin{figure}[tbp]
  \begin{center}
\begin{small}
\begin{verbatim}
  Procs2 p = new Procs2(2, 4);
  Range x = new BlockRange(100, p.dim(0), 1),
        y = new BlockRange(200, p.dim(1), 1); 
  on(p) {
    float [[,]] a = new float [[x,y]], b = new float [[x,y]];
    ... some code to initialize `a'

    Adlib.writeHalo(a);

    overall(i = x | : )
      overall(j = y | : )
        b[i,j] = 0.25 * (a[i-1,j] + a[i+1,j] + a[i,j-1] + a[i,j+1]);
    overall(i = x | : )
      overall(j = y | : )
        a[i,j] = b[i,j];
  }
\end{verbatim}
\end{small}
    \caption{Jacobi relaxation.}
    \label{fig:Jacobi}
  \end{center}
\end{figure}


\section{Issues in the Language Design}
\label{sec:design}

With some of the implementation mechanisms exposed, we can
better discuss the language design itself.

\subsection{Extending the Java Language}

The first question to answer is why use Java as a base language?
Actually, the programming model embodied in HPJava is largely language-independent.  
It can be bound to other languages like C, C++, and Fortran.
But Java is a convenient base language, especially for initial
experiments, because it provides full object-orientation --- convenient
for describing complex distributed data --- implemented in a relatively
simple setting, conducive to development of source-to-source
translators.  It has been noted elsewhere that Java has various
features suggesting it could be an attractive language for science
and engineering \cite{Java98}.

With Java as a base language, an obvious question is whether we can
extend the language by simply adding packages, instead of changing the
syntax.  There are two problems with doing this for data-parallel
programming.

Our baseline is HPF, and any package supporting parallel arrays as
general as HPF is likely to be cumbersome to code with.  Our run-time
system needs an (in principle) infinite series of class names
\begin{small}
\begin{verbatim}
   Array1dI, Array1cI, Array2ddI, Array2dcI, ...
\end{verbatim}
\end{small}
to express the HPJava types
\begin{small}
\begin{verbatim}
  int [[]], int [[*]], int [[,]], int [[,*]] ... 
\end{verbatim}
\end{small}
as well as the corresponding series for \texttt{char}, \texttt{float}, and so on.
To access an element of a distributed array in HPJava, one writes
\begin{small}
\begin{verbatim}
  a[i] = 3 ;
\end{verbatim}
\end{small}
In the adJava interface, it must be written as
\begin{small}
\begin{verbatim}
  a.dat()[a.pos(i)] = 3 ;
\end{verbatim}
\end{small}
This is only for simple subscripting. Constructing array
sections will be even more complex using the raw class library
interface.

The second problem is that a Java program using a package like adJava
in a direct, naive way will have very poor performance, because all
the local address of the global array are expressed by functions such
as \texttt{pos}.  An optimization pass is needed to transform offset
computation to a more intelligent style.  So if a preprocessor must do
these optimizations anyway, it makes sense to design a syntax to
express the concepts of the programming model more naturally.

\subsection{Why not HPF?}

The design of the HPJava language is strongly influenced by HPF.
The language emerged partly out of practices adopted in our efforts to
implement an HPF compilation system \cite{Zha97}.  For example:
\begin{small}
\begin{verbatim}
  !HPF$ POCESSOR    P(4)
  !HPF$ TEMPLET     T(100)
  !HPF$ DISTRIBUTE  T(BLOCK) ONTO P
        REAL        A(100,100), B(100)
  !HPF$ ALIGN       A(:,*) WITH T(:)
  !HPF$ ALIGN       B WITH T
\end{verbatim}
\end{small}
have their conterparts in HPJava:
\begin{small}
\begin{verbatim}
  Procs1 p = new Procs1(4);
  Range x = new BlockRange (100, p.dim(0));
  float [[,*]] a = new float [[x,100]] on p;
  float [[ ]] b = new float [[x]] on p;
\end{verbatim}
\end{small}
Both languages provide a globally addressed name space for data parallel
applications.  Both of them can specify how data are mapped on to a
processor grid.  The difference between the two lies in their
communication aspects.  In HPF, a simple assignment statement may cause data
movement.  For example, given the above distribution, the assignment
\begin{small}
\begin{verbatim}
  A(10,10) = B(30)
\end{verbatim}
\end{small}
will cause communication between processor 1 and 2.  In HPJava, similar
communication must be done through explicit function calls: \footnote{By default
Fortran array subscripts starts from 1, while HPJava global subscripts always
start from 0.}
\begin{small}
\begin{verbatim}
  Adlib.remap(a[[9,9]], b[[29]]);
\end{verbatim}
\end{small}
Experience from compiling the HPF language suggests that, while there
are various kinds of algorithms to detect communication automatically,
it is often difficult to give the generated node program acceptable
performance.  In HPF, the need to decide on which processor the
computation should be executed further complicates the situation.  One
may apply ``owner computes'' or ``majority computes'' rules to
partition computation, but these heuristics are difficult to apply in
many situations.

In HPJava, the SPMD programming model is emphasized.  The distributed
arrays just help the programmer organize data, and simplify
global-to-local address translation.  The tasks of computation
partition and communication are still under control of the programmer.
This is certainly an extra onus, and the language may be more difficult to
program than HPF; \footnote{The program must meet SPMD constraints, e.g.,
only the owner of an element can access that data.  Run-time checking
can be added automatically to ensure such conditions are met.} but
it helps programmers to understand the performance of the program much
better than in HPF, so algorithms exploiting locality and parallelism
are encouraged.  It also dramatically simplifies the work of the
compiler.

Because the communication sector is considered an ``add-on'' to the
basic language, HPJava should interoperate more smoothly than HPF with
other successful SPMD libraries, including MPI \cite{MPIStandard},
Global Arrays \cite{Nie96}, CHAOS \cite{Das94},
%DAGH \cite{HDDA_DAGH}, 
and so on.

\subsection{Datatypes in HPJava}

In a parallel language, it is desirable to have both local variables
(like the ones in MPI programming) and global variables (like
the ones in HPF programming).  The former provide flexibility and are
ideal for task parallel programming; the latter are convenient
especially for data parallel programming.

In HPJava, variable names are divided into two sets.  In general those
declared using ordinary Java syntax represent local variables and those
declared with \texttt{\#} or \texttt{[[]]} represent global variables.
The two sectors are independent.  In the implementation of HPJava the
global variables have special data descriptors associated with them,
defining how their components are divided or replicated across
processes.  The significance of the data descriptor is most obvious
when dealing with procedure calls.
Passing array sections to procedure calls is an important component in
the array processing facilities of Fortran90 \cite{Fortran90}.  The data
descriptor of Fortran90 will include stride information for each array
dimension.  One can assume that HPF needs a much more complex kind of
data descriptor to allow passing distributed arrays across procedure
boundaries.  In either case the descriptor is not visible to the
programmer.  Java has a more explicit data descriptor concept; its
arrays are considered as objects, with, for example, a publicly
accessible \texttt{length} field.  In HPJava, the data descriptors for
global data are similar to those used in HPF, but more explicitly
exposed to programmers.  Inquiry functions such as \texttt{grp} and
\texttt{rng} have a similar role in global data as the field
\texttt{length} in an ordinary Java array.

Keeping two data sectors seems to complicate the language and its
syntax.  But it provides convenience for both task and data parallel
processing.  There is no need for things like the \texttt{LOCAL}
mechanism in HPF to call a local procedure on the node processor.  The
descriptors for ordinary Java variables are unchanged in HPJava.  On
each node processor ordinary Java data will be used as local varables,
as in an MPI program.


\section{Projects in Progress\label{sec:progress}}

Projects related to this work include development of MPI, HPF,
and other parallel languages such as ZPL\cite{ZPL} and Spar\cite{Spar}.
Here, we explain the background and future developments of our own
project.

The work originated in our compilation practices for HPF. As described
in \cite{Zha97}, our compiler emphasizes run-time support.
Adlib\cite{Car98}, a PCRC run-time library, provides a
rich set of collective communication functions.  It was realized that
by raising the run-time interface to the user level, a rather
straightforward (compared to HPF) compiler could be developed to
translate the high level language code to a node program calling the
run-time functions.

Currently, a Java interface has been implemented on top of the Adlib
library.  With classes such as \texttt{Group}, \texttt{Range}, and
\texttt{Location} in the Java interface, one can write Java programs
quite similar to HPJava proposed here. But a program executed in
this way will have large overhead due to function calls (such as
address translation) when accessing data inside loop constructs.
Given the knowledge of data distribution plus inquiry functions inside
the run-time library, one can substitute address translation calls with
linear operations on the loop variable, and keep inquiry
function calls ``outside the loop.'' 

At the present time, we are implementing the translator.  Further
research work will include optimization and safety-checking techniques
in the compiler for HPspmd programming.

Figure \ref{fig:performance} shows a preliminary benchmark for a hand-translated 
versions of our examples. The parallel programs are executed
on 4 sparc-sun-solaris2.5.1 with MPICH and the Java JIT compiler in
JDK 1.2Beta2. For Jacobi iteration, the timing is for about 90
iterations; the array size is 1024X1024.

\begin{figure}[tbp]
  \begin{center}
    \leavevmode
    \psfig{figure=./zgs/Jacobi4.eps, height=1.8in}
    \caption{Preliminary performance.}
    \label{fig:performance}
  \end{center}
\end{figure}

We also compared the sequential Java, C++, and Fortran version of the
code, all with \texttt{-O} flag when compiling. The dotted lines shown
in the figure only represent times for the one processor case. We can
see that on a single processor, Java program use language supported
mechanism for calculating array element address; hence, it is slower than
HPJava (as HPJava uses an optimized address calculation scheme). We emphasize again that in the
picture we are comparing {\em sequential} Fortran, etc., with {\em
parallel} HPJava.  This is not supposed to be an comparative evaluation
of the various languages.  It is just supposed to give an impression of
the performance ballpark Java is currently operating in.

\section{Summary\label{summary.tex}}

Through the simple examples in this chapter we have tried to
illustrate that the programming language presented here provides the
flexibility of SPMD programming, and much of the convenience of HPF.
The language helps programmers to express parallel algorithms in a
more explicit way.  We suggest it will help programmers to solve real
application problems more easily, compared with using communication
packages such as MPI directly, and allow the compiler writer to
implement the language compiler without the difficulties met in the HPF
compilation.

The overall structure of the system is shown in Figure
\ref{fig:hpspmd}.  The central DAD stands for the Distributed Array
Descriptor.  Around it, we have different run-time libraries. The Java
interface is most relevant here.  But the Java binding is only an
introduction of the programming style.  (A Fortran binding is being
developed.)  Initially, the Java version can be used as a software tool
for teaching parallel programming.  As Java for scientific computation
becomes more mature, it will be a practical programming language to
solve real application problems in parallel and distributed
environments.

\begin{figure}[tbp]
  \begin{center}
    \leavevmode
    \psfig{figure=./zgs/HPspmdLayers.eps} 
    \caption{Layers of the HPspmd model.}
    \label{fig:hpspmd}
  \end{center}
\end{figure}


%--------------------

%\bibliographystyle{unsrt}
%\bibliographystyle{plain}
%\bibliography{zgs}

%%Postscript encapsulation macro(s)
\input epsf
%\putfig{filespec.eps}{scale}
\def\putfig#1#2{\mbox{ \def\epsfsize##1##2{#2##1} \epsfbox{#1}}}

\chapter[The HPspmd Model and its Java Binding]{The {\em HPspmd} Model and its Java Binding}
\label{zgs:chap}
\author[Guansong Zhang, Bryan Carpenter, Geoffrey Fox,\\
  Xinying Li, and Yuhong Wen]{Guansong Zhang, Bryan Carpenter, Geoffrey Fox,\\
  Xinying Li, and Yuhong Wen}
\affil{NPAC at Syracuse University\\
        Syracuse, NY, 13 244 \\
	\vspace{0.1in}
        Email: {\em \{zgs,dbc,gcf,xli,wen\}@npac.syr.edu}}

%\noindent
%{\bf Abstract
%{\small\em
%This chapter introduces a new language, \emph{HPJava}, for parallel
%programming on message passing systems, including workstation
%clusters.  The language provides a high level SPMD programming model,
%called \emph{HPspmd}.  Through examples and performance results, the
%features of the new programming style, and its implementation, are
%illustrated.  
%} 
%}

%\vspace{0.5cm}

%\noindent
%{\it Keywords:}
%{\small Java, data parallel programming, SPMD}


%--------------------


\section{Introduction}

In this chapter we introduce the \inxx{HPJava} \emph{HPJava}
language, a programming language that extends Java for parallel
programming on message passing systems --- from multiprocessor systems to
workstation clusters.

HPJava owes much to High Performanace Fortran (HPF)
\cite{HPFStandard}.  Its model of data distribution is adapted directly
from the HPF model.  The heritage of HPF can be traced back to Fortran
dialects that were implemented most successfully on SIMD and other
tightly-coupled MPP architectures.  While it was always a goal of the
HPF designers that the language should be efficiently implementable on
the more loosely coupled MIMD clusters that dominate today, the
complexity of the language --- and notably the design goal of emulating
exactly the semantics of a sequential Fortran program --- have made
efficient implementation on today's architectures quite hard.

HPJava, in contrast, starts from the assumption that the target
hardware is a set of interacting MIMD processors, and exposes that
assumption explictly in its programming model.  This greatly
simplifies the task of the compiler, and increases the chance of
obtaining efficient implementations on architectures including PC and
workstation clusters.  Instead of the HPF programming model, the
language introduces a high-level structured SPMD programming
style --- the \inxx {HPspmd} HPspmd model.  A program written in
this class of language explicitly coordinates well-defined process
groups.  These cooperate in a loosely synchronous manner, sharing
logical threads of control.  As in a conventional distributed-memory
SPMD program, only a process owning a data item such as an array
element is allowed to access the item directly.  The language provides
special constructs that allow programmers to meet this constraint
conveniently.

Besides the normal variables of the sequential base language, the
language model introduces classes of global variables that are stored
collectively across process groups.  Primarily, these are
distributed arrays.  They provide a global name space in the form of
globally subscripted arrays, with assorted distribution patterns.  This
helps to relieve programmers of error-prone activities such as the
local-to-global, global-to-local subscript translations which occur in
data parallel applications.

In addition to special data types, the language provides special
constructs to facilitate both data parallel and task parallel
programming.  Through these constructs different processors can work either
simultaneously on globally addressed data, or independently to execute
complex procedures on locally held data.  The conversion between these
phases is seamless.

In the traditional SPMD mold, the language itself does not provide
implicit data movement semantics.  This greatly simplifies the task of
the compiler, and should encourage programmers to use algorithms that
exploit locality.  Data on remote processors is accessed exclusively
through explicit library calls.  In particular, the initial HPJava
implementation relies on a library of collective communication
routines originally developed as part of an HPF run-time library.
Other distributed-array-oriented communication libraries may be
bound to the language later.  Due to the explicit SPMD programming
model, low level MPI communication is always available as a fallback.
The language itself only provides basic concepts to organize data
arrays and process groups.  Different communication patterns are
implemented as library functions.  This allows the possibility that if
a new communication pattern is needed, it is relatively easily
integrated through new libraries.

In our earlier work on HPF compilation \cite{Zha97} the role of run-time
support was emphasized.  Difficulties in compiling HPF efficiently
suggested to make the run-time communication library directly visible in
the programming model.  Since Java is a simple, elegant language,
we are implementing our prototype based upon this language.

Section \ref{sec:javaBinding} reviews the HPspmd model in the context
of the HPJava language.  Section \ref{sec:package} describes the class
library packages used in code generated by the HPJava translator, and
thus exposes many of the implementation issues.  Some examples of simple
algorithms expressed in HPJava are given in Section
\ref{sec:examples}.  Then Section \ref{sec:design} discusses the
rationale of various design decisions in the language.  The status of
the project and future goals are summarized in Sections
\ref{sec:progress} and \ref{summary.tex}.



\section{Java Language Binding\label{sec:javaBinding}} 

This section introduces the HPJava language.
HPJava contains the whole of standard Java as a subset.  It adds
various built-in classes for describing process groups and index ranges,
new global data types, and some syntax for accessing distributed data
and specifying which processes execute particular statements.

\subsection{Basic Concepts}

Key concepts in the programming model are built around the 
\inxx{group} 
\inxx{group, process groups}
process groups used to describe program execution control in a parallel
program.
\texttt{Group} is a class representing a process group, typically with
a grid structure and an associated set of process dimensions.
It has its subclasses that represent different grid dimensionalities,
such as \texttt{Procs1}, \texttt{Procs2}, etc. For example,
\small
\begin{verbatim}
  Procs2 p = new Procs2(2,4);
\end{verbatim}
\normalsize
{\tt p} is a 2-dimensional, 2 x 4 grid of processes.

The second category of concepts is associated with 
\inxx{range}
\inxx{range, distributed ranges}
distributed ranges.
The elements of an ordinary array can be represented by an array name and
an integer sequence.  There are two parameters associated with this
sequence: an index to access each array element, and the extent of the
range this index can be chosen from.  In describing a distributed
array, HPJava introduces two new kinds of entity to represent the
analogous concepts.  A {\em range} maps an integer interval into a
process dimension according to certain distribution format. Ranges
describe the extent and the mapping of array dimensions.  A {\em
location}, or slot, is an abstract element of a range.
For example,
\small
\begin{verbatim}
  Range x = new BlockRange(100, p.dim(0)) ;
  Range y = new CyclicRange(200, p.dim(1)) ;
\end{verbatim}
\normalsize
creates two ranges distributed over the two process dimensions of the
group \texttt{p}.  One is block distributed, the other is cyclic
distributed.  There are 100 different locations in the range \texttt{x}.
The first one, for example, is
\small
\begin{verbatim}
  x [0]
\end{verbatim}
\normalsize

Additional related concepts are 
\inxx{group, subgroups}
subgroups
and 
\inxx{range, subranges}
subranges.  A subgroup is some slice of a
process array, formed by restricting the process coordinates in one or
more dimensions to single values.  Suppose {\tt i} is a location in a
range distributed over a dimension of group {\tt p}. The expression
\small
\begin{verbatim}
  p / i
\end{verbatim}
\normalsize
represents a smaller group---the slice of {\tt p} to which location
{\tt i} is mapped.  
Similarly, a subrange is a section of a range, parameterized by
a global index triplet. Logically, it represents a subset of the
locations of the original range.
The syntax for a subrange expression is
\small
\begin{verbatim}
  x [1 : 49]
\end{verbatim}
\normalsize
The symbol ``\texttt{:}'' is a special separator.  It is used to
construct a Fortran-90 style triplet, with lower- and upper-bound
expressions defining an integer subset.  The optional third member of a
triplet is a stride.

When a process grid is defined, certain ranges and locations are
also implicitly defined.  As shown in Figure \ref{fig:group},
two primitive ranges are associated with dimensions of the group
\texttt{p}:
\small
\begin{verbatim}
  Range u = p.dim(0);  
  Range v = p.dim(1);
\end{verbatim}
\normalsize
\texttt{dim()} is a member function that returns a range reference,
directly representing a processor dimension.
We can obtain a location in range \texttt{v}, and use it
to create a new group,
\small
\begin{verbatim}
  Group q = p / v [1] ;
\end{verbatim}
\normalsize
\begin{figure}[tbp]
  \begin{center}
    \leavevmode
    \epsfig{figure=./zgs/group.eps,height=1.5in}
    \caption{Structured process group.}
    \label{fig:group}
  \end{center}
\end{figure}

In a traditional SPMD program, execution control is based on
{\em if} statements and process ID or rank numbers. In the new
programming language, switching execution control is based on the
structured process group.  For example, it is not difficult to guess
that the following code:
\small
\begin{verbatim}
  on(p) {
    ...
  }
\end{verbatim}
\normalsize
will restrict the execution control inside the bracket to processes in
group \texttt{p}.
The language also provided well-defined constructs to split
execution control across processes according to data items we want to
access.  This will be discussed later.

\subsection{Global Variables}

When an SPMD program starts on a group of $n$ processes, there will be
$n$ control threads mapped to $n$ physical processors.
In each control thread, the program can define variables in the same
way as in a sequential program.  The variables created in this way are
local variables.  Their names may be common to all
processes, but they will be accessed individually (their scope is
local to a process).

Besides local variables, HPJava allows a program to define
global variables, explicitly mapped to a process group.  A
global variable will be treated by the process group that created it
as a single entity. The language has special syntax for the definition
of global data.  Global variables are all defined by using the
\texttt{new} operator from free storage.  When a global variable is
created, a
\inxx{data descriptor} data descriptor 
is also allocated to describe where the data are
held.  On a single processor, an array variable might be parametrized
by a simple record containing a memory address and an \texttt{int}
value for its length.  On a multiprocessor, a more complicated
structure is needed to describe a distributed array.  The data
descriptor specifies where the data is created, and how they are
distributed.  The logical structure of a descriptor is shown in Figure
\ref{fig:descriptor}.

\begin{figure}[tbp]
  \begin{center}
    \leavevmode
    \epsfig{figure=./zgs/descriptor.eps}
    \caption{Descriptor.}
    \label{fig:descriptor}
  \end{center}
\end{figure}

\begin{figure}[bp]
  \begin{center}
    \leavevmode
    \epsfig{figure=./zgs/map.eps}
    \caption{Memory mapping.}
    \label{fig:mapping}
  \end{center}
\end{figure}

HPJava has special syntax to define global data.  The statement
\small
\begin{verbatim}
  int # s = new int # on p ;
\end{verbatim}
\normalsize
creates a global scalar replicated over process group {\tt p}. In the
statement, \texttt{s} is a data descriptor handle --- a
\inxx{global scalar reference} global scalar reference.  
The scalar contains an integer value.
Global scalar references can be defined for any primitive type (or,
in principle, class type) of Java.
The symbol \texttt{\#} in the type signature
distinguishes a global scalar from a primitive integer.
For a global scalar, a field \texttt{value} is used to access the
value:
\small
\begin{verbatim}
  on(p) {
    int # s = new int # ;
    s.value = 100 ;
  }
\end{verbatim}
\normalsize
Note how the \texttt{on} clause can be omitted from the constructor:
the whole of the active process group is the default distribution group.
Figure \ref{fig:mapping} shows a possible memory mapping for this
scalar on different processes. 
Note, the value field of \texttt{s} is identical in each process in the
distribution group.   Replicated value variables are
different from local variables with identical names.  The
associated descriptors can be used to ensure the value is maintained
identically in each process, throughout program execution.

When defining a global array, it is not necessary to allocate a data
descriptor for each array element---one descriptor suffices for the whole
array.
An array can defined with various kinds of range, introduced earlier.
Suppose we have, as before,
\small
\begin{verbatim}
  Range x = new BlockRange(100, p.dim(0)) ;
\end{verbatim}
\normalsize
and the process group defined in Figure \ref{fig:group}, then
\small
\begin{verbatim}
  float [[]] a = new float [[x]] on q ;
\end{verbatim}
\normalsize
will create a global array with range \texttt{x} on group \texttt{q}.
Here \texttt{a} is a descriptor handle describing a one-dimensional
array of \texttt{float}. It is block distributed on group
\texttt{q}.\footnote{The \texttt{on} clause restricts the data owner
group of the array to \texttt{q}. If group \texttt{p} is used instead,
the one-dimenstional array will be replicated in the first dimenstion of
the group, and block distributed over the second dimension.} In
HPJava \texttt{a} is called a 
\inxx{global array reference} global or distributed
array reference.

A distributed array range can also be collapsed (or
sequential).  An integer range is specified, eg
\small
\begin{verbatim}
  float [[*]] b = new float [[100]] ;
\end{verbatim}
\normalsize
When defining an array with collapsed dimensions an asterisk is
normally added in the type signatures to mark the collapsed
dimensions.

The typical method of accessing global array elements is not exactly
the same as for local array elements, or for global scalar references.
In distributed dimensions of arrays
we must use named locations as subscripts, for example
\small
\begin{verbatim}
  at(i = x [3])
    a [i] = 3 ;
\end{verbatim}
\normalsize
We will leave discussion of the {\em at} construct to Section
\ref{control}, and give a simpler example here:
if a global array is defined with a collapsed dimension, accessing
its elements is modelled on local arrays.  For example:
\small
\begin{verbatim}
  for(int i = 0 ; i < 100 ; i++)
    b [i] = i ;
\end{verbatim}
\normalsize
assigns the loop index to each corresponding element in the array.

When defining a multidimensional global array, a single descriptor
parametrizes a rectangular array of any dimension:
\small
\begin{verbatim}
  Range x = new BlockRange(100, p.dim(0)) ;  
  Range y = new CyclicRange(100, p.dim(1)) ;
  float [[,]] c = new float [[x, y]];
\end{verbatim}
\normalsize
This creates a two-dimension global array with the first dimension
block distributed and the second cyclic distributed.  Now \texttt{c} is a
global array reference.  Its elements can be accessed using
single brackets with two suitable locations inside.

The global array introduced here is a true multidimensional
array, not a Java-like array-of-arrays.
Java-style arrays-of-arrays are still useful. For example, one can
define a local array of distributed arrays:
\small
\begin{verbatim}
  int[] size = {100, 200, 400};
  float [[,]] d[] = new float [size.length][[,]] ;
  Range x[], y[];
  for (int l = 0; l < size.length; l++) {
    const int n = size [l] ;
    x[l] = new BlockRange(n, p.dim(0)) ;  
    y[l] = new BlockRange(n, p.dim(1)) ;  
    d[l] = new float [[x[l], y[l]]];
  }
\end{verbatim}
\normalsize
This creates the stack of distributed arrays shown in Figure \ref{fig:layer}.

\begin{figure}[tbp]
  \begin{center}
    \leavevmode
    \epsfig{figure=./zgs/layer.eps}
    \caption{Array of distributed arrays.}
    \label{fig:layer}
  \end{center}
\end{figure}


Like Fortran 90, HPJava allows construction of sections of global
arrays.  The syntax of section subscripting uses double brackets.  The
subscripts can be scalar (integers or locations) or triplets.

Suppose we have array \texttt{a} and \texttt{c} defined as
above.  Then \texttt{a[[i]]}, \texttt{c}, \texttt{c[[i, 1::2]]},
and \texttt{c[[i, :]]} are all array sections.  Here, {\tt i} is an
integer in the appropriate interval (it could also be a
location in the first range of {\tt a} and {\tt c}).
Both the expressions \texttt{c[[i, 1::2]]} and \texttt{c[[i, :]]}
represent one-dimensional distributed arrays, providing aliases for
subsets of the elements in \texttt{c}.  The expression \texttt{a[[i]]}
contains a single element of \texttt{a}, but the result is a global
scalar reference (unlike the expression \texttt{a[i]} which is a simple
variable).

Array section expressions are often used as arguments in function
calls.\footnote{When used in method calls, the collapsed dimension
array is a \emph{subtype} of the ordinary one, i.e., an argument of
\texttt{float[[*,*]]}, \texttt{float[[*,]]} and \texttt{float[[,*]]}
type can all be passed to a dummy of type \texttt{float[[,]]}.
The converse is not true.}  Table \ref{tab:section} shows the type
signatures of global data with different dimensions.

\begin{table}[tbp]
  \begin{center}
    \caption{Section Expression and Type Signature}
    \vspace{0.1in}
    \leavevmode
    \small
    \begin{tabular}[c]{|l|ll|} 
      \hline
      \textbf{global var}& \textbf{array section} & \textbf{type} \\ 
      \hline
      2-dimension & \texttt{c} & \texttt{float [[,]]} \\ 
      & \texttt{c[[:,:]]} & \texttt{float [[,]]} \\ 
      \hline
      1-dimension & \texttt{c[[i,:]]} & \texttt{float [[]]} \\ 
      & \texttt{c[[i,1::2]]} & \texttt{float [[]]} \\ 
      \hline
      scalar(0-dim) & \texttt{c[[i,j]]} & \texttt{float \#} \\ 
      \hline
    \end{tabular}
    \normalsize
    \label{tab:section}
  \end{center}
\end{table}

The size of an array in Java can be had from its
\texttt{length} field.  In HPJava, information like the
distributed group and distributed dimensions can be accessed from the
following inquiries, available on all global array types:
\small
\begin{verbatim}
  Group grp()       // distribution group

  Range rng(int d)  // d'th range
\end{verbatim}
\normalsize
Further inquiry functions on {\tt Range} yield values such as
extents and distribution formats.

\subsection{Program Execution Control}
\label{control}

HPJava has all the conventional Java statements for execution control
within a single process.  It introduces three new control
constructs: {\em on}, {\em at}, and {\em overall} for execution
control across processes.
A new concept, the active process group, is introduced. It is the
set of processes sharing the current thread of control.

In a traditional SPMD program, switching the active
process group is effectively implemented by {\em if} statements
such as:
\small
\begin{verbatim}
  if(myid >= 0 && myid < 4) {
    ...
  }
\end{verbatim}
\normalsize
Inside the braces, only processes numbered 0 to 3
share the control thread.
In HPJava, this effect is expressed using a \texttt{Group}.
When a HPJava program starts, the active process group has a
system-defined value.  During the execution, the active process group
can be changed explicitly through an {\em on} construct in the
program.

In a shared memory program, accessing the value of a variable is
straightforward. In a message passing system, only the process which
holds data can read and write the data.  We sometimes call this 
\inxx{SPMD constraint} SPMD constraint.  
A traditional SPMD program respects this constraint by using an idiom
like
\small
\begin{verbatim}
  if(myid == 1) 
    my_data = 3 ;
\end{verbatim}
\normalsize
The {\em if} statement makes sure that only \texttt{my\_data}
on process 1 is assigned as 3.

In the language we present here, similar constraints must be respected.
Besides \emph{on} construct introduced earlier, there is a
convenient way to change the active process group to access a required
array element, namely, the {\em at} construct.
Suppose array \texttt{a} is defined as in the previous section, then:
\small
\begin{verbatim}
  on(q) {
    a [1] = 3 ;    // error

    at(j = x [1])
      a [j] = 3 ;  // correct
  }
\end{verbatim}
\normalsize
The assignment statement guarded by an {\em at} construct is
correct; the one without is likely to imply access to an element
not held locally.  Formally it is illegal because, in a simple
subscripting operation, an integer expression cannot
be used to subscript a distributed dimension.  The
{\em at} construct introduces a new variable \texttt{j}, a {\em named
location}, with scope only inside the block controlled by the
{\em at}.  Named locations are the only legal element subscripts 
in distributed dimensions.

A more powerful construct called {\em overall}
combines restriction of the active process group with a loop:
\small
\begin{verbatim}
  on(q)
    overall(i = x | 0 : 3)
      a [i] = 3 ;
\end{verbatim}
\normalsize
is essentially equivalent to\footnote{A compiler can implement
overall construct in a more efficient way, using linearized
address calculation.  For detailed translation schemes for the
overall construct, please refer to \cite{Car98}.}
\small
\begin{verbatim}
  on(q) 
    for(int n = 0 ; n < 4 ; n++)
      at(i = x [n])
        a [i] = 3;
\end{verbatim}
\normalsize
In each iteration, the active process group is changed to \texttt{q /
i}.  In Section \ref{sec:examples}, we will illustrate with further
programs how {\em at} and {\em overall} constructs conveniently
allow one to keep the active process group equal to the data owner
group for the assigned data.

\subsection{Communication Library Functions}
\label{communication}

When accessing data on another process, HPJava needs explicit
communication, as in a normal SPMD program. 
Communication libraries are provided as packages in HPJava. Detailed
function specifications are given elsewhere.  The next
section will introduce a small number of top level collective
communication functions.


\section{Java Packages for HPspmd Programming}
\label{sec:package}

The implementation of the HPJava compiler is based on a run-time system.
It is actually a source-to-source translator converting an HPJava
program to a Java node program, with function calls to the run-time
library, called \emph{adJava}.

The run-time interface consists of several Java packages. The most
important one is the HPspmd runtime proper.  It includes the classes
needed to translate language constructs.  Other packages provide
communication and some simple I/O functions.  Important classes in the
first package include distributed array ``container classes'' and
related classes describing process groups and index ranges.  These
classes correspond directly to HPJava built-in classes.

The first hierarchy is based on \texttt{Group}.  A group, or
process group, defines some subset of the processes executing the SPMD
program.  They can be used to describe how program variables,
such as arrays, are distributed or replicated across the process pool,
or to specify which subset of processes executes a particular code
fragment.  Important members of adJava {\tt Group} class include the
pair {\tt on()}, {\tt no()} used to translate the {\em on} construct.

The most common way to create a group object is through the
constructor for one of the subclasses representing a process
grid.  The subclass \texttt{Procs} represents a grid of processes
and carries information on process dimensions: in particular an
inquiry function {\tt dim(r)} returns a range object describing the
$r$-th process dimension.  {\tt Procs} is further subclassed by
\texttt{Procs0}, \texttt{Procs1}, \texttt{Procs2}, \ldots which
provide simpler constructors for fixed dimensionality process grids.

The second hierarchy in the package is based on \texttt{Range}.
A range is a map from the integer interval $0,\ldots,n-1$ into
some process dimension (i.e., some dimension of a process grid).
Ranges are used to parametrize distributed arrays and the
{\em overall} distributed loop.

The most common way to create a range object is to use the constructor
for one of the subclasses representing ranges with specific
distribution formats.  
Simple block distribution format is implemented by
{\tt BlockRange}, while \texttt{CyclicRange} and \texttt{BlockCyclicRange}
represent other standard distribution formats of HPF.  The subclass
\texttt{CollapsedRange} represents a sequential (undistributed range).
Finally, a \texttt{DimRange} is associated with each process dimension
and represents the range of coordinates for the process dimension
itself---just one element is mapped to each process.

The related adJava class \texttt{Location} represents an individual
location in a particular distributed range.  Important members of the
adJava {\tt Range} class include the function {\tt location(i)}, which
returns the $i$th location in a range, and its inverse, {\tt idx(l)},
which returns the global subscript associated with a given location.
Important members of the {\tt Location} class include {\tt at()} and
{\tt ta()}, used in the implementation of the HPJava that {\em at}
construct.

Finally, we have the rather complex hierarchy of classes representing
distributed arrays.  HPJava global arrays declared using
\texttt{[[]]} are represented by Java objects belonging to classes
such as:
\begin{small}
\begin{verbatim}
   Array1dI, Array1cI,
   Array2ddI, Array2dcI, Array2cdI, Array2ccI,
   ...
   Array1dF, Array1cF,
   Array2ddF, Array2dcF, Array2cdF, Array2ccF,
   ...
\end{verbatim}
\end{small}
Generally speaking, the class
``\texttt{Array}\textit{n}\texttt{c|d}\textit{{\ldots}T}''
represents $n$-dimensional distributed arrays with elements of type
{\em T}---currently one of {\tt I}, {\tt F}, \ldots, meaning {\tt int},
{\tt float}, \ldots. \footnote{In the inital implementation, the element
  type is restricted to the Java primitive types.}  The penultimate
part of the class name is a string of $n$ ``c''s and ``d''s specifying
whether each dimension is collapsed or normally distributed.  These
correlate with presence or absence of an asterisk in slots of the
HPJava type signature.  The concrete {\tt Array\ldots} classes
implement a series of abstract interfaces.  These follow a similar
naming convention, but the root of their names is {\tt Section} rather
than {\tt Array} (so {\tt Array2dcI}, for example, implements {\tt
Section2dcI}).  The hierarchy of {\tt Section} interfaces is
illustrated in Figure \ref{fig:arrays}.
The need to introduce the {\tt Section} interfaces should be evident
from the hierarchy diagram.  The type hierarchy of HPJava involves a
kind of multiple inheritance.  The array type {\tt int [[*, *]]}, for
example, is a specialization of both the types {\tt int [[*, ]]}
and {\tt int [[, *]]}.  Java allows ``multiple inheritance'' only from
interfaces, not classes.

\begin{figure}[tbp]
  \begin{center}
    \leavevmode
    \epsfig{figure=./zgs/arrays.eps,height=2.4in} 
    \caption{The adJava {\tt Section} hierarchy}
    \label{fig:arrays}
  \end{center}
\end{figure}

Important members of the {\tt Section} interfaces include
inquiry functions {\tt dat()}, which returns an ordinary one
dimensional Java array used to store the locally held elements of the
distributed array, and the member {\tt pos(i, ...)}, which takes $n$
subscript arguments and returns the local offset of the element implied by
those subscripts.  Each argument of {\tt pos} is a location or an
integer (only allowed if the corresponding dimension is collapsed).
These functions are used to implement elemental subscripting.
The inquiry {\tt grp()} returns the group over which
elements of the array are distributed.  The inquiry {\tt rng(d)}
returns the $d$th range of the array.

Another package in adJava is the communication library.
The adJava communication package includes classes corresponding to the
various collective communication schedules provided in the NPAC PCRC
kernel.  Most of them provide a constructor to establish a schedule,
and an \texttt{execute} method, which carries out the data movement
specified by the schedule.  Different
communication models may eventually be added through further packages.

The collective communication schedules can be used directly by
the programmer or invoked through certain wrapper functions.
A class named \texttt{Adlib} is defined
with static members that create and execute
communication schedules and perform simple I/O functions. 
This class includes, for example, the following
methods, each implemented by
constructing the appropriate schedule and then executing it:
\begin{small}
\begin{verbatim}
  static void remap(Section dst, Section src)
  static void shift(Section dst, Section src, int shift, int dim, int mode)
  static void copy(Section dst, Section src)
  static void writeHalo(Section src, int [] wlo, int [] whi, int [] mode)
\end{verbatim}
\end{small}
\texttt{Adlib.remap} will copy the
corresponding elements from one array to another, regardless of their
respective distribution format.  \texttt{Adlib.shift} will shift data
by a certain amount in a specific dimension of the array, in either
cyclic or edge-off mode.  \texttt{Adlib.writeHalo} is used to update
ghost regions.

Given the classes described above, one can program in the HPspmd style
in pure Java program.  Then the idea of the HPJava compiler is just to
translate the HPJava program onto this interface, in the meanwhile, to use
optimized address calculation instead of function calls to access
elements of arrays. (For a detailed translation scheme, please refer to
\cite{Zha98}.)


\section{Programming Examples}
\label{sec:examples}

\begin{figure}[tbp]
  \begin{center}
\begin{small}
\begin{verbatim}
  Procs1 p = new Procs1(4);
  on(p) {
    Range x = new CyclicRange(n, p.dim(0));
    float a[[*,]] = new float [[n, x]];
    ... some code to initialise `a' ...
    float b[[*]] = new float [[n]]; // buffer

    for(int k = 0 ; k < N - 1 ; k++) {
      at(l = x[k]) {
        float d =  Math.sqrt(a[k,l]) ;
        a[k,l] = d ;
        for(int s = k + 1 ; s < N ; s++)
          a[s,l] /= d ;
      }
      Adlib.remap(b[[k+1:]], a[[k+1:,  k]]);

      overall(m = x | k + 1 : )
        for(int i = x.idx(m) ; i < N ; i++)
          a[i,m] -= b[i] * b[x.idx(m)] ;
    }
    at(l = x [N - 1])
      a[N - 1,l] = Math.sqrt(a[N-1,l]) ;
  }
\end{verbatim}
\end{small}
    \caption{Choleski decomposition.}
    \label{fig:choleski}
  \end{center}
\end{figure}

In this section we give two example programs to show the new
language features.  The first example is the Choleski decomposition, seen in
Figure \ref{fig:choleski}.
Here, \texttt{remap} is used to broadcast one updated column to each
process. The function \texttt{idx} gets the global index of location
\texttt{m} relative to the parent range \texttt{x}.
The second example is the Jacobi iteration, seen in Figure \ref{fig:Jacobi}.
In the displayed code there is only one iteration, but it demonstrates
how to define range references with ghost areas, how to use
the \texttt{writeHalo} function, and how to use shifted locations
as subscripts. 


\begin{figure}[tbp]
  \begin{center}
\begin{small}
\begin{verbatim}
  Procs2 p = new Procs2(2, 4);
  Range x = new BlockRange(100, p.dim(0), 1),
        y = new BlockRange(200, p.dim(1), 1); 
  on(p) {
    float [[,]] a = new float [[x,y]], b = new float [[x,y]];
    ... some code to initialize `a'

    Adlib.writeHalo(a);

    overall(i = x | : )
      overall(j = y | : )
        b[i,j] = 0.25 * (a[i-1,j] + a[i+1,j] + a[i,j-1] + a[i,j+1]);
    overall(i = x | : )
      overall(j = y | : )
        a[i,j] = b[i,j];
  }
\end{verbatim}
\end{small}
    \caption{Jacobi relaxation.}
    \label{fig:Jacobi}
  \end{center}
\end{figure}


\section{Issues in the Language Design}
\label{sec:design}

With some of the implementation mechanisms exposed, we can
better discuss the language design itself.

\subsection{Extending the Java Language}

The first question to answer is why use Java as a base language?
Actually, the programming model embodied in HPJava is largely language-independent.  
It can be bound to other languages like C, C++, and Fortran.
But Java is a convenient base language, especially for initial
experiments, because it provides full object-orientation --- convenient
for describing complex distributed data --- implemented in a relatively
simple setting, conducive to development of source-to-source
translators.  It has been noted elsewhere that Java has various
features suggesting it could be an attractive language for science
and engineering \cite{Java98}.

With Java as a base language, an obvious question is whether we can
extend the language by simply adding packages, instead of changing the
syntax.  There are two problems with doing this for data-parallel
programming.

Our baseline is HPF, and any package supporting parallel arrays as
general as HPF is likely to be cumbersome to code with.  Our run-time
system needs an (in principle) infinite series of class names
\begin{small}
\begin{verbatim}
   Array1dI, Array1cI, Array2ddI, Array2dcI, ...
\end{verbatim}
\end{small}
to express the HPJava types
\begin{small}
\begin{verbatim}
  int [[]], int [[*]], int [[,]], int [[,*]] ... 
\end{verbatim}
\end{small}
as well as the corresponding series for \texttt{char}, \texttt{float}, and so on.
To access an element of a distributed array in HPJava, one writes
\begin{small}
\begin{verbatim}
  a[i] = 3 ;
\end{verbatim}
\end{small}
In the adJava interface, it must be written as
\begin{small}
\begin{verbatim}
  a.dat()[a.pos(i)] = 3 ;
\end{verbatim}
\end{small}
This is only for simple subscripting. Constructing array
sections will be even more complex using the raw class library
interface.

The second problem is that a Java program using a package like adJava
in a direct, naive way will have very poor performance, because all
the local address of the global array are expressed by functions such
as \texttt{pos}.  An optimization pass is needed to transform offset
computation to a more intelligent style.  So if a preprocessor must do
these optimizations anyway, it makes sense to design a syntax to
express the concepts of the programming model more naturally.

\subsection{Why not HPF?}

The design of the HPJava language is strongly influenced by HPF.
The language emerged partly out of practices adopted in our efforts to
implement an HPF compilation system \cite{Zha97}.  For example:
\begin{small}
\begin{verbatim}
  !HPF$ POCESSOR    P(4)
  !HPF$ TEMPLET     T(100)
  !HPF$ DISTRIBUTE  T(BLOCK) ONTO P
        REAL        A(100,100), B(100)
  !HPF$ ALIGN       A(:,*) WITH T(:)
  !HPF$ ALIGN       B WITH T
\end{verbatim}
\end{small}
have their conterparts in HPJava:
\begin{small}
\begin{verbatim}
  Procs1 p = new Procs1(4);
  Range x = new BlockRange (100, p.dim(0));
  float [[,*]] a = new float [[x,100]] on p;
  float [[ ]] b = new float [[x]] on p;
\end{verbatim}
\end{small}
Both languages provide a globally addressed name space for data parallel
applications.  Both of them can specify how data are mapped on to a
processor grid.  The difference between the two lies in their
communication aspects.  In HPF, a simple assignment statement may cause data
movement.  For example, given the above distribution, the assignment
\begin{small}
\begin{verbatim}
  A(10,10) = B(30)
\end{verbatim}
\end{small}
will cause communication between processor 1 and 2.  In HPJava, similar
communication must be done through explicit function calls: \footnote{By default
Fortran array subscripts starts from 1, while HPJava global subscripts always
start from 0.}
\begin{small}
\begin{verbatim}
  Adlib.remap(a[[9,9]], b[[29]]);
\end{verbatim}
\end{small}
Experience from compiling the HPF language suggests that, while there
are various kinds of algorithms to detect communication automatically,
it is often difficult to give the generated node program acceptable
performance.  In HPF, the need to decide on which processor the
computation should be executed further complicates the situation.  One
may apply ``owner computes'' or ``majority computes'' rules to
partition computation, but these heuristics are difficult to apply in
many situations.

In HPJava, the SPMD programming model is emphasized.  The distributed
arrays just help the programmer organize data, and simplify
global-to-local address translation.  The tasks of computation
partition and communication are still under control of the programmer.
This is certainly an extra onus, and the language may be more difficult to
program than HPF; \footnote{The program must meet SPMD constraints, e.g.,
only the owner of an element can access that data.  Run-time checking
can be added automatically to ensure such conditions are met.} but
it helps programmers to understand the performance of the program much
better than in HPF, so algorithms exploiting locality and parallelism
are encouraged.  It also dramatically simplifies the work of the
compiler.

Because the communication sector is considered an ``add-on'' to the
basic language, HPJava should interoperate more smoothly than HPF with
other successful SPMD libraries, including MPI \cite{MPIStandard},
Global Arrays \cite{Nie96}, CHAOS \cite{Das94},
%DAGH \cite{HDDA_DAGH}, 
and so on.

\subsection{Datatypes in HPJava}

In a parallel language, it is desirable to have both local variables
(like the ones in MPI programming) and global variables (like
the ones in HPF programming).  The former provide flexibility and are
ideal for task parallel programming; the latter are convenient
especially for data parallel programming.

In HPJava, variable names are divided into two sets.  In general those
declared using ordinary Java syntax represent local variables and those
declared with \texttt{\#} or \texttt{[[]]} represent global variables.
The two sectors are independent.  In the implementation of HPJava the
global variables have special data descriptors associated with them,
defining how their components are divided or replicated across
processes.  The significance of the data descriptor is most obvious
when dealing with procedure calls.
Passing array sections to procedure calls is an important component in
the array processing facilities of Fortran90 \cite{Fortran90}.  The data
descriptor of Fortran90 will include stride information for each array
dimension.  One can assume that HPF needs a much more complex kind of
data descriptor to allow passing distributed arrays across procedure
boundaries.  In either case the descriptor is not visible to the
programmer.  Java has a more explicit data descriptor concept; its
arrays are considered as objects, with, for example, a publicly
accessible \texttt{length} field.  In HPJava, the data descriptors for
global data are similar to those used in HPF, but more explicitly
exposed to programmers.  Inquiry functions such as \texttt{grp} and
\texttt{rng} have a similar role in global data as the field
\texttt{length} in an ordinary Java array.

Keeping two data sectors seems to complicate the language and its
syntax.  But it provides convenience for both task and data parallel
processing.  There is no need for things like the \texttt{LOCAL}
mechanism in HPF to call a local procedure on the node processor.  The
descriptors for ordinary Java variables are unchanged in HPJava.  On
each node processor ordinary Java data will be used as local varables,
as in an MPI program.


\section{Projects in Progress\label{sec:progress}}

Projects related to this work include development of MPI, HPF,
and other parallel languages such as ZPL\cite{ZPL} and Spar\cite{Spar}.
Here, we explain the background and future developments of our own
project.

The work originated in our compilation practices for HPF. As described
in \cite{Zha97}, our compiler emphasizes run-time support.
Adlib\cite{Car98}, a PCRC run-time library, provides a
rich set of collective communication functions.  It was realized that
by raising the run-time interface to the user level, a rather
straightforward (compared to HPF) compiler could be developed to
translate the high level language code to a node program calling the
run-time functions.

Currently, a Java interface has been implemented on top of the Adlib
library.  With classes such as \texttt{Group}, \texttt{Range}, and
\texttt{Location} in the Java interface, one can write Java programs
quite similar to HPJava proposed here. But a program executed in
this way will have large overhead due to function calls (such as
address translation) when accessing data inside loop constructs.
Given the knowledge of data distribution plus inquiry functions inside
the run-time library, one can substitute address translation calls with
linear operations on the loop variable, and keep inquiry
function calls ``outside the loop.'' 

At the present time, we are implementing the translator.  Further
research work will include optimization and safety-checking techniques
in the compiler for HPspmd programming.

Figure \ref{fig:performance} shows a preliminary benchmark for a hand-translated 
versions of our examples. The parallel programs are executed
on 4 sparc-sun-solaris2.5.1 with MPICH and the Java JIT compiler in
JDK 1.2Beta2. For Jacobi iteration, the timing is for about 90
iterations; the array size is 1024X1024.

\begin{figure}[tbp]
  \begin{center}
    \leavevmode
    \epsfig{figure=./zgs/Jacobi4.eps, height=1.8in}
    \caption{Preliminary performance.}
    \label{fig:performance}
  \end{center}
\end{figure}

We also compared the sequential Java, C++, and Fortran version of the
code, all with \texttt{-O} flag when compiling. The dotted lines shown
in the figure only represent times for the one processor case. We can
see that on a single processor, Java program use language supported
mechanism for calculating array element address; hence, it is slower than
HPJava (as HPJava uses an optimized address calculation scheme). We emphasize again that in the
picture we are comparing {\em sequential} Fortran, etc., with {\em
parallel} HPJava.  This is not supposed to be an comparative evaluation
of the various languages.  It is just supposed to give an impression of
the performance ballpark Java is currently operating in.

\section{Summary\label{summary.tex}}

Through the simple examples in this chapter we have tried to
illustrate that the programming language presented here provides the
flexibility of SPMD programming, and much of the convenience of HPF.
The language helps programmers to express parallel algorithms in a
more explicit way.  We suggest it will help programmers to solve real
application problems more easily, compared with using communication
packages such as MPI directly, and allow the compiler writer to
implement the language compiler without the difficulties met in the HPF
compilation.

The overall structure of the system is shown in Figure
\ref{fig:hpspmd}.  The central DAD stands for the Distributed Array
Descriptor.  Around it, we have different run-time libraries. The Java
interface is most relevant here.  But the Java binding is only an
introduction of the programming style.  (A Fortran binding is being
developed.)  Initially, the Java version can be used as a software tool
for teaching parallel programming.  As Java for scientific computation
becomes more mature, it will be a practical programming language to
solve real application problems in parallel and distributed
environments.

\begin{figure}[tbp]
  \begin{center}
    \leavevmode
    \epsfig{figure=./zgs/HPspmdLayers.eps} 
    \caption{Layers of the HPspmd model.}
    \label{fig:hpspmd}
  \end{center}
\end{figure}


%--------------------

%\bibliographystyle{unsrt}
%\bibliographystyle{plain}
%\bibliography{zgs}

%%Postscript encapsulation macro(s)
\input epsf
%\putfig{filespec.eps}{scale}
\def\putfig#1#2{\mbox{ \def\epsfsize##1##2{#2##1} \epsfbox{#1}}}

\chapter[The HPspmd Model and its Java Binding]{The {\em HPspmd} Model and its Java Binding}
\label{zgs:chap}
\author[Guansong Zhang, Bryan Carpenter, Geoffrey Fox,\\
  Xinying Li, and Yuhong Wen]{Guansong Zhang, Bryan Carpenter, Geoffrey Fox,\\
  Xinying Li, and Yuhong Wen}
\affil{NPAC at Syracuse University\\
        Syracuse, NY, 13 244 \\
	\vspace{0.1in}
        Email: {\em \{zgs,dbc,gcf,xli,wen\}@npac.syr.edu}}

%\noindent
%{\bf Abstract
%{\small\em
%This chapter introduces a new language, \emph{HPJava}, for parallel
%programming on message passing systems, including workstation
%clusters.  The language provides a high level SPMD programming model,
%called \emph{HPspmd}.  Through examples and performance results, the
%features of the new programming style, and its implementation, are
%illustrated.  
%} 
%}

%\vspace{0.5cm}

%\noindent
%{\it Keywords:}
%{\small Java, data parallel programming, SPMD}


%--------------------


\section{Introduction}

In this chapter we introduce the \inxx{HPJava} \emph{HPJava}
language, a programming language that extends Java for parallel
programming on message passing systems --- from multiprocessor systems to
workstation clusters.

HPJava owes much to High Performanace Fortran (HPF)
\cite{HPFStandard}.  Its model of data distribution is adapted directly
from the HPF model.  The heritage of HPF can be traced back to Fortran
dialects that were implemented most successfully on SIMD and other
tightly-coupled MPP architectures.  While it was always a goal of the
HPF designers that the language should be efficiently implementable on
the more loosely coupled MIMD clusters that dominate today, the
complexity of the language --- and notably the design goal of emulating
exactly the semantics of a sequential Fortran program --- have made
efficient implementation on today's architectures quite hard.

HPJava, in contrast, starts from the assumption that the target
hardware is a set of interacting MIMD processors, and exposes that
assumption explictly in its programming model.  This greatly
simplifies the task of the compiler, and increases the chance of
obtaining efficient implementations on architectures including PC and
workstation clusters.  Instead of the HPF programming model, the
language introduces a high-level structured SPMD programming
style --- the \inxx {HPspmd} HPspmd model.  A program written in
this class of language explicitly coordinates well-defined process
groups.  These cooperate in a loosely synchronous manner, sharing
logical threads of control.  As in a conventional distributed-memory
SPMD program, only a process owning a data item such as an array
element is allowed to access the item directly.  The language provides
special constructs that allow programmers to meet this constraint
conveniently.

Besides the normal variables of the sequential base language, the
language model introduces classes of global variables that are stored
collectively across process groups.  Primarily, these are
distributed arrays.  They provide a global name space in the form of
globally subscripted arrays, with assorted distribution patterns.  This
helps to relieve programmers of error-prone activities such as the
local-to-global, global-to-local subscript translations which occur in
data parallel applications.

In addition to special data types, the language provides special
constructs to facilitate both data parallel and task parallel
programming.  Through these constructs different processors can work either
simultaneously on globally addressed data, or independently to execute
complex procedures on locally held data.  The conversion between these
phases is seamless.

In the traditional SPMD mold, the language itself does not provide
implicit data movement semantics.  This greatly simplifies the task of
the compiler, and should encourage programmers to use algorithms that
exploit locality.  Data on remote processors is accessed exclusively
through explicit library calls.  In particular, the initial HPJava
implementation relies on a library of collective communication
routines originally developed as part of an HPF run-time library.
Other distributed-array-oriented communication libraries may be
bound to the language later.  Due to the explicit SPMD programming
model, low level MPI communication is always available as a fallback.
The language itself only provides basic concepts to organize data
arrays and process groups.  Different communication patterns are
implemented as library functions.  This allows the possibility that if
a new communication pattern is needed, it is relatively easily
integrated through new libraries.

In our earlier work on HPF compilation \cite{Zha97} the role of run-time
support was emphasized.  Difficulties in compiling HPF efficiently
suggested to make the run-time communication library directly visible in
the programming model.  Since Java is a simple, elegant language,
we are implementing our prototype based upon this language.

Section \ref{sec:javaBinding} reviews the HPspmd model in the context
of the HPJava language.  Section \ref{sec:package} describes the class
library packages used in code generated by the HPJava translator, and
thus exposes many of the implementation issues.  Some examples of simple
algorithms expressed in HPJava are given in Section
\ref{sec:examples}.  Then Section \ref{sec:design} discusses the
rationale of various design decisions in the language.  The status of
the project and future goals are summarized in Sections
\ref{sec:progress} and \ref{summary.tex}.



\section{Java Language Binding\label{sec:javaBinding}} 

This section introduces the HPJava language.
HPJava contains the whole of standard Java as a subset.  It adds
various built-in classes for describing process groups and index ranges,
new global data types, and some syntax for accessing distributed data
and specifying which processes execute particular statements.

\subsection{Basic Concepts}

Key concepts in the programming model are built around the 
\inxx{group} 
\inxx{group, process groups}
process groups used to describe program execution control in a parallel
program.
\texttt{Group} is a class representing a process group, typically with
a grid structure and an associated set of process dimensions.
It has its subclasses that represent different grid dimensionalities,
such as \texttt{Procs1}, \texttt{Procs2}, etc. For example,
\small
\begin{verbatim}
  Procs2 p = new Procs2(2,4);
\end{verbatim}
\normalsize
{\tt p} is a 2-dimensional, 2 x 4 grid of processes.

The second category of concepts is associated with 
\inxx{range}
\inxx{range, distributed ranges}
distributed ranges.
The elements of an ordinary array can be represented by an array name and
an integer sequence.  There are two parameters associated with this
sequence: an index to access each array element, and the extent of the
range this index can be chosen from.  In describing a distributed
array, HPJava introduces two new kinds of entity to represent the
analogous concepts.  A {\em range} maps an integer interval into a
process dimension according to certain distribution format. Ranges
describe the extent and the mapping of array dimensions.  A {\em
location}, or slot, is an abstract element of a range.
For example,
\small
\begin{verbatim}
  Range x = new BlockRange(100, p.dim(0)) ;
  Range y = new CyclicRange(200, p.dim(1)) ;
\end{verbatim}
\normalsize
creates two ranges distributed over the two process dimensions of the
group \texttt{p}.  One is block distributed, the other is cyclic
distributed.  There are 100 different locations in the range \texttt{x}.
The first one, for example, is
\small
\begin{verbatim}
  x [0]
\end{verbatim}
\normalsize

Additional related concepts are 
\inxx{group, subgroups}
subgroups
and 
\inxx{range, subranges}
subranges.  A subgroup is some slice of a
process array, formed by restricting the process coordinates in one or
more dimensions to single values.  Suppose {\tt i} is a location in a
range distributed over a dimension of group {\tt p}. The expression
\small
\begin{verbatim}
  p / i
\end{verbatim}
\normalsize
represents a smaller group---the slice of {\tt p} to which location
{\tt i} is mapped.  
Similarly, a subrange is a section of a range, parameterized by
a global index triplet. Logically, it represents a subset of the
locations of the original range.
The syntax for a subrange expression is
\small
\begin{verbatim}
  x [1 : 49]
\end{verbatim}
\normalsize
The symbol ``\texttt{:}'' is a special separator.  It is used to
construct a Fortran-90 style triplet, with lower- and upper-bound
expressions defining an integer subset.  The optional third member of a
triplet is a stride.

When a process grid is defined, certain ranges and locations are
also implicitly defined.  As shown in Figure \ref{fig:group},
two primitive ranges are associated with dimensions of the group
\texttt{p}:
\small
\begin{verbatim}
  Range u = p.dim(0);  
  Range v = p.dim(1);
\end{verbatim}
\normalsize
\texttt{dim()} is a member function that returns a range reference,
directly representing a processor dimension.
We can obtain a location in range \texttt{v}, and use it
to create a new group,
\small
\begin{verbatim}
  Group q = p / v [1] ;
\end{verbatim}
\normalsize
\begin{figure}[tbp]
  \begin{center}
    \leavevmode
    \epsfig{figure=./zgs/group.eps,height=1.5in}
    \caption{Structured process group.}
    \label{fig:group}
  \end{center}
\end{figure}

In a traditional SPMD program, execution control is based on
{\em if} statements and process ID or rank numbers. In the new
programming language, switching execution control is based on the
structured process group.  For example, it is not difficult to guess
that the following code:
\small
\begin{verbatim}
  on(p) {
    ...
  }
\end{verbatim}
\normalsize
will restrict the execution control inside the bracket to processes in
group \texttt{p}.
The language also provided well-defined constructs to split
execution control across processes according to data items we want to
access.  This will be discussed later.

\subsection{Global Variables}

When an SPMD program starts on a group of $n$ processes, there will be
$n$ control threads mapped to $n$ physical processors.
In each control thread, the program can define variables in the same
way as in a sequential program.  The variables created in this way are
local variables.  Their names may be common to all
processes, but they will be accessed individually (their scope is
local to a process).

Besides local variables, HPJava allows a program to define
global variables, explicitly mapped to a process group.  A
global variable will be treated by the process group that created it
as a single entity. The language has special syntax for the definition
of global data.  Global variables are all defined by using the
\texttt{new} operator from free storage.  When a global variable is
created, a
\inxx{data descriptor} data descriptor 
is also allocated to describe where the data are
held.  On a single processor, an array variable might be parametrized
by a simple record containing a memory address and an \texttt{int}
value for its length.  On a multiprocessor, a more complicated
structure is needed to describe a distributed array.  The data
descriptor specifies where the data is created, and how they are
distributed.  The logical structure of a descriptor is shown in Figure
\ref{fig:descriptor}.

\begin{figure}[tbp]
  \begin{center}
    \leavevmode
    \epsfig{figure=./zgs/descriptor.eps}
    \caption{Descriptor.}
    \label{fig:descriptor}
  \end{center}
\end{figure}

\begin{figure}[bp]
  \begin{center}
    \leavevmode
    \epsfig{figure=./zgs/map.eps}
    \caption{Memory mapping.}
    \label{fig:mapping}
  \end{center}
\end{figure}

HPJava has special syntax to define global data.  The statement
\small
\begin{verbatim}
  int # s = new int # on p ;
\end{verbatim}
\normalsize
creates a global scalar replicated over process group {\tt p}. In the
statement, \texttt{s} is a data descriptor handle --- a
\inxx{global scalar reference} global scalar reference.  
The scalar contains an integer value.
Global scalar references can be defined for any primitive type (or,
in principle, class type) of Java.
The symbol \texttt{\#} in the type signature
distinguishes a global scalar from a primitive integer.
For a global scalar, a field \texttt{value} is used to access the
value:
\small
\begin{verbatim}
  on(p) {
    int # s = new int # ;
    s.value = 100 ;
  }
\end{verbatim}
\normalsize
Note how the \texttt{on} clause can be omitted from the constructor:
the whole of the active process group is the default distribution group.
Figure \ref{fig:mapping} shows a possible memory mapping for this
scalar on different processes. 
Note, the value field of \texttt{s} is identical in each process in the
distribution group.   Replicated value variables are
different from local variables with identical names.  The
associated descriptors can be used to ensure the value is maintained
identically in each process, throughout program execution.

When defining a global array, it is not necessary to allocate a data
descriptor for each array element---one descriptor suffices for the whole
array.
An array can defined with various kinds of range, introduced earlier.
Suppose we have, as before,
\small
\begin{verbatim}
  Range x = new BlockRange(100, p.dim(0)) ;
\end{verbatim}
\normalsize
and the process group defined in Figure \ref{fig:group}, then
\small
\begin{verbatim}
  float [[]] a = new float [[x]] on q ;
\end{verbatim}
\normalsize
will create a global array with range \texttt{x} on group \texttt{q}.
Here \texttt{a} is a descriptor handle describing a one-dimensional
array of \texttt{float}. It is block distributed on group
\texttt{q}.\footnote{The \texttt{on} clause restricts the data owner
group of the array to \texttt{q}. If group \texttt{p} is used instead,
the one-dimenstional array will be replicated in the first dimenstion of
the group, and block distributed over the second dimension.} In
HPJava \texttt{a} is called a 
\inxx{global array reference} global or distributed
array reference.

A distributed array range can also be collapsed (or
sequential).  An integer range is specified, eg
\small
\begin{verbatim}
  float [[*]] b = new float [[100]] ;
\end{verbatim}
\normalsize
When defining an array with collapsed dimensions an asterisk is
normally added in the type signatures to mark the collapsed
dimensions.

The typical method of accessing global array elements is not exactly
the same as for local array elements, or for global scalar references.
In distributed dimensions of arrays
we must use named locations as subscripts, for example
\small
\begin{verbatim}
  at(i = x [3])
    a [i] = 3 ;
\end{verbatim}
\normalsize
We will leave discussion of the {\em at} construct to Section
\ref{control}, and give a simpler example here:
if a global array is defined with a collapsed dimension, accessing
its elements is modelled on local arrays.  For example:
\small
\begin{verbatim}
  for(int i = 0 ; i < 100 ; i++)
    b [i] = i ;
\end{verbatim}
\normalsize
assigns the loop index to each corresponding element in the array.

When defining a multidimensional global array, a single descriptor
parametrizes a rectangular array of any dimension:
\small
\begin{verbatim}
  Range x = new BlockRange(100, p.dim(0)) ;  
  Range y = new CyclicRange(100, p.dim(1)) ;
  float [[,]] c = new float [[x, y]];
\end{verbatim}
\normalsize
This creates a two-dimension global array with the first dimension
block distributed and the second cyclic distributed.  Now \texttt{c} is a
global array reference.  Its elements can be accessed using
single brackets with two suitable locations inside.

The global array introduced here is a true multidimensional
array, not a Java-like array-of-arrays.
Java-style arrays-of-arrays are still useful. For example, one can
define a local array of distributed arrays:
\small
\begin{verbatim}
  int[] size = {100, 200, 400};
  float [[,]] d[] = new float [size.length][[,]] ;
  Range x[], y[];
  for (int l = 0; l < size.length; l++) {
    const int n = size [l] ;
    x[l] = new BlockRange(n, p.dim(0)) ;  
    y[l] = new BlockRange(n, p.dim(1)) ;  
    d[l] = new float [[x[l], y[l]]];
  }
\end{verbatim}
\normalsize
This creates the stack of distributed arrays shown in Figure \ref{fig:layer}.

\begin{figure}[tbp]
  \begin{center}
    \leavevmode
    \epsfig{figure=./zgs/layer.eps}
    \caption{Array of distributed arrays.}
    \label{fig:layer}
  \end{center}
\end{figure}


Like Fortran 90, HPJava allows construction of sections of global
arrays.  The syntax of section subscripting uses double brackets.  The
subscripts can be scalar (integers or locations) or triplets.

Suppose we have array \texttt{a} and \texttt{c} defined as
above.  Then \texttt{a[[i]]}, \texttt{c}, \texttt{c[[i, 1::2]]},
and \texttt{c[[i, :]]} are all array sections.  Here, {\tt i} is an
integer in the appropriate interval (it could also be a
location in the first range of {\tt a} and {\tt c}).
Both the expressions \texttt{c[[i, 1::2]]} and \texttt{c[[i, :]]}
represent one-dimensional distributed arrays, providing aliases for
subsets of the elements in \texttt{c}.  The expression \texttt{a[[i]]}
contains a single element of \texttt{a}, but the result is a global
scalar reference (unlike the expression \texttt{a[i]} which is a simple
variable).

Array section expressions are often used as arguments in function
calls.\footnote{When used in method calls, the collapsed dimension
array is a \emph{subtype} of the ordinary one, i.e., an argument of
\texttt{float[[*,*]]}, \texttt{float[[*,]]} and \texttt{float[[,*]]}
type can all be passed to a dummy of type \texttt{float[[,]]}.
The converse is not true.}  Table \ref{tab:section} shows the type
signatures of global data with different dimensions.

\begin{table}[tbp]
  \begin{center}
    \caption{Section Expression and Type Signature}
    \vspace{0.1in}
    \leavevmode
    \small
    \begin{tabular}[c]{|l|ll|} 
      \hline
      \textbf{global var}& \textbf{array section} & \textbf{type} \\ 
      \hline
      2-dimension & \texttt{c} & \texttt{float [[,]]} \\ 
      & \texttt{c[[:,:]]} & \texttt{float [[,]]} \\ 
      \hline
      1-dimension & \texttt{c[[i,:]]} & \texttt{float [[]]} \\ 
      & \texttt{c[[i,1::2]]} & \texttt{float [[]]} \\ 
      \hline
      scalar(0-dim) & \texttt{c[[i,j]]} & \texttt{float \#} \\ 
      \hline
    \end{tabular}
    \normalsize
    \label{tab:section}
  \end{center}
\end{table}

The size of an array in Java can be had from its
\texttt{length} field.  In HPJava, information like the
distributed group and distributed dimensions can be accessed from the
following inquiries, available on all global array types:
\small
\begin{verbatim}
  Group grp()       // distribution group

  Range rng(int d)  // d'th range
\end{verbatim}
\normalsize
Further inquiry functions on {\tt Range} yield values such as
extents and distribution formats.

\subsection{Program Execution Control}
\label{control}

HPJava has all the conventional Java statements for execution control
within a single process.  It introduces three new control
constructs: {\em on}, {\em at}, and {\em overall} for execution
control across processes.
A new concept, the active process group, is introduced. It is the
set of processes sharing the current thread of control.

In a traditional SPMD program, switching the active
process group is effectively implemented by {\em if} statements
such as:
\small
\begin{verbatim}
  if(myid >= 0 && myid < 4) {
    ...
  }
\end{verbatim}
\normalsize
Inside the braces, only processes numbered 0 to 3
share the control thread.
In HPJava, this effect is expressed using a \texttt{Group}.
When a HPJava program starts, the active process group has a
system-defined value.  During the execution, the active process group
can be changed explicitly through an {\em on} construct in the
program.

In a shared memory program, accessing the value of a variable is
straightforward. In a message passing system, only the process which
holds data can read and write the data.  We sometimes call this 
\inxx{SPMD constraint} SPMD constraint.  
A traditional SPMD program respects this constraint by using an idiom
like
\small
\begin{verbatim}
  if(myid == 1) 
    my_data = 3 ;
\end{verbatim}
\normalsize
The {\em if} statement makes sure that only \texttt{my\_data}
on process 1 is assigned as 3.

In the language we present here, similar constraints must be respected.
Besides \emph{on} construct introduced earlier, there is a
convenient way to change the active process group to access a required
array element, namely, the {\em at} construct.
Suppose array \texttt{a} is defined as in the previous section, then:
\small
\begin{verbatim}
  on(q) {
    a [1] = 3 ;    // error

    at(j = x [1])
      a [j] = 3 ;  // correct
  }
\end{verbatim}
\normalsize
The assignment statement guarded by an {\em at} construct is
correct; the one without is likely to imply access to an element
not held locally.  Formally it is illegal because, in a simple
subscripting operation, an integer expression cannot
be used to subscript a distributed dimension.  The
{\em at} construct introduces a new variable \texttt{j}, a {\em named
location}, with scope only inside the block controlled by the
{\em at}.  Named locations are the only legal element subscripts 
in distributed dimensions.

A more powerful construct called {\em overall}
combines restriction of the active process group with a loop:
\small
\begin{verbatim}
  on(q)
    overall(i = x | 0 : 3)
      a [i] = 3 ;
\end{verbatim}
\normalsize
is essentially equivalent to\footnote{A compiler can implement
overall construct in a more efficient way, using linearized
address calculation.  For detailed translation schemes for the
overall construct, please refer to \cite{Car98}.}
\small
\begin{verbatim}
  on(q) 
    for(int n = 0 ; n < 4 ; n++)
      at(i = x [n])
        a [i] = 3;
\end{verbatim}
\normalsize
In each iteration, the active process group is changed to \texttt{q /
i}.  In Section \ref{sec:examples}, we will illustrate with further
programs how {\em at} and {\em overall} constructs conveniently
allow one to keep the active process group equal to the data owner
group for the assigned data.

\subsection{Communication Library Functions}
\label{communication}

When accessing data on another process, HPJava needs explicit
communication, as in a normal SPMD program. 
Communication libraries are provided as packages in HPJava. Detailed
function specifications are given elsewhere.  The next
section will introduce a small number of top level collective
communication functions.


\section{Java Packages for HPspmd Programming}
\label{sec:package}

The implementation of the HPJava compiler is based on a run-time system.
It is actually a source-to-source translator converting an HPJava
program to a Java node program, with function calls to the run-time
library, called \emph{adJava}.

The run-time interface consists of several Java packages. The most
important one is the HPspmd runtime proper.  It includes the classes
needed to translate language constructs.  Other packages provide
communication and some simple I/O functions.  Important classes in the
first package include distributed array ``container classes'' and
related classes describing process groups and index ranges.  These
classes correspond directly to HPJava built-in classes.

The first hierarchy is based on \texttt{Group}.  A group, or
process group, defines some subset of the processes executing the SPMD
program.  They can be used to describe how program variables,
such as arrays, are distributed or replicated across the process pool,
or to specify which subset of processes executes a particular code
fragment.  Important members of adJava {\tt Group} class include the
pair {\tt on()}, {\tt no()} used to translate the {\em on} construct.

The most common way to create a group object is through the
constructor for one of the subclasses representing a process
grid.  The subclass \texttt{Procs} represents a grid of processes
and carries information on process dimensions: in particular an
inquiry function {\tt dim(r)} returns a range object describing the
$r$-th process dimension.  {\tt Procs} is further subclassed by
\texttt{Procs0}, \texttt{Procs1}, \texttt{Procs2}, \ldots which
provide simpler constructors for fixed dimensionality process grids.

The second hierarchy in the package is based on \texttt{Range}.
A range is a map from the integer interval $0,\ldots,n-1$ into
some process dimension (i.e., some dimension of a process grid).
Ranges are used to parametrize distributed arrays and the
{\em overall} distributed loop.

The most common way to create a range object is to use the constructor
for one of the subclasses representing ranges with specific
distribution formats.  
Simple block distribution format is implemented by
{\tt BlockRange}, while \texttt{CyclicRange} and \texttt{BlockCyclicRange}
represent other standard distribution formats of HPF.  The subclass
\texttt{CollapsedRange} represents a sequential (undistributed range).
Finally, a \texttt{DimRange} is associated with each process dimension
and represents the range of coordinates for the process dimension
itself---just one element is mapped to each process.

The related adJava class \texttt{Location} represents an individual
location in a particular distributed range.  Important members of the
adJava {\tt Range} class include the function {\tt location(i)}, which
returns the $i$th location in a range, and its inverse, {\tt idx(l)},
which returns the global subscript associated with a given location.
Important members of the {\tt Location} class include {\tt at()} and
{\tt ta()}, used in the implementation of the HPJava that {\em at}
construct.

Finally, we have the rather complex hierarchy of classes representing
distributed arrays.  HPJava global arrays declared using
\texttt{[[]]} are represented by Java objects belonging to classes
such as:
\begin{small}
\begin{verbatim}
   Array1dI, Array1cI,
   Array2ddI, Array2dcI, Array2cdI, Array2ccI,
   ...
   Array1dF, Array1cF,
   Array2ddF, Array2dcF, Array2cdF, Array2ccF,
   ...
\end{verbatim}
\end{small}
Generally speaking, the class
``\texttt{Array}\textit{n}\texttt{c|d}\textit{{\ldots}T}''
represents $n$-dimensional distributed arrays with elements of type
{\em T}---currently one of {\tt I}, {\tt F}, \ldots, meaning {\tt int},
{\tt float}, \ldots. \footnote{In the inital implementation, the element
  type is restricted to the Java primitive types.}  The penultimate
part of the class name is a string of $n$ ``c''s and ``d''s specifying
whether each dimension is collapsed or normally distributed.  These
correlate with presence or absence of an asterisk in slots of the
HPJava type signature.  The concrete {\tt Array\ldots} classes
implement a series of abstract interfaces.  These follow a similar
naming convention, but the root of their names is {\tt Section} rather
than {\tt Array} (so {\tt Array2dcI}, for example, implements {\tt
Section2dcI}).  The hierarchy of {\tt Section} interfaces is
illustrated in Figure \ref{fig:arrays}.
The need to introduce the {\tt Section} interfaces should be evident
from the hierarchy diagram.  The type hierarchy of HPJava involves a
kind of multiple inheritance.  The array type {\tt int [[*, *]]}, for
example, is a specialization of both the types {\tt int [[*, ]]}
and {\tt int [[, *]]}.  Java allows ``multiple inheritance'' only from
interfaces, not classes.

\begin{figure}[tbp]
  \begin{center}
    \leavevmode
    \epsfig{figure=./zgs/arrays.eps,height=2.4in} 
    \caption{The adJava {\tt Section} hierarchy}
    \label{fig:arrays}
  \end{center}
\end{figure}

Important members of the {\tt Section} interfaces include
inquiry functions {\tt dat()}, which returns an ordinary one
dimensional Java array used to store the locally held elements of the
distributed array, and the member {\tt pos(i, ...)}, which takes $n$
subscript arguments and returns the local offset of the element implied by
those subscripts.  Each argument of {\tt pos} is a location or an
integer (only allowed if the corresponding dimension is collapsed).
These functions are used to implement elemental subscripting.
The inquiry {\tt grp()} returns the group over which
elements of the array are distributed.  The inquiry {\tt rng(d)}
returns the $d$th range of the array.

Another package in adJava is the communication library.
The adJava communication package includes classes corresponding to the
various collective communication schedules provided in the NPAC PCRC
kernel.  Most of them provide a constructor to establish a schedule,
and an \texttt{execute} method, which carries out the data movement
specified by the schedule.  Different
communication models may eventually be added through further packages.

The collective communication schedules can be used directly by
the programmer or invoked through certain wrapper functions.
A class named \texttt{Adlib} is defined
with static members that create and execute
communication schedules and perform simple I/O functions. 
This class includes, for example, the following
methods, each implemented by
constructing the appropriate schedule and then executing it:
\begin{small}
\begin{verbatim}
  static void remap(Section dst, Section src)
  static void shift(Section dst, Section src, int shift, int dim, int mode)
  static void copy(Section dst, Section src)
  static void writeHalo(Section src, int [] wlo, int [] whi, int [] mode)
\end{verbatim}
\end{small}
\texttt{Adlib.remap} will copy the
corresponding elements from one array to another, regardless of their
respective distribution format.  \texttt{Adlib.shift} will shift data
by a certain amount in a specific dimension of the array, in either
cyclic or edge-off mode.  \texttt{Adlib.writeHalo} is used to update
ghost regions.

Given the classes described above, one can program in the HPspmd style
in pure Java program.  Then the idea of the HPJava compiler is just to
translate the HPJava program onto this interface, in the meanwhile, to use
optimized address calculation instead of function calls to access
elements of arrays. (For a detailed translation scheme, please refer to
\cite{Zha98}.)


\section{Programming Examples}
\label{sec:examples}

\begin{figure}[tbp]
  \begin{center}
\begin{small}
\begin{verbatim}
  Procs1 p = new Procs1(4);
  on(p) {
    Range x = new CyclicRange(n, p.dim(0));
    float a[[*,]] = new float [[n, x]];
    ... some code to initialise `a' ...
    float b[[*]] = new float [[n]]; // buffer

    for(int k = 0 ; k < N - 1 ; k++) {
      at(l = x[k]) {
        float d =  Math.sqrt(a[k,l]) ;
        a[k,l] = d ;
        for(int s = k + 1 ; s < N ; s++)
          a[s,l] /= d ;
      }
      Adlib.remap(b[[k+1:]], a[[k+1:,  k]]);

      overall(m = x | k + 1 : )
        for(int i = x.idx(m) ; i < N ; i++)
          a[i,m] -= b[i] * b[x.idx(m)] ;
    }
    at(l = x [N - 1])
      a[N - 1,l] = Math.sqrt(a[N-1,l]) ;
  }
\end{verbatim}
\end{small}
    \caption{Choleski decomposition.}
    \label{fig:choleski}
  \end{center}
\end{figure}

In this section we give two example programs to show the new
language features.  The first example is the Choleski decomposition, seen in
Figure \ref{fig:choleski}.
Here, \texttt{remap} is used to broadcast one updated column to each
process. The function \texttt{idx} gets the global index of location
\texttt{m} relative to the parent range \texttt{x}.
The second example is the Jacobi iteration, seen in Figure \ref{fig:Jacobi}.
In the displayed code there is only one iteration, but it demonstrates
how to define range references with ghost areas, how to use
the \texttt{writeHalo} function, and how to use shifted locations
as subscripts. 


\begin{figure}[tbp]
  \begin{center}
\begin{small}
\begin{verbatim}
  Procs2 p = new Procs2(2, 4);
  Range x = new BlockRange(100, p.dim(0), 1),
        y = new BlockRange(200, p.dim(1), 1); 
  on(p) {
    float [[,]] a = new float [[x,y]], b = new float [[x,y]];
    ... some code to initialize `a'

    Adlib.writeHalo(a);

    overall(i = x | : )
      overall(j = y | : )
        b[i,j] = 0.25 * (a[i-1,j] + a[i+1,j] + a[i,j-1] + a[i,j+1]);
    overall(i = x | : )
      overall(j = y | : )
        a[i,j] = b[i,j];
  }
\end{verbatim}
\end{small}
    \caption{Jacobi relaxation.}
    \label{fig:Jacobi}
  \end{center}
\end{figure}


\section{Issues in the Language Design}
\label{sec:design}

With some of the implementation mechanisms exposed, we can
better discuss the language design itself.

\subsection{Extending the Java Language}

The first question to answer is why use Java as a base language?
Actually, the programming model embodied in HPJava is largely language-independent.  
It can be bound to other languages like C, C++, and Fortran.
But Java is a convenient base language, especially for initial
experiments, because it provides full object-orientation --- convenient
for describing complex distributed data --- implemented in a relatively
simple setting, conducive to development of source-to-source
translators.  It has been noted elsewhere that Java has various
features suggesting it could be an attractive language for science
and engineering \cite{Java98}.

With Java as a base language, an obvious question is whether we can
extend the language by simply adding packages, instead of changing the
syntax.  There are two problems with doing this for data-parallel
programming.

Our baseline is HPF, and any package supporting parallel arrays as
general as HPF is likely to be cumbersome to code with.  Our run-time
system needs an (in principle) infinite series of class names
\begin{small}
\begin{verbatim}
   Array1dI, Array1cI, Array2ddI, Array2dcI, ...
\end{verbatim}
\end{small}
to express the HPJava types
\begin{small}
\begin{verbatim}
  int [[]], int [[*]], int [[,]], int [[,*]] ... 
\end{verbatim}
\end{small}
as well as the corresponding series for \texttt{char}, \texttt{float}, and so on.
To access an element of a distributed array in HPJava, one writes
\begin{small}
\begin{verbatim}
  a[i] = 3 ;
\end{verbatim}
\end{small}
In the adJava interface, it must be written as
\begin{small}
\begin{verbatim}
  a.dat()[a.pos(i)] = 3 ;
\end{verbatim}
\end{small}
This is only for simple subscripting. Constructing array
sections will be even more complex using the raw class library
interface.

The second problem is that a Java program using a package like adJava
in a direct, naive way will have very poor performance, because all
the local address of the global array are expressed by functions such
as \texttt{pos}.  An optimization pass is needed to transform offset
computation to a more intelligent style.  So if a preprocessor must do
these optimizations anyway, it makes sense to design a syntax to
express the concepts of the programming model more naturally.

\subsection{Why not HPF?}

The design of the HPJava language is strongly influenced by HPF.
The language emerged partly out of practices adopted in our efforts to
implement an HPF compilation system \cite{Zha97}.  For example:
\begin{small}
\begin{verbatim}
  !HPF$ POCESSOR    P(4)
  !HPF$ TEMPLET     T(100)
  !HPF$ DISTRIBUTE  T(BLOCK) ONTO P
        REAL        A(100,100), B(100)
  !HPF$ ALIGN       A(:,*) WITH T(:)
  !HPF$ ALIGN       B WITH T
\end{verbatim}
\end{small}
have their conterparts in HPJava:
\begin{small}
\begin{verbatim}
  Procs1 p = new Procs1(4);
  Range x = new BlockRange (100, p.dim(0));
  float [[,*]] a = new float [[x,100]] on p;
  float [[ ]] b = new float [[x]] on p;
\end{verbatim}
\end{small}
Both languages provide a globally addressed name space for data parallel
applications.  Both of them can specify how data are mapped on to a
processor grid.  The difference between the two lies in their
communication aspects.  In HPF, a simple assignment statement may cause data
movement.  For example, given the above distribution, the assignment
\begin{small}
\begin{verbatim}
  A(10,10) = B(30)
\end{verbatim}
\end{small}
will cause communication between processor 1 and 2.  In HPJava, similar
communication must be done through explicit function calls: \footnote{By default
Fortran array subscripts starts from 1, while HPJava global subscripts always
start from 0.}
\begin{small}
\begin{verbatim}
  Adlib.remap(a[[9,9]], b[[29]]);
\end{verbatim}
\end{small}
Experience from compiling the HPF language suggests that, while there
are various kinds of algorithms to detect communication automatically,
it is often difficult to give the generated node program acceptable
performance.  In HPF, the need to decide on which processor the
computation should be executed further complicates the situation.  One
may apply ``owner computes'' or ``majority computes'' rules to
partition computation, but these heuristics are difficult to apply in
many situations.

In HPJava, the SPMD programming model is emphasized.  The distributed
arrays just help the programmer organize data, and simplify
global-to-local address translation.  The tasks of computation
partition and communication are still under control of the programmer.
This is certainly an extra onus, and the language may be more difficult to
program than HPF; \footnote{The program must meet SPMD constraints, e.g.,
only the owner of an element can access that data.  Run-time checking
can be added automatically to ensure such conditions are met.} but
it helps programmers to understand the performance of the program much
better than in HPF, so algorithms exploiting locality and parallelism
are encouraged.  It also dramatically simplifies the work of the
compiler.

Because the communication sector is considered an ``add-on'' to the
basic language, HPJava should interoperate more smoothly than HPF with
other successful SPMD libraries, including MPI \cite{MPIStandard},
Global Arrays \cite{Nie96}, CHAOS \cite{Das94},
%DAGH \cite{HDDA_DAGH}, 
and so on.

\subsection{Datatypes in HPJava}

In a parallel language, it is desirable to have both local variables
(like the ones in MPI programming) and global variables (like
the ones in HPF programming).  The former provide flexibility and are
ideal for task parallel programming; the latter are convenient
especially for data parallel programming.

In HPJava, variable names are divided into two sets.  In general those
declared using ordinary Java syntax represent local variables and those
declared with \texttt{\#} or \texttt{[[]]} represent global variables.
The two sectors are independent.  In the implementation of HPJava the
global variables have special data descriptors associated with them,
defining how their components are divided or replicated across
processes.  The significance of the data descriptor is most obvious
when dealing with procedure calls.
Passing array sections to procedure calls is an important component in
the array processing facilities of Fortran90 \cite{Fortran90}.  The data
descriptor of Fortran90 will include stride information for each array
dimension.  One can assume that HPF needs a much more complex kind of
data descriptor to allow passing distributed arrays across procedure
boundaries.  In either case the descriptor is not visible to the
programmer.  Java has a more explicit data descriptor concept; its
arrays are considered as objects, with, for example, a publicly
accessible \texttt{length} field.  In HPJava, the data descriptors for
global data are similar to those used in HPF, but more explicitly
exposed to programmers.  Inquiry functions such as \texttt{grp} and
\texttt{rng} have a similar role in global data as the field
\texttt{length} in an ordinary Java array.

Keeping two data sectors seems to complicate the language and its
syntax.  But it provides convenience for both task and data parallel
processing.  There is no need for things like the \texttt{LOCAL}
mechanism in HPF to call a local procedure on the node processor.  The
descriptors for ordinary Java variables are unchanged in HPJava.  On
each node processor ordinary Java data will be used as local varables,
as in an MPI program.


\section{Projects in Progress\label{sec:progress}}

Projects related to this work include development of MPI, HPF,
and other parallel languages such as ZPL\cite{ZPL} and Spar\cite{Spar}.
Here, we explain the background and future developments of our own
project.

The work originated in our compilation practices for HPF. As described
in \cite{Zha97}, our compiler emphasizes run-time support.
Adlib\cite{Car98}, a PCRC run-time library, provides a
rich set of collective communication functions.  It was realized that
by raising the run-time interface to the user level, a rather
straightforward (compared to HPF) compiler could be developed to
translate the high level language code to a node program calling the
run-time functions.

Currently, a Java interface has been implemented on top of the Adlib
library.  With classes such as \texttt{Group}, \texttt{Range}, and
\texttt{Location} in the Java interface, one can write Java programs
quite similar to HPJava proposed here. But a program executed in
this way will have large overhead due to function calls (such as
address translation) when accessing data inside loop constructs.
Given the knowledge of data distribution plus inquiry functions inside
the run-time library, one can substitute address translation calls with
linear operations on the loop variable, and keep inquiry
function calls ``outside the loop.'' 

At the present time, we are implementing the translator.  Further
research work will include optimization and safety-checking techniques
in the compiler for HPspmd programming.

Figure \ref{fig:performance} shows a preliminary benchmark for a hand-translated 
versions of our examples. The parallel programs are executed
on 4 sparc-sun-solaris2.5.1 with MPICH and the Java JIT compiler in
JDK 1.2Beta2. For Jacobi iteration, the timing is for about 90
iterations; the array size is 1024X1024.

\begin{figure}[tbp]
  \begin{center}
    \leavevmode
    \epsfig{figure=./zgs/Jacobi4.eps, height=1.8in}
    \caption{Preliminary performance.}
    \label{fig:performance}
  \end{center}
\end{figure}

We also compared the sequential Java, C++, and Fortran version of the
code, all with \texttt{-O} flag when compiling. The dotted lines shown
in the figure only represent times for the one processor case. We can
see that on a single processor, Java program use language supported
mechanism for calculating array element address; hence, it is slower than
HPJava (as HPJava uses an optimized address calculation scheme). We emphasize again that in the
picture we are comparing {\em sequential} Fortran, etc., with {\em
parallel} HPJava.  This is not supposed to be an comparative evaluation
of the various languages.  It is just supposed to give an impression of
the performance ballpark Java is currently operating in.

\section{Summary\label{summary.tex}}

Through the simple examples in this chapter we have tried to
illustrate that the programming language presented here provides the
flexibility of SPMD programming, and much of the convenience of HPF.
The language helps programmers to express parallel algorithms in a
more explicit way.  We suggest it will help programmers to solve real
application problems more easily, compared with using communication
packages such as MPI directly, and allow the compiler writer to
implement the language compiler without the difficulties met in the HPF
compilation.

The overall structure of the system is shown in Figure
\ref{fig:hpspmd}.  The central DAD stands for the Distributed Array
Descriptor.  Around it, we have different run-time libraries. The Java
interface is most relevant here.  But the Java binding is only an
introduction of the programming style.  (A Fortran binding is being
developed.)  Initially, the Java version can be used as a software tool
for teaching parallel programming.  As Java for scientific computation
becomes more mature, it will be a practical programming language to
solve real application problems in parallel and distributed
environments.

\begin{figure}[tbp]
  \begin{center}
    \leavevmode
    \epsfig{figure=./zgs/HPspmdLayers.eps} 
    \caption{Layers of the HPspmd model.}
    \label{fig:hpspmd}
  \end{center}
\end{figure}


%--------------------

%\bibliographystyle{unsrt}
%\bibliographystyle{plain}
%\bibliography{zgs}

%%Postscript encapsulation macro(s)
\input epsf
%\putfig{filespec.eps}{scale}
\def\putfig#1#2{\mbox{ \def\epsfsize##1##2{#2##1} \epsfbox{#1}}}

\chapter[The HPspmd Model and its Java Binding]{The {\em HPspmd} Model and its Java Binding}
\label{zgs:chap}
\author[Guansong Zhang, Bryan Carpenter, Geoffrey Fox,\\
  Xinying Li, and Yuhong Wen]{Guansong Zhang, Bryan Carpenter, Geoffrey Fox,\\
  Xinying Li, and Yuhong Wen}
\affil{NPAC at Syracuse University\\
        Syracuse, NY, 13 244 \\
	\vspace{0.1in}
        Email: {\em \{zgs,dbc,gcf,xli,wen\}@npac.syr.edu}}

%\noindent
%{\bf Abstract
%{\small\em
%This chapter introduces a new language, \emph{HPJava}, for parallel
%programming on message passing systems, including workstation
%clusters.  The language provides a high level SPMD programming model,
%called \emph{HPspmd}.  Through examples and performance results, the
%features of the new programming style, and its implementation, are
%illustrated.  
%} 
%}

%\vspace{0.5cm}

%\noindent
%{\it Keywords:}
%{\small Java, data parallel programming, SPMD}


%--------------------


\section{Introduction}

In this chapter we introduce the \inxx{HPJava} \emph{HPJava}
language, a programming language that extends Java for parallel
programming on message passing systems --- from multiprocessor systems to
workstation clusters.

HPJava owes much to High Performanace Fortran (HPF)
\cite{HPFStandard}.  Its model of data distribution is adapted directly
from the HPF model.  The heritage of HPF can be traced back to Fortran
dialects that were implemented most successfully on SIMD and other
tightly-coupled MPP architectures.  While it was always a goal of the
HPF designers that the language should be efficiently implementable on
the more loosely coupled MIMD clusters that dominate today, the
complexity of the language --- and notably the design goal of emulating
exactly the semantics of a sequential Fortran program --- have made
efficient implementation on today's architectures quite hard.

HPJava, in contrast, starts from the assumption that the target
hardware is a set of interacting MIMD processors, and exposes that
assumption explictly in its programming model.  This greatly
simplifies the task of the compiler, and increases the chance of
obtaining efficient implementations on architectures including PC and
workstation clusters.  Instead of the HPF programming model, the
language introduces a high-level structured SPMD programming
style --- the \inxx {HPspmd} HPspmd model.  A program written in
this class of language explicitly coordinates well-defined process
groups.  These cooperate in a loosely synchronous manner, sharing
logical threads of control.  As in a conventional distributed-memory
SPMD program, only a process owning a data item such as an array
element is allowed to access the item directly.  The language provides
special constructs that allow programmers to meet this constraint
conveniently.

Besides the normal variables of the sequential base language, the
language model introduces classes of global variables that are stored
collectively across process groups.  Primarily, these are
distributed arrays.  They provide a global name space in the form of
globally subscripted arrays, with assorted distribution patterns.  This
helps to relieve programmers of error-prone activities such as the
local-to-global, global-to-local subscript translations which occur in
data parallel applications.

In addition to special data types, the language provides special
constructs to facilitate both data parallel and task parallel
programming.  Through these constructs different processors can work either
simultaneously on globally addressed data, or independently to execute
complex procedures on locally held data.  The conversion between these
phases is seamless.

In the traditional SPMD mold, the language itself does not provide
implicit data movement semantics.  This greatly simplifies the task of
the compiler, and should encourage programmers to use algorithms that
exploit locality.  Data on remote processors is accessed exclusively
through explicit library calls.  In particular, the initial HPJava
implementation relies on a library of collective communication
routines originally developed as part of an HPF run-time library.
Other distributed-array-oriented communication libraries may be
bound to the language later.  Due to the explicit SPMD programming
model, low level MPI communication is always available as a fallback.
The language itself only provides basic concepts to organize data
arrays and process groups.  Different communication patterns are
implemented as library functions.  This allows the possibility that if
a new communication pattern is needed, it is relatively easily
integrated through new libraries.

In our earlier work on HPF compilation \cite{Zha97} the role of run-time
support was emphasized.  Difficulties in compiling HPF efficiently
suggested to make the run-time communication library directly visible in
the programming model.  Since Java is a simple, elegant language,
we are implementing our prototype based upon this language.

Section \ref{sec:javaBinding} reviews the HPspmd model in the context
of the HPJava language.  Section \ref{sec:package} describes the class
library packages used in code generated by the HPJava translator, and
thus exposes many of the implementation issues.  Some examples of simple
algorithms expressed in HPJava are given in Section
\ref{sec:examples}.  Then Section \ref{sec:design} discusses the
rationale of various design decisions in the language.  The status of
the project and future goals are summarized in Sections
\ref{sec:progress} and \ref{summary.tex}.



\section{Java Language Binding\label{sec:javaBinding}} 

This section introduces the HPJava language.
HPJava contains the whole of standard Java as a subset.  It adds
various built-in classes for describing process groups and index ranges,
new global data types, and some syntax for accessing distributed data
and specifying which processes execute particular statements.

\subsection{Basic Concepts}

Key concepts in the programming model are built around the 
\inxx{group} 
\inxx{group, process groups}
process groups used to describe program execution control in a parallel
program.
\texttt{Group} is a class representing a process group, typically with
a grid structure and an associated set of process dimensions.
It has its subclasses that represent different grid dimensionalities,
such as \texttt{Procs1}, \texttt{Procs2}, etc. For example,
\small
\begin{verbatim}
  Procs2 p = new Procs2(2,4);
\end{verbatim}
\normalsize
{\tt p} is a 2-dimensional, 2 x 4 grid of processes.

The second category of concepts is associated with 
\inxx{range}
\inxx{range, distributed ranges}
distributed ranges.
The elements of an ordinary array can be represented by an array name and
an integer sequence.  There are two parameters associated with this
sequence: an index to access each array element, and the extent of the
range this index can be chosen from.  In describing a distributed
array, HPJava introduces two new kinds of entity to represent the
analogous concepts.  A {\em range} maps an integer interval into a
process dimension according to certain distribution format. Ranges
describe the extent and the mapping of array dimensions.  A {\em
location}, or slot, is an abstract element of a range.
For example,
\small
\begin{verbatim}
  Range x = new BlockRange(100, p.dim(0)) ;
  Range y = new CyclicRange(200, p.dim(1)) ;
\end{verbatim}
\normalsize
creates two ranges distributed over the two process dimensions of the
group \texttt{p}.  One is block distributed, the other is cyclic
distributed.  There are 100 different locations in the range \texttt{x}.
The first one, for example, is
\small
\begin{verbatim}
  x [0]
\end{verbatim}
\normalsize

Additional related concepts are 
\inxx{group, subgroups}
subgroups
and 
\inxx{range, subranges}
subranges.  A subgroup is some slice of a
process array, formed by restricting the process coordinates in one or
more dimensions to single values.  Suppose {\tt i} is a location in a
range distributed over a dimension of group {\tt p}. The expression
\small
\begin{verbatim}
  p / i
\end{verbatim}
\normalsize
represents a smaller group---the slice of {\tt p} to which location
{\tt i} is mapped.  
Similarly, a subrange is a section of a range, parameterized by
a global index triplet. Logically, it represents a subset of the
locations of the original range.
The syntax for a subrange expression is
\small
\begin{verbatim}
  x [1 : 49]
\end{verbatim}
\normalsize
The symbol ``\texttt{:}'' is a special separator.  It is used to
construct a Fortran-90 style triplet, with lower- and upper-bound
expressions defining an integer subset.  The optional third member of a
triplet is a stride.

When a process grid is defined, certain ranges and locations are
also implicitly defined.  As shown in Figure \ref{fig:group},
two primitive ranges are associated with dimensions of the group
\texttt{p}:
\small
\begin{verbatim}
  Range u = p.dim(0);  
  Range v = p.dim(1);
\end{verbatim}
\normalsize
\texttt{dim()} is a member function that returns a range reference,
directly representing a processor dimension.
We can obtain a location in range \texttt{v}, and use it
to create a new group,
\small
\begin{verbatim}
  Group q = p / v [1] ;
\end{verbatim}
\normalsize
\begin{figure}[tbp]
  \begin{center}
    \leavevmode
    \epsfig{figure=./zgs/group.eps,height=1.5in}
    \caption{Structured process group.}
    \label{fig:group}
  \end{center}
\end{figure}

In a traditional SPMD program, execution control is based on
{\em if} statements and process ID or rank numbers. In the new
programming language, switching execution control is based on the
structured process group.  For example, it is not difficult to guess
that the following code:
\small
\begin{verbatim}
  on(p) {
    ...
  }
\end{verbatim}
\normalsize
will restrict the execution control inside the bracket to processes in
group \texttt{p}.
The language also provided well-defined constructs to split
execution control across processes according to data items we want to
access.  This will be discussed later.

\subsection{Global Variables}

When an SPMD program starts on a group of $n$ processes, there will be
$n$ control threads mapped to $n$ physical processors.
In each control thread, the program can define variables in the same
way as in a sequential program.  The variables created in this way are
local variables.  Their names may be common to all
processes, but they will be accessed individually (their scope is
local to a process).

Besides local variables, HPJava allows a program to define
global variables, explicitly mapped to a process group.  A
global variable will be treated by the process group that created it
as a single entity. The language has special syntax for the definition
of global data.  Global variables are all defined by using the
\texttt{new} operator from free storage.  When a global variable is
created, a
\inxx{data descriptor} data descriptor 
is also allocated to describe where the data are
held.  On a single processor, an array variable might be parametrized
by a simple record containing a memory address and an \texttt{int}
value for its length.  On a multiprocessor, a more complicated
structure is needed to describe a distributed array.  The data
descriptor specifies where the data is created, and how they are
distributed.  The logical structure of a descriptor is shown in Figure
\ref{fig:descriptor}.

\begin{figure}[tbp]
  \begin{center}
    \leavevmode
    \epsfig{figure=./zgs/descriptor.eps}
    \caption{Descriptor.}
    \label{fig:descriptor}
  \end{center}
\end{figure}

\begin{figure}[bp]
  \begin{center}
    \leavevmode
    \epsfig{figure=./zgs/map.eps}
    \caption{Memory mapping.}
    \label{fig:mapping}
  \end{center}
\end{figure}

HPJava has special syntax to define global data.  The statement
\small
\begin{verbatim}
  int # s = new int # on p ;
\end{verbatim}
\normalsize
creates a global scalar replicated over process group {\tt p}. In the
statement, \texttt{s} is a data descriptor handle --- a
\inxx{global scalar reference} global scalar reference.  
The scalar contains an integer value.
Global scalar references can be defined for any primitive type (or,
in principle, class type) of Java.
The symbol \texttt{\#} in the type signature
distinguishes a global scalar from a primitive integer.
For a global scalar, a field \texttt{value} is used to access the
value:
\small
\begin{verbatim}
  on(p) {
    int # s = new int # ;
    s.value = 100 ;
  }
\end{verbatim}
\normalsize
Note how the \texttt{on} clause can be omitted from the constructor:
the whole of the active process group is the default distribution group.
Figure \ref{fig:mapping} shows a possible memory mapping for this
scalar on different processes. 
Note, the value field of \texttt{s} is identical in each process in the
distribution group.   Replicated value variables are
different from local variables with identical names.  The
associated descriptors can be used to ensure the value is maintained
identically in each process, throughout program execution.

When defining a global array, it is not necessary to allocate a data
descriptor for each array element---one descriptor suffices for the whole
array.
An array can defined with various kinds of range, introduced earlier.
Suppose we have, as before,
\small
\begin{verbatim}
  Range x = new BlockRange(100, p.dim(0)) ;
\end{verbatim}
\normalsize
and the process group defined in Figure \ref{fig:group}, then
\small
\begin{verbatim}
  float [[]] a = new float [[x]] on q ;
\end{verbatim}
\normalsize
will create a global array with range \texttt{x} on group \texttt{q}.
Here \texttt{a} is a descriptor handle describing a one-dimensional
array of \texttt{float}. It is block distributed on group
\texttt{q}.\footnote{The \texttt{on} clause restricts the data owner
group of the array to \texttt{q}. If group \texttt{p} is used instead,
the one-dimenstional array will be replicated in the first dimenstion of
the group, and block distributed over the second dimension.} In
HPJava \texttt{a} is called a 
\inxx{global array reference} global or distributed
array reference.

A distributed array range can also be collapsed (or
sequential).  An integer range is specified, eg
\small
\begin{verbatim}
  float [[*]] b = new float [[100]] ;
\end{verbatim}
\normalsize
When defining an array with collapsed dimensions an asterisk is
normally added in the type signatures to mark the collapsed
dimensions.

The typical method of accessing global array elements is not exactly
the same as for local array elements, or for global scalar references.
In distributed dimensions of arrays
we must use named locations as subscripts, for example
\small
\begin{verbatim}
  at(i = x [3])
    a [i] = 3 ;
\end{verbatim}
\normalsize
We will leave discussion of the {\em at} construct to Section
\ref{control}, and give a simpler example here:
if a global array is defined with a collapsed dimension, accessing
its elements is modelled on local arrays.  For example:
\small
\begin{verbatim}
  for(int i = 0 ; i < 100 ; i++)
    b [i] = i ;
\end{verbatim}
\normalsize
assigns the loop index to each corresponding element in the array.

When defining a multidimensional global array, a single descriptor
parametrizes a rectangular array of any dimension:
\small
\begin{verbatim}
  Range x = new BlockRange(100, p.dim(0)) ;  
  Range y = new CyclicRange(100, p.dim(1)) ;
  float [[,]] c = new float [[x, y]];
\end{verbatim}
\normalsize
This creates a two-dimension global array with the first dimension
block distributed and the second cyclic distributed.  Now \texttt{c} is a
global array reference.  Its elements can be accessed using
single brackets with two suitable locations inside.

The global array introduced here is a true multidimensional
array, not a Java-like array-of-arrays.
Java-style arrays-of-arrays are still useful. For example, one can
define a local array of distributed arrays:
\small
\begin{verbatim}
  int[] size = {100, 200, 400};
  float [[,]] d[] = new float [size.length][[,]] ;
  Range x[], y[];
  for (int l = 0; l < size.length; l++) {
    const int n = size [l] ;
    x[l] = new BlockRange(n, p.dim(0)) ;  
    y[l] = new BlockRange(n, p.dim(1)) ;  
    d[l] = new float [[x[l], y[l]]];
  }
\end{verbatim}
\normalsize
This creates the stack of distributed arrays shown in Figure \ref{fig:layer}.

\begin{figure}[tbp]
  \begin{center}
    \leavevmode
    \epsfig{figure=./zgs/layer.eps}
    \caption{Array of distributed arrays.}
    \label{fig:layer}
  \end{center}
\end{figure}


Like Fortran 90, HPJava allows construction of sections of global
arrays.  The syntax of section subscripting uses double brackets.  The
subscripts can be scalar (integers or locations) or triplets.

Suppose we have array \texttt{a} and \texttt{c} defined as
above.  Then \texttt{a[[i]]}, \texttt{c}, \texttt{c[[i, 1::2]]},
and \texttt{c[[i, :]]} are all array sections.  Here, {\tt i} is an
integer in the appropriate interval (it could also be a
location in the first range of {\tt a} and {\tt c}).
Both the expressions \texttt{c[[i, 1::2]]} and \texttt{c[[i, :]]}
represent one-dimensional distributed arrays, providing aliases for
subsets of the elements in \texttt{c}.  The expression \texttt{a[[i]]}
contains a single element of \texttt{a}, but the result is a global
scalar reference (unlike the expression \texttt{a[i]} which is a simple
variable).

Array section expressions are often used as arguments in function
calls.\footnote{When used in method calls, the collapsed dimension
array is a \emph{subtype} of the ordinary one, i.e., an argument of
\texttt{float[[*,*]]}, \texttt{float[[*,]]} and \texttt{float[[,*]]}
type can all be passed to a dummy of type \texttt{float[[,]]}.
The converse is not true.}  Table \ref{tab:section} shows the type
signatures of global data with different dimensions.

\begin{table}[tbp]
  \begin{center}
    \caption{Section Expression and Type Signature}
    \vspace{0.1in}
    \leavevmode
    \small
    \begin{tabular}[c]{|l|ll|} 
      \hline
      \textbf{global var}& \textbf{array section} & \textbf{type} \\ 
      \hline
      2-dimension & \texttt{c} & \texttt{float [[,]]} \\ 
      & \texttt{c[[:,:]]} & \texttt{float [[,]]} \\ 
      \hline
      1-dimension & \texttt{c[[i,:]]} & \texttt{float [[]]} \\ 
      & \texttt{c[[i,1::2]]} & \texttt{float [[]]} \\ 
      \hline
      scalar(0-dim) & \texttt{c[[i,j]]} & \texttt{float \#} \\ 
      \hline
    \end{tabular}
    \normalsize
    \label{tab:section}
  \end{center}
\end{table}

The size of an array in Java can be had from its
\texttt{length} field.  In HPJava, information like the
distributed group and distributed dimensions can be accessed from the
following inquiries, available on all global array types:
\small
\begin{verbatim}
  Group grp()       // distribution group

  Range rng(int d)  // d'th range
\end{verbatim}
\normalsize
Further inquiry functions on {\tt Range} yield values such as
extents and distribution formats.

\subsection{Program Execution Control}
\label{control}

HPJava has all the conventional Java statements for execution control
within a single process.  It introduces three new control
constructs: {\em on}, {\em at}, and {\em overall} for execution
control across processes.
A new concept, the active process group, is introduced. It is the
set of processes sharing the current thread of control.

In a traditional SPMD program, switching the active
process group is effectively implemented by {\em if} statements
such as:
\small
\begin{verbatim}
  if(myid >= 0 && myid < 4) {
    ...
  }
\end{verbatim}
\normalsize
Inside the braces, only processes numbered 0 to 3
share the control thread.
In HPJava, this effect is expressed using a \texttt{Group}.
When a HPJava program starts, the active process group has a
system-defined value.  During the execution, the active process group
can be changed explicitly through an {\em on} construct in the
program.

In a shared memory program, accessing the value of a variable is
straightforward. In a message passing system, only the process which
holds data can read and write the data.  We sometimes call this 
\inxx{SPMD constraint} SPMD constraint.  
A traditional SPMD program respects this constraint by using an idiom
like
\small
\begin{verbatim}
  if(myid == 1) 
    my_data = 3 ;
\end{verbatim}
\normalsize
The {\em if} statement makes sure that only \texttt{my\_data}
on process 1 is assigned as 3.

In the language we present here, similar constraints must be respected.
Besides \emph{on} construct introduced earlier, there is a
convenient way to change the active process group to access a required
array element, namely, the {\em at} construct.
Suppose array \texttt{a} is defined as in the previous section, then:
\small
\begin{verbatim}
  on(q) {
    a [1] = 3 ;    // error

    at(j = x [1])
      a [j] = 3 ;  // correct
  }
\end{verbatim}
\normalsize
The assignment statement guarded by an {\em at} construct is
correct; the one without is likely to imply access to an element
not held locally.  Formally it is illegal because, in a simple
subscripting operation, an integer expression cannot
be used to subscript a distributed dimension.  The
{\em at} construct introduces a new variable \texttt{j}, a {\em named
location}, with scope only inside the block controlled by the
{\em at}.  Named locations are the only legal element subscripts 
in distributed dimensions.

A more powerful construct called {\em overall}
combines restriction of the active process group with a loop:
\small
\begin{verbatim}
  on(q)
    overall(i = x | 0 : 3)
      a [i] = 3 ;
\end{verbatim}
\normalsize
is essentially equivalent to\footnote{A compiler can implement
overall construct in a more efficient way, using linearized
address calculation.  For detailed translation schemes for the
overall construct, please refer to \cite{Car98}.}
\small
\begin{verbatim}
  on(q) 
    for(int n = 0 ; n < 4 ; n++)
      at(i = x [n])
        a [i] = 3;
\end{verbatim}
\normalsize
In each iteration, the active process group is changed to \texttt{q /
i}.  In Section \ref{sec:examples}, we will illustrate with further
programs how {\em at} and {\em overall} constructs conveniently
allow one to keep the active process group equal to the data owner
group for the assigned data.

\subsection{Communication Library Functions}
\label{communication}

When accessing data on another process, HPJava needs explicit
communication, as in a normal SPMD program. 
Communication libraries are provided as packages in HPJava. Detailed
function specifications are given elsewhere.  The next
section will introduce a small number of top level collective
communication functions.


\section{Java Packages for HPspmd Programming}
\label{sec:package}

The implementation of the HPJava compiler is based on a run-time system.
It is actually a source-to-source translator converting an HPJava
program to a Java node program, with function calls to the run-time
library, called \emph{adJava}.

The run-time interface consists of several Java packages. The most
important one is the HPspmd runtime proper.  It includes the classes
needed to translate language constructs.  Other packages provide
communication and some simple I/O functions.  Important classes in the
first package include distributed array ``container classes'' and
related classes describing process groups and index ranges.  These
classes correspond directly to HPJava built-in classes.

The first hierarchy is based on \texttt{Group}.  A group, or
process group, defines some subset of the processes executing the SPMD
program.  They can be used to describe how program variables,
such as arrays, are distributed or replicated across the process pool,
or to specify which subset of processes executes a particular code
fragment.  Important members of adJava {\tt Group} class include the
pair {\tt on()}, {\tt no()} used to translate the {\em on} construct.

The most common way to create a group object is through the
constructor for one of the subclasses representing a process
grid.  The subclass \texttt{Procs} represents a grid of processes
and carries information on process dimensions: in particular an
inquiry function {\tt dim(r)} returns a range object describing the
$r$-th process dimension.  {\tt Procs} is further subclassed by
\texttt{Procs0}, \texttt{Procs1}, \texttt{Procs2}, \ldots which
provide simpler constructors for fixed dimensionality process grids.

The second hierarchy in the package is based on \texttt{Range}.
A range is a map from the integer interval $0,\ldots,n-1$ into
some process dimension (i.e., some dimension of a process grid).
Ranges are used to parametrize distributed arrays and the
{\em overall} distributed loop.

The most common way to create a range object is to use the constructor
for one of the subclasses representing ranges with specific
distribution formats.  
Simple block distribution format is implemented by
{\tt BlockRange}, while \texttt{CyclicRange} and \texttt{BlockCyclicRange}
represent other standard distribution formats of HPF.  The subclass
\texttt{CollapsedRange} represents a sequential (undistributed range).
Finally, a \texttt{DimRange} is associated with each process dimension
and represents the range of coordinates for the process dimension
itself---just one element is mapped to each process.

The related adJava class \texttt{Location} represents an individual
location in a particular distributed range.  Important members of the
adJava {\tt Range} class include the function {\tt location(i)}, which
returns the $i$th location in a range, and its inverse, {\tt idx(l)},
which returns the global subscript associated with a given location.
Important members of the {\tt Location} class include {\tt at()} and
{\tt ta()}, used in the implementation of the HPJava that {\em at}
construct.

Finally, we have the rather complex hierarchy of classes representing
distributed arrays.  HPJava global arrays declared using
\texttt{[[]]} are represented by Java objects belonging to classes
such as:
\begin{small}
\begin{verbatim}
   Array1dI, Array1cI,
   Array2ddI, Array2dcI, Array2cdI, Array2ccI,
   ...
   Array1dF, Array1cF,
   Array2ddF, Array2dcF, Array2cdF, Array2ccF,
   ...
\end{verbatim}
\end{small}
Generally speaking, the class
``\texttt{Array}\textit{n}\texttt{c|d}\textit{{\ldots}T}''
represents $n$-dimensional distributed arrays with elements of type
{\em T}---currently one of {\tt I}, {\tt F}, \ldots, meaning {\tt int},
{\tt float}, \ldots. \footnote{In the inital implementation, the element
  type is restricted to the Java primitive types.}  The penultimate
part of the class name is a string of $n$ ``c''s and ``d''s specifying
whether each dimension is collapsed or normally distributed.  These
correlate with presence or absence of an asterisk in slots of the
HPJava type signature.  The concrete {\tt Array\ldots} classes
implement a series of abstract interfaces.  These follow a similar
naming convention, but the root of their names is {\tt Section} rather
than {\tt Array} (so {\tt Array2dcI}, for example, implements {\tt
Section2dcI}).  The hierarchy of {\tt Section} interfaces is
illustrated in Figure \ref{fig:arrays}.
The need to introduce the {\tt Section} interfaces should be evident
from the hierarchy diagram.  The type hierarchy of HPJava involves a
kind of multiple inheritance.  The array type {\tt int [[*, *]]}, for
example, is a specialization of both the types {\tt int [[*, ]]}
and {\tt int [[, *]]}.  Java allows ``multiple inheritance'' only from
interfaces, not classes.

\begin{figure}[tbp]
  \begin{center}
    \leavevmode
    \epsfig{figure=./zgs/arrays.eps,height=2.4in} 
    \caption{The adJava {\tt Section} hierarchy}
    \label{fig:arrays}
  \end{center}
\end{figure}

Important members of the {\tt Section} interfaces include
inquiry functions {\tt dat()}, which returns an ordinary one
dimensional Java array used to store the locally held elements of the
distributed array, and the member {\tt pos(i, ...)}, which takes $n$
subscript arguments and returns the local offset of the element implied by
those subscripts.  Each argument of {\tt pos} is a location or an
integer (only allowed if the corresponding dimension is collapsed).
These functions are used to implement elemental subscripting.
The inquiry {\tt grp()} returns the group over which
elements of the array are distributed.  The inquiry {\tt rng(d)}
returns the $d$th range of the array.

Another package in adJava is the communication library.
The adJava communication package includes classes corresponding to the
various collective communication schedules provided in the NPAC PCRC
kernel.  Most of them provide a constructor to establish a schedule,
and an \texttt{execute} method, which carries out the data movement
specified by the schedule.  Different
communication models may eventually be added through further packages.

The collective communication schedules can be used directly by
the programmer or invoked through certain wrapper functions.
A class named \texttt{Adlib} is defined
with static members that create and execute
communication schedules and perform simple I/O functions. 
This class includes, for example, the following
methods, each implemented by
constructing the appropriate schedule and then executing it:
\begin{small}
\begin{verbatim}
  static void remap(Section dst, Section src)
  static void shift(Section dst, Section src, int shift, int dim, int mode)
  static void copy(Section dst, Section src)
  static void writeHalo(Section src, int [] wlo, int [] whi, int [] mode)
\end{verbatim}
\end{small}
\texttt{Adlib.remap} will copy the
corresponding elements from one array to another, regardless of their
respective distribution format.  \texttt{Adlib.shift} will shift data
by a certain amount in a specific dimension of the array, in either
cyclic or edge-off mode.  \texttt{Adlib.writeHalo} is used to update
ghost regions.

Given the classes described above, one can program in the HPspmd style
in pure Java program.  Then the idea of the HPJava compiler is just to
translate the HPJava program onto this interface, in the meanwhile, to use
optimized address calculation instead of function calls to access
elements of arrays. (For a detailed translation scheme, please refer to
\cite{Zha98}.)


\section{Programming Examples}
\label{sec:examples}

\begin{figure}[tbp]
  \begin{center}
\begin{small}
\begin{verbatim}
  Procs1 p = new Procs1(4);
  on(p) {
    Range x = new CyclicRange(n, p.dim(0));
    float a[[*,]] = new float [[n, x]];
    ... some code to initialise `a' ...
    float b[[*]] = new float [[n]]; // buffer

    for(int k = 0 ; k < N - 1 ; k++) {
      at(l = x[k]) {
        float d =  Math.sqrt(a[k,l]) ;
        a[k,l] = d ;
        for(int s = k + 1 ; s < N ; s++)
          a[s,l] /= d ;
      }
      Adlib.remap(b[[k+1:]], a[[k+1:,  k]]);

      overall(m = x | k + 1 : )
        for(int i = x.idx(m) ; i < N ; i++)
          a[i,m] -= b[i] * b[x.idx(m)] ;
    }
    at(l = x [N - 1])
      a[N - 1,l] = Math.sqrt(a[N-1,l]) ;
  }
\end{verbatim}
\end{small}
    \caption{Choleski decomposition.}
    \label{fig:choleski}
  \end{center}
\end{figure}

In this section we give two example programs to show the new
language features.  The first example is the Choleski decomposition, seen in
Figure \ref{fig:choleski}.
Here, \texttt{remap} is used to broadcast one updated column to each
process. The function \texttt{idx} gets the global index of location
\texttt{m} relative to the parent range \texttt{x}.
The second example is the Jacobi iteration, seen in Figure \ref{fig:Jacobi}.
In the displayed code there is only one iteration, but it demonstrates
how to define range references with ghost areas, how to use
the \texttt{writeHalo} function, and how to use shifted locations
as subscripts. 


\begin{figure}[tbp]
  \begin{center}
\begin{small}
\begin{verbatim}
  Procs2 p = new Procs2(2, 4);
  Range x = new BlockRange(100, p.dim(0), 1),
        y = new BlockRange(200, p.dim(1), 1); 
  on(p) {
    float [[,]] a = new float [[x,y]], b = new float [[x,y]];
    ... some code to initialize `a'

    Adlib.writeHalo(a);

    overall(i = x | : )
      overall(j = y | : )
        b[i,j] = 0.25 * (a[i-1,j] + a[i+1,j] + a[i,j-1] + a[i,j+1]);
    overall(i = x | : )
      overall(j = y | : )
        a[i,j] = b[i,j];
  }
\end{verbatim}
\end{small}
    \caption{Jacobi relaxation.}
    \label{fig:Jacobi}
  \end{center}
\end{figure}


\section{Issues in the Language Design}
\label{sec:design}

With some of the implementation mechanisms exposed, we can
better discuss the language design itself.

\subsection{Extending the Java Language}

The first question to answer is why use Java as a base language?
Actually, the programming model embodied in HPJava is largely language-independent.  
It can be bound to other languages like C, C++, and Fortran.
But Java is a convenient base language, especially for initial
experiments, because it provides full object-orientation --- convenient
for describing complex distributed data --- implemented in a relatively
simple setting, conducive to development of source-to-source
translators.  It has been noted elsewhere that Java has various
features suggesting it could be an attractive language for science
and engineering \cite{Java98}.

With Java as a base language, an obvious question is whether we can
extend the language by simply adding packages, instead of changing the
syntax.  There are two problems with doing this for data-parallel
programming.

Our baseline is HPF, and any package supporting parallel arrays as
general as HPF is likely to be cumbersome to code with.  Our run-time
system needs an (in principle) infinite series of class names
\begin{small}
\begin{verbatim}
   Array1dI, Array1cI, Array2ddI, Array2dcI, ...
\end{verbatim}
\end{small}
to express the HPJava types
\begin{small}
\begin{verbatim}
  int [[]], int [[*]], int [[,]], int [[,*]] ... 
\end{verbatim}
\end{small}
as well as the corresponding series for \texttt{char}, \texttt{float}, and so on.
To access an element of a distributed array in HPJava, one writes
\begin{small}
\begin{verbatim}
  a[i] = 3 ;
\end{verbatim}
\end{small}
In the adJava interface, it must be written as
\begin{small}
\begin{verbatim}
  a.dat()[a.pos(i)] = 3 ;
\end{verbatim}
\end{small}
This is only for simple subscripting. Constructing array
sections will be even more complex using the raw class library
interface.

The second problem is that a Java program using a package like adJava
in a direct, naive way will have very poor performance, because all
the local address of the global array are expressed by functions such
as \texttt{pos}.  An optimization pass is needed to transform offset
computation to a more intelligent style.  So if a preprocessor must do
these optimizations anyway, it makes sense to design a syntax to
express the concepts of the programming model more naturally.

\subsection{Why not HPF?}

The design of the HPJava language is strongly influenced by HPF.
The language emerged partly out of practices adopted in our efforts to
implement an HPF compilation system \cite{Zha97}.  For example:
\begin{small}
\begin{verbatim}
  !HPF$ POCESSOR    P(4)
  !HPF$ TEMPLET     T(100)
  !HPF$ DISTRIBUTE  T(BLOCK) ONTO P
        REAL        A(100,100), B(100)
  !HPF$ ALIGN       A(:,*) WITH T(:)
  !HPF$ ALIGN       B WITH T
\end{verbatim}
\end{small}
have their conterparts in HPJava:
\begin{small}
\begin{verbatim}
  Procs1 p = new Procs1(4);
  Range x = new BlockRange (100, p.dim(0));
  float [[,*]] a = new float [[x,100]] on p;
  float [[ ]] b = new float [[x]] on p;
\end{verbatim}
\end{small}
Both languages provide a globally addressed name space for data parallel
applications.  Both of them can specify how data are mapped on to a
processor grid.  The difference between the two lies in their
communication aspects.  In HPF, a simple assignment statement may cause data
movement.  For example, given the above distribution, the assignment
\begin{small}
\begin{verbatim}
  A(10,10) = B(30)
\end{verbatim}
\end{small}
will cause communication between processor 1 and 2.  In HPJava, similar
communication must be done through explicit function calls: \footnote{By default
Fortran array subscripts starts from 1, while HPJava global subscripts always
start from 0.}
\begin{small}
\begin{verbatim}
  Adlib.remap(a[[9,9]], b[[29]]);
\end{verbatim}
\end{small}
Experience from compiling the HPF language suggests that, while there
are various kinds of algorithms to detect communication automatically,
it is often difficult to give the generated node program acceptable
performance.  In HPF, the need to decide on which processor the
computation should be executed further complicates the situation.  One
may apply ``owner computes'' or ``majority computes'' rules to
partition computation, but these heuristics are difficult to apply in
many situations.

In HPJava, the SPMD programming model is emphasized.  The distributed
arrays just help the programmer organize data, and simplify
global-to-local address translation.  The tasks of computation
partition and communication are still under control of the programmer.
This is certainly an extra onus, and the language may be more difficult to
program than HPF; \footnote{The program must meet SPMD constraints, e.g.,
only the owner of an element can access that data.  Run-time checking
can be added automatically to ensure such conditions are met.} but
it helps programmers to understand the performance of the program much
better than in HPF, so algorithms exploiting locality and parallelism
are encouraged.  It also dramatically simplifies the work of the
compiler.

Because the communication sector is considered an ``add-on'' to the
basic language, HPJava should interoperate more smoothly than HPF with
other successful SPMD libraries, including MPI \cite{MPIStandard},
Global Arrays \cite{Nie96}, CHAOS \cite{Das94},
%DAGH \cite{HDDA_DAGH}, 
and so on.

\subsection{Datatypes in HPJava}

In a parallel language, it is desirable to have both local variables
(like the ones in MPI programming) and global variables (like
the ones in HPF programming).  The former provide flexibility and are
ideal for task parallel programming; the latter are convenient
especially for data parallel programming.

In HPJava, variable names are divided into two sets.  In general those
declared using ordinary Java syntax represent local variables and those
declared with \texttt{\#} or \texttt{[[]]} represent global variables.
The two sectors are independent.  In the implementation of HPJava the
global variables have special data descriptors associated with them,
defining how their components are divided or replicated across
processes.  The significance of the data descriptor is most obvious
when dealing with procedure calls.
Passing array sections to procedure calls is an important component in
the array processing facilities of Fortran90 \cite{Fortran90}.  The data
descriptor of Fortran90 will include stride information for each array
dimension.  One can assume that HPF needs a much more complex kind of
data descriptor to allow passing distributed arrays across procedure
boundaries.  In either case the descriptor is not visible to the
programmer.  Java has a more explicit data descriptor concept; its
arrays are considered as objects, with, for example, a publicly
accessible \texttt{length} field.  In HPJava, the data descriptors for
global data are similar to those used in HPF, but more explicitly
exposed to programmers.  Inquiry functions such as \texttt{grp} and
\texttt{rng} have a similar role in global data as the field
\texttt{length} in an ordinary Java array.

Keeping two data sectors seems to complicate the language and its
syntax.  But it provides convenience for both task and data parallel
processing.  There is no need for things like the \texttt{LOCAL}
mechanism in HPF to call a local procedure on the node processor.  The
descriptors for ordinary Java variables are unchanged in HPJava.  On
each node processor ordinary Java data will be used as local varables,
as in an MPI program.


\section{Projects in Progress\label{sec:progress}}

Projects related to this work include development of MPI, HPF,
and other parallel languages such as ZPL\cite{ZPL} and Spar\cite{Spar}.
Here, we explain the background and future developments of our own
project.

The work originated in our compilation practices for HPF. As described
in \cite{Zha97}, our compiler emphasizes run-time support.
Adlib\cite{Car98}, a PCRC run-time library, provides a
rich set of collective communication functions.  It was realized that
by raising the run-time interface to the user level, a rather
straightforward (compared to HPF) compiler could be developed to
translate the high level language code to a node program calling the
run-time functions.

Currently, a Java interface has been implemented on top of the Adlib
library.  With classes such as \texttt{Group}, \texttt{Range}, and
\texttt{Location} in the Java interface, one can write Java programs
quite similar to HPJava proposed here. But a program executed in
this way will have large overhead due to function calls (such as
address translation) when accessing data inside loop constructs.
Given the knowledge of data distribution plus inquiry functions inside
the run-time library, one can substitute address translation calls with
linear operations on the loop variable, and keep inquiry
function calls ``outside the loop.'' 

At the present time, we are implementing the translator.  Further
research work will include optimization and safety-checking techniques
in the compiler for HPspmd programming.

Figure \ref{fig:performance} shows a preliminary benchmark for a hand-translated 
versions of our examples. The parallel programs are executed
on 4 sparc-sun-solaris2.5.1 with MPICH and the Java JIT compiler in
JDK 1.2Beta2. For Jacobi iteration, the timing is for about 90
iterations; the array size is 1024X1024.

\begin{figure}[tbp]
  \begin{center}
    \leavevmode
    \epsfig{figure=./zgs/Jacobi4.eps, height=1.8in}
    \caption{Preliminary performance.}
    \label{fig:performance}
  \end{center}
\end{figure}

We also compared the sequential Java, C++, and Fortran version of the
code, all with \texttt{-O} flag when compiling. The dotted lines shown
in the figure only represent times for the one processor case. We can
see that on a single processor, Java program use language supported
mechanism for calculating array element address; hence, it is slower than
HPJava (as HPJava uses an optimized address calculation scheme). We emphasize again that in the
picture we are comparing {\em sequential} Fortran, etc., with {\em
parallel} HPJava.  This is not supposed to be an comparative evaluation
of the various languages.  It is just supposed to give an impression of
the performance ballpark Java is currently operating in.

\section{Summary\label{summary.tex}}

Through the simple examples in this chapter we have tried to
illustrate that the programming language presented here provides the
flexibility of SPMD programming, and much of the convenience of HPF.
The language helps programmers to express parallel algorithms in a
more explicit way.  We suggest it will help programmers to solve real
application problems more easily, compared with using communication
packages such as MPI directly, and allow the compiler writer to
implement the language compiler without the difficulties met in the HPF
compilation.

The overall structure of the system is shown in Figure
\ref{fig:hpspmd}.  The central DAD stands for the Distributed Array
Descriptor.  Around it, we have different run-time libraries. The Java
interface is most relevant here.  But the Java binding is only an
introduction of the programming style.  (A Fortran binding is being
developed.)  Initially, the Java version can be used as a software tool
for teaching parallel programming.  As Java for scientific computation
becomes more mature, it will be a practical programming language to
solve real application problems in parallel and distributed
environments.

\begin{figure}[tbp]
  \begin{center}
    \leavevmode
    \epsfig{figure=./zgs/HPspmdLayers.eps} 
    \caption{Layers of the HPspmd model.}
    \label{fig:hpspmd}
  \end{center}
\end{figure}


%--------------------

%\bibliographystyle{unsrt}
%\bibliographystyle{plain}
%\bibliography{zgs}

%\input{zgs.bbl}

\begin{thebibliography}{}

\bibitem{Fortran90}
Jeanne~C. Adams, Walter~S. Brainerd, Jeanne~T. Martin, Brian~T. Smith, and
  Jerrold~L. Wagener.
\newblock {\em Fortran 90 Handbook}.
\newblock McGraw-Hill, 1992.

\bibitem{Car98}
Bryan Carpenter, Guansong Zhang, and Yuhong Wen.
\newblock {NPAC} {PCRC} Runtime Kernel Definition.
\newblock Technical Report CRPC-TR97726. Center for Research on Parallel
  Computation, 1997.
\newblock Up-to-date version maintained at
  \url{http://www.npac.syr.edu/projects/pcrc}.

\bibitem{Das94}
R.~Das, M.~Uysal, J.H. Salz, and Y.-S. Hwang.
\newblock Communication Optimizations for Irregular Scientific Computations on
  Distributed Aemory Architectures.
\newblock {\em Journal of Parallel and Distributed Computing}, vol. 22(3), pages 462--479,
  September 1994.

\bibitem{HPFStandard}
High Performance~Fortran Forum.
\newblock {High} {Performance} {Fortran} Language Specification.
\newblock {\em Scientific Programming}, special issue, February 1993.

\bibitem{MPIStandard}
Message Passing~Interface Forum.
\newblock {\em {MPI}: A Message-Passing Interface Standard}.
\newblock University of Tenessee, Knoxville, TN, June 1995.
\newblock \url{http://www.mcs.anl.gov/mpi}.

\bibitem{Java98}
Geoffrey~C. Fox, editor.
\newblock {\em {ACM} 1998 Workshop on {Java} for High-Performance Network
  Computing}. February 1998.

\bibitem{Nie96}
J.~Nieplocha, R.J. Harrison, and R.J. Littlefield.
\newblock The {Global Array}: Non-Uniform-Memory-Access Programming Model for
  High-Performance Computers.
\newblock {\em The Journal of Supercomputing}, vol. 10, pages 197--220, 1996.

\bibitem{ZPL}
Lawrence Snyder.
\newblock A {ZPL} Programming Guide.
\newblock Technical Report, University of Washington, May 1997.
\newblock \url{http://www.cs.washington.edu/research/projects/zpl/}.

\bibitem{Spar}
Kees van Reeuwijk, Arjan J.~C. van Gemund, and Henk~J. Sips.
\newblock {Spar}: A Programming Language for Semi-Automatic Compilation of
  Parallel Programs.
\newblock {\em Concurrency: Practice and Experience}, vol. 9(11), pages 1193--1205, 1997.

\bibitem{Zha97}
Guansong Zhang, Bryan Carpenter, Geoffrey Fox, Xiaoming Li, Xinying Li, and
  Yuhong Wen.
\newblock {PCRC}-Based {HPF} Compilation.
\newblock In {\em 10th International Workshop on Languages and Compilers for
  Parallel Computing}, August 1997.
\newblock To appear in Lecture Notes in Computer Science.

\bibitem{Zha98}
Guansong Zhang, Bryan Carpenter, Geoffrey Fox, Xinying Li, and Yuhong Wen.
\newblock Considerations in {HPJava} Language Design and Implementation.
\newblock In {\em 11th International Workshop on Languages and Compilers for
  Parallel Computing}, August 1998.
\newblock To appear in Lecture Notes in Computer Science.

\end{thebibliography}



\begin{thebibliography}{}

\bibitem{Fortran90}
Jeanne~C. Adams, Walter~S. Brainerd, Jeanne~T. Martin, Brian~T. Smith, and
  Jerrold~L. Wagener.
\newblock {\em Fortran 90 Handbook}.
\newblock McGraw-Hill, 1992.

\bibitem{Car98}
Bryan Carpenter, Guansong Zhang, and Yuhong Wen.
\newblock {NPAC} {PCRC} Runtime Kernel Definition.
\newblock Technical Report CRPC-TR97726. Center for Research on Parallel
  Computation, 1997.
\newblock Up-to-date version maintained at
  \url{http://www.npac.syr.edu/projects/pcrc}.

\bibitem{Das94}
R.~Das, M.~Uysal, J.H. Salz, and Y.-S. Hwang.
\newblock Communication Optimizations for Irregular Scientific Computations on
  Distributed Aemory Architectures.
\newblock {\em Journal of Parallel and Distributed Computing}, vol. 22(3), pages 462--479,
  September 1994.

\bibitem{HPFStandard}
High Performance~Fortran Forum.
\newblock {High} {Performance} {Fortran} Language Specification.
\newblock {\em Scientific Programming}, special issue, February 1993.

\bibitem{MPIStandard}
Message Passing~Interface Forum.
\newblock {\em {MPI}: A Message-Passing Interface Standard}.
\newblock University of Tenessee, Knoxville, TN, June 1995.
\newblock \url{http://www.mcs.anl.gov/mpi}.

\bibitem{Java98}
Geoffrey~C. Fox, editor.
\newblock {\em {ACM} 1998 Workshop on {Java} for High-Performance Network
  Computing}. February 1998.

\bibitem{Nie96}
J.~Nieplocha, R.J. Harrison, and R.J. Littlefield.
\newblock The {Global Array}: Non-Uniform-Memory-Access Programming Model for
  High-Performance Computers.
\newblock {\em The Journal of Supercomputing}, vol. 10, pages 197--220, 1996.

\bibitem{ZPL}
Lawrence Snyder.
\newblock A {ZPL} Programming Guide.
\newblock Technical Report, University of Washington, May 1997.
\newblock \url{http://www.cs.washington.edu/research/projects/zpl/}.

\bibitem{Spar}
Kees van Reeuwijk, Arjan J.~C. van Gemund, and Henk~J. Sips.
\newblock {Spar}: A Programming Language for Semi-Automatic Compilation of
  Parallel Programs.
\newblock {\em Concurrency: Practice and Experience}, vol. 9(11), pages 1193--1205, 1997.

\bibitem{Zha97}
Guansong Zhang, Bryan Carpenter, Geoffrey Fox, Xiaoming Li, Xinying Li, and
  Yuhong Wen.
\newblock {PCRC}-Based {HPF} Compilation.
\newblock In {\em 10th International Workshop on Languages and Compilers for
  Parallel Computing}, August 1997.
\newblock To appear in Lecture Notes in Computer Science.

\bibitem{Zha98}
Guansong Zhang, Bryan Carpenter, Geoffrey Fox, Xinying Li, and Yuhong Wen.
\newblock Considerations in {HPJava} Language Design and Implementation.
\newblock In {\em 11th International Workshop on Languages and Compilers for
  Parallel Computing}, August 1998.
\newblock To appear in Lecture Notes in Computer Science.

\end{thebibliography}



\begin{thebibliography}{}

\bibitem{Fortran90}
Jeanne~C. Adams, Walter~S. Brainerd, Jeanne~T. Martin, Brian~T. Smith, and
  Jerrold~L. Wagener.
\newblock {\em Fortran 90 Handbook}.
\newblock McGraw-Hill, 1992.

\bibitem{Car98}
Bryan Carpenter, Guansong Zhang, and Yuhong Wen.
\newblock {NPAC} {PCRC} Runtime Kernel Definition.
\newblock Technical Report CRPC-TR97726. Center for Research on Parallel
  Computation, 1997.
\newblock Up-to-date version maintained at
  \url{http://www.npac.syr.edu/projects/pcrc}.

\bibitem{Das94}
R.~Das, M.~Uysal, J.H. Salz, and Y.-S. Hwang.
\newblock Communication Optimizations for Irregular Scientific Computations on
  Distributed Aemory Architectures.
\newblock {\em Journal of Parallel and Distributed Computing}, vol. 22(3), pages 462--479,
  September 1994.

\bibitem{HPFStandard}
High Performance~Fortran Forum.
\newblock {High} {Performance} {Fortran} Language Specification.
\newblock {\em Scientific Programming}, special issue, February 1993.

\bibitem{MPIStandard}
Message Passing~Interface Forum.
\newblock {\em {MPI}: A Message-Passing Interface Standard}.
\newblock University of Tenessee, Knoxville, TN, June 1995.
\newblock \url{http://www.mcs.anl.gov/mpi}.

\bibitem{Java98}
Geoffrey~C. Fox, editor.
\newblock {\em {ACM} 1998 Workshop on {Java} for High-Performance Network
  Computing}. February 1998.

\bibitem{Nie96}
J.~Nieplocha, R.J. Harrison, and R.J. Littlefield.
\newblock The {Global Array}: Non-Uniform-Memory-Access Programming Model for
  High-Performance Computers.
\newblock {\em The Journal of Supercomputing}, vol. 10, pages 197--220, 1996.

\bibitem{ZPL}
Lawrence Snyder.
\newblock A {ZPL} Programming Guide.
\newblock Technical Report, University of Washington, May 1997.
\newblock \url{http://www.cs.washington.edu/research/projects/zpl/}.

\bibitem{Spar}
Kees van Reeuwijk, Arjan J.~C. van Gemund, and Henk~J. Sips.
\newblock {Spar}: A Programming Language for Semi-Automatic Compilation of
  Parallel Programs.
\newblock {\em Concurrency: Practice and Experience}, vol. 9(11), pages 1193--1205, 1997.

\bibitem{Zha97}
Guansong Zhang, Bryan Carpenter, Geoffrey Fox, Xiaoming Li, Xinying Li, and
  Yuhong Wen.
\newblock {PCRC}-Based {HPF} Compilation.
\newblock In {\em 10th International Workshop on Languages and Compilers for
  Parallel Computing}, August 1997.
\newblock To appear in Lecture Notes in Computer Science.

\bibitem{Zha98}
Guansong Zhang, Bryan Carpenter, Geoffrey Fox, Xinying Li, and Yuhong Wen.
\newblock Considerations in {HPJava} Language Design and Implementation.
\newblock In {\em 11th International Workshop on Languages and Compilers for
  Parallel Computing}, August 1998.
\newblock To appear in Lecture Notes in Computer Science.

\end{thebibliography}



\begin{thebibliography}{}

\bibitem{Fortran90}
Jeanne~C. Adams, Walter~S. Brainerd, Jeanne~T. Martin, Brian~T. Smith, and
  Jerrold~L. Wagener.
\newblock {\em Fortran 90 Handbook}.
\newblock McGraw-Hill, 1992.

\bibitem{Car98}
Bryan Carpenter, Guansong Zhang, and Yuhong Wen.
\newblock {NPAC} {PCRC} Runtime Kernel Definition.
\newblock Technical Report CRPC-TR97726. Center for Research on Parallel
  Computation, 1997.
\newblock Up-to-date version maintained at
  \url{http://www.npac.syr.edu/projects/pcrc}.

\bibitem{Das94}
R.~Das, M.~Uysal, J.H. Salz, and Y.-S. Hwang.
\newblock Communication Optimizations for Irregular Scientific Computations on
  Distributed Aemory Architectures.
\newblock {\em Journal of Parallel and Distributed Computing}, vol. 22(3), pages 462--479,
  September 1994.

\bibitem{HPFStandard}
High Performance~Fortran Forum.
\newblock {High} {Performance} {Fortran} Language Specification.
\newblock {\em Scientific Programming}, special issue, February 1993.

\bibitem{MPIStandard}
Message Passing~Interface Forum.
\newblock {\em {MPI}: A Message-Passing Interface Standard}.
\newblock University of Tenessee, Knoxville, TN, June 1995.
\newblock \url{http://www.mcs.anl.gov/mpi}.

\bibitem{Java98}
Geoffrey~C. Fox, editor.
\newblock {\em {ACM} 1998 Workshop on {Java} for High-Performance Network
  Computing}. February 1998.

\bibitem{Nie96}
J.~Nieplocha, R.J. Harrison, and R.J. Littlefield.
\newblock The {Global Array}: Non-Uniform-Memory-Access Programming Model for
  High-Performance Computers.
\newblock {\em The Journal of Supercomputing}, vol. 10, pages 197--220, 1996.

\bibitem{ZPL}
Lawrence Snyder.
\newblock A {ZPL} Programming Guide.
\newblock Technical Report, University of Washington, May 1997.
\newblock \url{http://www.cs.washington.edu/research/projects/zpl/}.

\bibitem{Spar}
Kees van Reeuwijk, Arjan J.~C. van Gemund, and Henk~J. Sips.
\newblock {Spar}: A Programming Language for Semi-Automatic Compilation of
  Parallel Programs.
\newblock {\em Concurrency: Practice and Experience}, vol. 9(11), pages 1193--1205, 1997.

\bibitem{Zha97}
Guansong Zhang, Bryan Carpenter, Geoffrey Fox, Xiaoming Li, Xinying Li, and
  Yuhong Wen.
\newblock {PCRC}-Based {HPF} Compilation.
\newblock In {\em 10th International Workshop on Languages and Compilers for
  Parallel Computing}, August 1997.
\newblock To appear in Lecture Notes in Computer Science.

\bibitem{Zha98}
Guansong Zhang, Bryan Carpenter, Geoffrey Fox, Xinying Li, and Yuhong Wen.
\newblock Considerations in {HPJava} Language Design and Implementation.
\newblock In {\em 11th International Workshop on Languages and Compilers for
  Parallel Computing}, August 1998.
\newblock To appear in Lecture Notes in Computer Science.

\end{thebibliography}

