

\documentstyle[epsfig]{article}

\begin{document}

\bibliographystyle{plain}

\pagestyle{plain}

%\title{\vspace{2in}{\em HPJava}: data parallel extensions to Java}
\title{{\em HPJava}: data parallel extensions to Java}

\author{Bryan Carpenter, Guansong Zhang, Geoffrey Fox \\
        Xinying Li, Yuhong Wen\vspace{5mm} \\
        \em NPAC at Syracuse University \\
        \em Syracuse, NY 13244 \\
        \em \{dbc,zgs,gcf,xli,wen\}@npac.syr.edu}

\date{December 7, 1997}

\maketitle

\begin{abstract}
We outline an extension of Java for programming
with distributed arrays.  The basic programming style is Single Program
Multiple Data (SPMD), but parallel arrays are provided as new language
primitives.  Further extensions include three {\em distributed control}
constructs, the most important being a data-parallel loop construct.
Communications involving distributed arrays are handled through a
standard library of collective operations.  Because the underlying
programming model is SPMD programming, direct calls to MPI or other
communication packages are also allowed in an HPJava program.
\end{abstract}

%newpage

\section{Introduction}

The idea that Java may enable new programming environments, combining
attractive user interfaces with high performance computation, is gaining
increasing attention amongst computational scientists.  Java boasts a
direct simplicity reminiscent of Fortran, but also incorporates many of
the important ideas of modern object-oriented programming.  Of course
it comes with an established track-record in the domains of Web
and Internet programming.

This article will focus specifically on the potential of Java as a
language for scientific parallel programming.  We envisage a framework
called {\em HPJava}.  This would be a general environment for parallel
computation.  Ultimately it should combine tools, class libraries, and
language extensions to support various established paradigms for
parallel computation, including shared memory programming, explicit
message-passing, and array-parallel programming.  Other paradigms
(for example, Linda or coarse-grained data-flow) may come later, together with
bindings to higher-level libraries and application-specific libraries
such as CHAOS \cite{CHAOS}, ScaLAPACK \cite{ScaLAPACK},
Global Arrays \cite{Global_Arrays} or DAGH \cite{HDDA_DAGH}.

This is a large vision, and the current article only discusses some
first steps towards a general framework.  In particular we will make
specific proposals for the sector of HPJava most directly related to
its namesake: High Performance Fortran.  We will be concentrating on
array-parallel programming.

For now we do not propose to import
the full HPF programming model to Java.  After several years of effort by
various compiler groups, HPF compilers are still quite immature.
It seems difficult justify a comparable effort for Java before success
has been convincingly demonstrated in Fortran.  In any case there are
features of the HPF model that make it less attractive in the context of the
integrated parallel programming environment we envisage.
Although an HPF program {\em can} interoperate with modules
written in other parallel programming styles through the HPF extrinsic
procedure interface, that mechanism is quite awkward.
Rather than follow the HPF model directly, we propose introducing some
of the characteristic ideas of HPF---specifically its distributed array
model and array intrinsic functions and libraries---into a basically
SPMD programming model.  Because the programming model is SPMD, direct
calls to MPI \cite{Experiments_HPJava} or other communication
packages are allowed from the HPJava program.

The language outlined here provides HPF-like distributed
arrays as language primitives, and new {\em distributed control}
constructs to facilitate access to the local elements of these arrays.
In the SPMD mold, the model allows processors the freedom to
independently execute complex procedures on local elements: it is not
limited by SIMD-style array syntax.
All access to {\em non-local} array elements must go through
library functions---typically collective communication operations.
This puts an extra onus on the programmer; but making communication
explicit encourages the programmer to write algorithms that exploit
locality, and simplifies the task of the compiler writer.
On the other hand, by providing distributed arrays as language
primitives we are able to simplify error-prone tasks
such as converting between local and global array subscripts and
determining which processor holds a particular element.
As in HPF, it is possible to write programs at a natural level of
abstraction where the meaning is insensitive to the detailed mapping of
elements.  Lower-level styles of programming are also possible.

Our compiler will be implemented as a translator to ordinary Java with
calls to a suitable run-time library.  At the time of writing the
underlying library is already available \cite{NPAC_PCRC_kernel}, and
the Java interface needed by the translator is under development.  The
translator itself is being implemented in a compiler construction
framework developed in the PCRC project \cite{Common_runtime,PCRC_based}.

\section{Multidimensional arrays\label{multiDim}}

First we describe a modest extension to Java that adds a class of true
multi-dimensional arrays to the standard Java language.  The new arrays
allow regular section subscripting, similar to Fortran 90 arrays.  The
syntax described in this section is a subset of the syntax introduced
later for parallel arrays and algorithms: the only motivation for
discussing the sequential subset first is to simplify the overall
presentation.
No attempt is made to integrate the new multidimensional arrays
with the standard Java arrays: they are a new kind of entity that
coexists in the language with ordinary Java arrays.  There
are good technical reasons for keeping the two kinds of array
separate\footnote{The run-time representation of our multi-dimensional arrays
includes extra descriptor information that would simply
encumber the large class ``non-scientific'' Java applications.}.

The type-signatures and constructors of the multidimensional array use
double brackets to distinguish them from ordinary arrays:
\small
\begin{verbatim}
  int [[,]] a = new int [[5, 5]] ;

  float [[,,]] b = new float [[10, n, 20]] ;

  int [[]] c = new int [[100]] ;
\end{verbatim}
\normalsize
{\tt a}, {\tt b} and {\tt c} are respectively 2-, 3- and one- dimensional
arrays.  Of course {\tt c} is very similar in structure to the standard
array {\tt d}, created by
\small
\begin{verbatim}
  int [] d = new int [100] ;
\end{verbatim}
\normalsize
{\tt c} and {\tt d} are not identical, though\footnote{For example,
{\tt c} allows section subscripting, whereas {\tt d} does not.}.

Access to individual elements of a multidimensional array goes through
a subscripting operation involving single brackets, for example
\small
\begin{verbatim}
  for(int i = 0 ; i < 4 ; i++)
    a [i, i + 1] = i + c [i] ;
\end{verbatim}
\normalsize
For reasons that will become clearer in later sections, this style of
subscripting is called {\em local subscripting}.  In the current
sequential context, apart from the fact that a single pair of brackest
may include several comma-separated subscripts, this kind of
subscripting works just like ordinary Java array subscripting.
Subscripts always start at zero, in the ordinary Java or C style (there
is no Fortran-like lower bound).

In general our language has no idea of Fortran-like array assignments.  In
\small
\begin{verbatim}
  int [[,]] e = new int [[n, m]] ;
  ...
  a = e ;
\end{verbatim}
\normalsize
the assignment simply copies a handle to object referenced by {\tt e}
into {\tt a}.  There is no element-by-element copy involved.  Similarly we
introduce no idea of elemental arithmetic or elemental function application.
If {\tt e} and {\tt a} are arrays, the expressions
\small
\begin{verbatim}
  e + a
  Math.cos(e)
\end{verbatim}
\normalsize
are type errors.

Our HPJava {\em does} import a Fortran-90-like idea of array {\em regular
sections}.  The syntax for {\em section subscripting} is different to
the syntax for local subscripting.  Double brackets are used.  These
brackets can include scalar subscripts or {\em subscript triplets}.

A section is an object in its own right---its type is that of a suitable
multi-dimensional array.  It describes some subset of the elements of
the parent array.  This is slightly different to the situation in
Fortran, where sections cannot usually be captured as named
entities\footnote{Unless a section appears as an actual argument to
a procedure, in
which case the dummy argument names that section, or it is the target
of a pointer assignment.}.
\small
\begin{verbatim}
  int [[]] e = a [[2,  2 :]] ;

  foo(b [[ : , 0, 1 : 10 : 2]]) ;
\end{verbatim}
\normalsize
{\tt e} becomes an alias for the 3rd row of elements of {\tt a}.
The procedure {\tt foo} should expect a two-dimensional array as argument.
It can read or write to the set of elements of {\tt b} selected
by the section.
As in Fortran, upper or lower bounds can be omitted in triplets,
defaulting to the actual bound of the parent array, and the
stride entry of the triplet is optional.
The subscripts of {\tt e}, like any other array, start at 0,
although the first element is identified with {\tt a [2, 2]}.

In our language, unlike Fortran, it is not allowed to use vectors of
integers as subscripts.  The only sections
recognized are regular sections defined through scalar and triplet
subscripts.

The language provides a library of functions for manipulating its
arrays, closely analogous to the array transformational intrinsic
functions of Fortran 90:
\small
\begin{verbatim}
  int [[,]] f = new int [[5, 5]] ;
  HPJlib.shift(f, a, -1, 0, CYCL) ;

  float g = HPJlib.sum(b) ;

  int [[]] h = new int [[100]] ;
  HPJlib.copy(h, c) ;
\end{verbatim}
\normalsize
The {\tt shift} operation with shift-mode {\tt CYCL} executes a cyclic
shift on the data in its second argument, copying the result to its
first argument---an array of the same shape.  In the example
the shift amount is -1, and the shift is performed in dimension 0 of
the array---the first of its two dimensions.  The {\tt sum} operation
simply adds all elements of its argument array.  The {\tt copy} operation
copies the elements of its second argument to its first---it is something
like an array assignment.  These functions may have to be overloaded
to apply to some finite set of array types, eg they may be defined for
arrays with elements of any suitable Java primitive type, up to some maximum
rank of array.  Alternatively the type-hierarchy for arrays can be defined
in a way that allows these functions to be more polymorphic.

\section{Process arrays\label{procArrays}}

HPJava adds class libraries and some additional syntax for dealing
with {\em distributed arrays}.  These arrays are viewed as
coherent global entities, but their elements are divided
across a set of cooperating processes.  As a pre-requisite to
introducing distributed arrays we discuss the {\em process arrays} over
which their elements are scattered.

An abstract base class {\tt Procs}
has subclasses {\tt Procs1}, {\tt Procs2}, \ldots, representing
one-dimensional process arrays, two-dimensional process arrays, 
and so on.
\small
\begin{verbatim}
  Procs2 p = new Procs2(2, 2) ;
  Procs1 q = new Procs1(4) ;
\end{verbatim}
\normalsize
These declarations set {\tt p} to represent a 2 by 2 process array and {\tt q}
to represent a 4-element, one-dimensional process array.  In either case
the object created describes a group of 4 processes.
At the time the {\tt Procs} constructors are executed the program should be
executing on four or more processes.  Either constructor selects four
processes from this set and identifies them as members of
the constructed group\footnote{There is no cooperation between
the two constructor calls for {\tt p} and {\tt q}, so an individual
physical process might occur in both groups or in
neither.  As an option not illustrated here, vectors of
ids can be passed to the {\tt Procs} constructors to specify
exactly which processes are included in a particular group.}.

{\tt Procs} has a member function called {\tt member}, returning
a boolean value.  This is {\tt true} if the local process is a member
of the group, {\tt false} otherwise.
\small
\begin{verbatim}
  if(p.member()) {
    ...
  }
\end{verbatim}
\normalsize
The code inside the {\tt if} is executed only if the
local process is a member {\tt p}.  We will say that inside this
construct the {\em active process group} is restricted to {\tt p}.

The multi-dimensional structure of a process array is reflected in
its set of {\em process dimensions}.  An object is associated
with each dimension.  These objects are accessed through the
inquiry member {\tt dim}:
\small
\begin{verbatim}
  Dimension x = p.dim(0) ;
  Dimension y = p.dim(1) ;

  Dimension z = q.dim(0) ;
\end{verbatim}
\normalsize
The object returned by the {\tt dim} inquiry has class {\tt Dimension}.
The members of this class include the inquiry {\tt crd}.  This
returns the coordinate of the local process with respect to the
process dimension.  The result is only well-defined if the local
process is a member of the parent process array.  The inner
body code in
\small
\begin{verbatim}
  if(p.member())
    if(x.crd() == 0)
      if(y.crd() == 0) {
        ...
      }
\end{verbatim}
\normalsize
will only execute on the first process from {\tt p}, with coordinates
$(0, 0)$.


\section{Distributed arrays\label{distArrays}}

Some or all of the dimensions of a multi-dimensional array can be
declared to be {\em distributed ranges}.  In general a distributed
range is represented by an object of class {\tt Range}.  A {\tt Range}
object defines a range of integer subscripts, and defines how they are mapped
into a process array dimension.  In fact the {\tt Dimension}
class introduced in the previous section is a subclass of {\tt
Range}.  In this case the integer range is just the range of coordinate
values associated with the dimension.  Each value in the range
is mapped, of course, to the process (or slice of processes) with
that coordinate.  This kind of range is also called a {\em primitive range}.
More complex subclasses of {\tt Range} implement more
elaborate maps from integer ranges to process dimensions.
Some of these will be introduced in later sections.  For now we concentrate
on arrays
constructed with {\tt Dimension} objects as their distributed ranges.

The syntax of section \ref{multiDim} is extended in the following way
to support distributed arrays
\begin{itemize}
\item
A distributed range object may appear in place of an integer extent in the
``constructor'' of the array (the expression following the {\tt new}
keyword).
\item
If a particular dimension of the array has a distributed range, the
corresponding slot in the type signature of the array should include
a {\tt \#} symbol.
\item
In general the constructor of the distributed
array must be followed by an {\tt on} clause, specifying the process
group over which the array is distributed.  Distributed ranges
of the array must be distributed over distinct dimensions of this
group\footnote{The {\tt on} clause can be omitted in some circumstances---see
section \ref{on}.}.
\end{itemize}
Assume {\tt p},
{\tt x} and {\tt y} are declared as in the previous section, then
\small
\begin{verbatim}
  float [[#,#,]] a = new float [[x, y, 100]] on p ;
\end{verbatim}
\normalsize
defines {\tt a} as a 2 by 2 by 100 array of floating point
numbers.  Because the first two dimensions of the array are distributed
ranges---dimensions of {\tt p}---{\tt a} is actually realized as four
segments of 100 elements, one in each of the processes of {\tt p}.
The process in {\tt p} with coordinates {\tt i}, {\tt j}
holds the section {\tt a [[i, j, :]]}.

The distributed array {\tt a} is equivalent in terms of storage
to four local arrays defined by
\small
\begin{verbatim}
  float [] b = new float [100] ;
\end{verbatim}
\normalsize
But because {\tt a} is declared as a collective object we can apply
collective operations to it.  The {\tt HPJlib} functions introduced in
section \ref{multiDim} apply equally well to distributed arrays, but
now they imply inter-processor communication.
\small
\begin{verbatim}
  float [[#,#,]] a = new float [[x, y, 100]] on p,
                 b = new float [[x, y, 100]] on p ;

  HPJlib.shift(a, b, -1, 0, CYCL) ;
\end{verbatim}
\normalsize
The {\tt shift} operation causes the local values of {\tt a} to be overwritten
with values of {\tt b} from a processor adjacent in the {\tt x} dimension.

There is a catch in this.
When subscripting the distributed dimensions of an array it
is {\em simply disallowed} to use subscripts that refer to
off-processor elements.  While this:
\small
\begin{verbatim}
  int i = x.crd(), j = y.crd() ;

  a [i, j, 20] = a [i, j, 21] ;
\end{verbatim}
\normalsize
is allowed, this:
\small
\begin{verbatim}
  int i = x.crd(), j = y.crd() ;

  a [i, j, 20] = b [(i + 1) % 2, j, 20] ;
\end{verbatim}
\normalsize
is forbidden.  The second example could apparently be implemented using
a nearest neighbour communication, quite similar to the {\tt shift}
example above.  But our language imposes an strict policy
distinguishing it from most data parallel languages: while
library functions may introduce communications, language
primitives such as array subscripting {\em never} imply
communication.

If subscripting distributed dimensions is so restricted, why are the
{\tt i}, {\tt j} subscripts on the arrays needed at all?  In the
examples of this section these subscripts are only allowed one value on each
processor.
Well, the inconvience of specifying the subscripts will be reduced by
language constructs introduced later, and the fact that only one subscript
value is local is a special feature of the {\em primitive ranges}
used here.  The higher level distributed ranges introduced later
map multiple elements to individual processes.  Subscripting will no
longer look so redundant.

\section{The {\em on} construct and the active process group\label{on}}

In the section \ref{procArrays} the idiom 
\small
\begin{verbatim}
  if(p.member()) {
    ...
  }
\end{verbatim}
\normalsize
appeared.  Our language provides a short way of writing
this construct
\small
\begin{verbatim}
  on(p) {
    ...
  }
\end{verbatim}
\normalsize
In fact the {\em on} construct provides some extra value.
Informally we said in section \ref{procArrays} that the {\em active
process group} is restricted to {\tt p} inside the body of the {\tt
p.member()} conditional construct.  The language
incorporates a more formal idea of an active process group (APG).  At any
point of execution some process group is singled out as the APG.  An
{\tt on(p)} construct specifically changes the value of the APG to {\tt
p}.  On exit from the construct, the APG is restored to its value on
entry.

Elevating the active process group to a part of the language
allows some simplifications.  For example, it provides a natural
default for the {\tt on} clause in array constructors.
More importantly, formally defining the active process group
simplifies the statement of various rules about what operations are legal
{\em inside} distributed control constructs like {\em on}.

\section{Higher-level ranges and locations}

The class {\tt BlockRange} is a subclass of {\tt Range} which describes
a simple block-distributed range of subscripts.  Like {\tt BLOCK}
distribution format in HPF, it maps blocks of contiguous subscripts
to each element of its target process dimension\footnote{Other higher-level
ranges include {\tt CyclicRange}, which produces the equivalent
of {\tt CYCLIC} distribution format in HPF.}.  The constructor
of {\tt BlockRange} usually takes two arguments: the extent of the range
and a {\tt Dimension} object defining the process dimension over
which the new range is distributed.
\small
\begin{verbatim}
  Procs2 p = new Procs2(3, 2) ;
  
  Range x = new BlockRange(100, p.dim(0)) ;
  Range y = new BlockRange(200, p.dim(1)) ;

  float [[#,#]] a = new float [[x, y]] on p ;
\end{verbatim}
\normalsize
{\tt a} is created as a 100 $\times$ 200 array, block-distributed
over the 6 processes in {\tt p}.  The fragment is
essentially equivalent to the HPF declarations
\small
\begin{verbatim}
  !HPF$ PROCESSORS p(3, 2)
  
  REAL a(100, 200)

  !HPF$ DISTRIBUTE a(BLOCK, BLOCK) ONTO p
\end{verbatim}
\normalsize
Subscripting distributed arrays with non-primitive ranges
introduces some new problems.  An array access such as
\small
\begin{verbatim}
  a [17, 23] = 13 ;
\end{verbatim}
\normalsize
is perfectly legal {\em if} the local process holds the element in question.
But deterimining whether an element is local is no longer so easy.
When arrays had only {\em primitive} distributed ranges, it
was straightforward to check that accesses were local---the
subscript simply had to be equal to the local coordinate.  With
higher-level ranges, that simple condition no longer holds.

In practise it is unusual to use integer values directly as local
subscripts in distributed array dimensions.
Instead the idea of a {\em location} is introduced.  A
location can be viewed as an abstract element, or ``slot'', of a
distributed range.
Conversely, a range can be thought of as a set of locations.
This model of a range is visualized in
figure \ref{rangeSlots}.
An individual location is described by an object of the class {\tt Location}.
Each {\tt Location} element is mapped
to a particular slice of a process grid.
In general two locations are identical only
if they come from the same position in the same range.
A subscripting syntax is used to represent
location {\tt n} in range {\tt x}:
\small
\begin{verbatim}
  Location i = x [n]
\end{verbatim}
\normalsize

\begin{figure}[btp]
\centerline{\epsfig{figure=rangeSlots.eps,height=0.6in}}
\caption{\label{rangeSlots}A range regarded as a set of locations,
or slots.}
\end{figure}

This is an important idea in HPJava.  By working in terms of
abstract locations---elements of distributed ranges---one can usually
respect locality of reference without resorting explicitly to low-level
local subscripts and process ids.  In fact the location can be viewed
as an abstract data type incorporating these lower-level offsets.

Publically accessible fields of {\tt Location} include {\tt dim}
and {\tt crd}.  The first is the process dimension of the parent
range.  The second is coordinate in that dimension to which the element
is mapped.  So the access to element {\tt a [17, 23]}
could now be guarded by conditionals as follows:
\small
\begin{verbatim}
  Location i = x [17], j = y [23] ;

  if(i.crd == i.dim.crd())
    if(j.crd == j.dim.crd())
      a [17, 23] = 13 ;
\end{verbatim}
\normalsize
This is still quite verbose and error-prone.  The language provides a
second {\em distributed control construct} (analogous to {\em on}) to deal
with this common situation.  The new construct is called {\em at}, and
takes a location as its argument.  The
fragment above can be replaced with
\small
\begin{verbatim}
  Location i = x [17], j = y [23] ;

  at(i)
    at(j)
      a [17, 23] = 13 ;
\end{verbatim}
\normalsize
This is more concise, but still involves some redundancy because the
subscripts 17 and 23 appear twice.  A natural extension is to allow
locations to be used directly as array subscripts:
\small
\begin{verbatim}
  Location i = x [17], j = y [23] ;

  at(i)
    at(j)
      a [i, j] = 13 ;
\end{verbatim}
\normalsize
Locations used as array subscripts must be elements of the
corresponding ranges of the array.

The range class has a member function
\small
\begin{verbatim}
  int Range.idx(Location i)
\end{verbatim}
\normalsize
which can be used to recover the integer subscript, given a
location in the range.

There is a restriction that an {\tt at(i)} construct should only appear
at a point of execution where {\tt i.dim} is a dimension of the active
process group.  In the examples of this section this means that an {\tt
at(i)} construct, say, should normally be nested directly or indirectly
inside an {\tt on(p)} construct.

\section{Distributed loops}

Good parallel algorithms don't usually
expend many lines of code assigning to isolated elements of distributed
arrays.  The {\em at} mechanism of the previous section is
often useful, but a more pressing need is
a mechanism for {\em parallel} access to distributed array elements.
The last and most important distributed control construct in the language
is called {\em over}.
It implements a distributed parallel loop.  Conceptually it is quite
similar to the {\tt FORALL} construct of Fortran, except that the
{\em over} construct specifies exactly where its parallel iterations
are to be performed.
The argument of {\em over} is a member of the special class {\tt Index}.
The class {\tt Index} is a subclass of {\tt Location}, so it is syntactically
correct to use an index as an array subscript\footnote{But the effect
of such subscripting is only well-defined inside an {\em over} construct
parametrised by the index in question.}.  Here is an example of
a pair of nested {\em over} loops:
\small
\begin{verbatim}
  float [[#,#]] a = new float [[x, y]],
                b = new float [[x, y]] ;
  ...
  Index i, j ;
  over(i = x | :)
    over(j = y | :)
      a [i, j] = 2 * b [i, j] ;
\end{verbatim}
\normalsize
The body of an {\em over} construct executes, conceptually in parallel,
for every location in the range of its index (or some subrange if a
non-trivial triplet is specified)\footnote{Formally {\tt |} is being
used here as an operator that combines a range and a triplet to return
an object of the iterator class {\tt Index}.}.  An individual
``iteration'' executes on just those processors holding the location
associated with the iteration.  In a particular iteration, the location
component of the index (the base class object) is equal to that
location.  The net effect of the example above should be reasonably
clear.  It assigns twice the value of each element of {\tt b} to the
corresponding element of {\tt a}.  Because of the rules about {\em
where} an individual iteration iterates, the body of an {\em over} can
usually only combine elements of arrays that have some simple
alignment relation relative to one another.  The {\tt idx} member of
range can be used in parallel updates to give expressions that depend
on global index values.

With the {\em over} construct we can give some useful
examples of parallel programs.  Figure \ref{Jacobi} is the
famous Jacobi iteration for a two dimensionsional Laplace
equation.  We have used cyclic {\tt shift} to implement nearest
neighbour communications\footnote{Laplace's equation with
cyclic boundary conditions is not particularly useful,
but it illustrates the language features.
More interesting boundary conditions can easily be incorporated later.
Incidentally, this is a suitable place
to mention that the array arguments of {\tt shift} must be
{\em aligned} arrays---they must have identical distributed ranges.}.

\begin{figure}[btp]
\small
\begin{verbatim}
  Procs2 p = new Procs2(2, 2) ;

  on(p) {
    Range x = new BlockRange(100, p.dim(0)) ;
    Range y = new BlockRange(200, p.dim(1)) ;

    float [[#,#]] u = new float [[x, y]] ;

    // ... some code to initialise `u'

    float [[#,#]] unx = new float [[x, y]], upx = new float [[x, y]],
                  uny = new float [[x, y]], upy = new float [[x, y]] ;

    HPJlib.shift(unx, u,  1, 0, CYCL) ;
    HPJlib.shift(upx, u, -1, 0, CYCL) ;
    HPJlib.shift(uny, u,  1, 1, CYCL) ;
    HPJlib.shift(upy, u, -1, 1, CYCL) ;

    Index i, j ;
    over(i = x | :)
      over(j = y | :)
        u [i, j] = 0.25 * (unx [i, j] + upx [i, j] +
                           uny [i, j] + upy [i, j]) ;
  }
\end{verbatim}
\normalsize
\caption{\label{Jacobi}Jacobi iteration using {\tt shift}.}
\end{figure}

Copying whole arrays into temporaries is not an efficient way
of accessing nearest neighbours in an array.  Because this is
such a common pattern of communication, the standard library supports
{\em ghost regions}.  Distributed arrays can be created in such a way
that the segment stored locally is extended with some halo.  This halo
caches values stored in the segments of adjacent processes.  The cached
values are explicitly bought up to date by the library operation {\tt
writeHalo}.

An optimized version of the Jacobi program is give in figure
\ref{JacobiGhost}.  This version only involves a singe array temporary.
A new constructor for {\tt BlockRange} is provided.  This
allows the width of the ghost extensions to be specified.
The arguments of {\tt writeHalo} itself are an array with
suitable extensions and two vectors.  The first defines in each dimension
the width of the halo that must actually be updated, and the
second defines the treatment at the ends of the range---in
this case the ghost edges are updated with cyclic wraparound.
The new constructor and new {\tt writeHalo} function are simply
standard library extensions.  One new piece of syntax is needed:
the addition and subtraction operators are overloaded so that
integer offsets can be added or subtracted to {\tt Location}
objects, yielding new, shifted, locations.  The
usual access rules apply---this kind of shifted access is illegal
if it implies access to off-processor data.  It only works if
the subscripted array has suitable ghost extensions.

\begin{figure}[btp]
\small
\begin{verbatim}
  Procs2 p(2, 2) ;

  on(p) {
    Range x = new BlockRange(100, p.dim(0), 1) ;  // ghost width 1
    Range y = new BlockRange(200, p.dim(1), 1) ;  // ghost width 1

    float [[#,#]] u = new float [[x, y]] ;

    // ... some code to initialise `u'

    int [] widths = {1, 1} ;        // Widths actually updated
    Mode [] modes = {CYCL, CYCL} ;  // Wraparound at ends.

    HPJlib.writeHalo(u, widths, modes) ;

    float [[#,#]] v = new float [[x, y]] ;

    Index i, j ;
    over(i = x | :)
      over(j = y | :)
        v [i, j] = 0.25 * (u [i + 1, j] + u [i - 1, j] +
                           u [i, j + 1] + u [i, j + 1]) ;

    HPJlib.copy(u, v) ;
  }
\end{verbatim}
\normalsize
\caption{\label{JacobiGhost}Jacobi iteration using {\tt writeHalo}.}
\end{figure}

\section{Other features}

We have already described most of the important {\em language} features
we propose to implement.  Two additional features that are quite important in
practice but have not been discussed are {\em subranges} and {\em
subgroups}.  A subrange is simply a range which is a regular section of
some other range, created by syntax like \verb$x [0 : 49]$.  Subranges
are created tacitly when a distributed array is subscripted with a
triplet, and they can also be used directly to create distributed
arrays with general HPF-like alignments.  A {\em subgroup} is some
slice of a process array, formed by restricting process coordinates in
one or more dimensions to single values.  Again they may be created
implicitly by section subscripting, this time using a scalar subscript.
They also formally describe the state of the active process group
inside {\em at} and {\em over} constructs.

The framework described is much more powerful than space allows
us to demonstrate in this paper.  This power comes in part from the
flexibility to add features by extending the libraries associated
with the language.  We have only illustrated the simplest kinds of
distribution format.  But any HPF 1.0 array distribution format, plus
various others, can be incorporated by extending the {\tt Range}
hierarchy in the run-time library.  We have only illustrated {\tt
shift} and {\tt writeHalo} operations from the communication library,
but the library also includes much more powerful operations for
remapping arrays and performing irregular data accesses.  Our intention is
to provide minimal language support for distributed arrays, just enough
to facilitate further extension through construction of new libraries.

For a more complete description of a slightly earlier version of
the proposed language, see \cite{JavaAd}.

\section{Discussion and related work}

We have described a conservative set of extensions to Java.  In the
context of an explicitly SPMD programming environment with a good
communication library, we claim these extensions provide much of the
concise expressiveness of HPF, without relying on very sophisticated
compiler analysis.
The object-oriented features of Java
are exploited to give an elegant parameterization of the distributed
arrays of the extended language.
Because of the relatively low-level
programming model, interfacing to other parallel-programming paradigms
is more natural than in HPF.  With suitable care, it is possible to
make direct calls to, say, MPI from within the data parallel
program.  In \cite{JavaMPI} we suggest a concrete Java binding
for MPI.

We will mention two related projects.
Spar \cite{Spar} is a Java-based language for array-parallel programming.
Like our language it introduces multi-dimensional arrays, array
sections, and a parallel loop.  There are some similarities in syntax, but
semantically Spar is very different to our language.  Spar expresses
parallelism but not explicit data placement or communication---it is a
higher level language.
ZPL \cite{ZPL} is a new programming language for scientific
computations.  Like Spar, it is an array language.  It has an idea of
performing computations over a {\em region}, or set of indices.  Within
a compound statement prefixed by a {\em region specifier}, aligned
elements of arrays distributed over the same region can be accessed.
This idea has certain similarities to our {\em over} construct.
Communication is more explicit than Spar, but not as explicit as
in the language discussed in this article.

\bibliography{Java98}

\end{document}

