
\documentclass{article}

\usepackage{psfig}

\begin{document}

\newtheorem{usageRule}{Rule}

\sloppy

\pagestyle{plain}

\title{Semantic Checking in {\em HPJava}}

\author{Bryan Carpenter, Geoffrey Fox and Guansong Zhang\vspace{0.2in} \\ 
        \em Northeast Parallel Architectures Centre, \\
        \em Syracuse University, \\
        \em 111 College Place, \\
        \em Syracuse, New York 13244-410 \\
        \em \{dbc,gcf,zgs\}@npac.syr.edu}

\maketitle

\begin{abstract}
The article discusses various rules about use of distributed arrays
in {\em HPJava} programs.  These rules are peculiar
to the HPspmd programming model.  They can be enforced by a
combination of static semantic checks, compile-time analysis and
compiler-generated run-time checks.  We argue that the the cost of any
necessary run-time checks should be acceptable, because, by design,
the associated computations can be lifted out of inner loops.
\end{abstract}

\section{Introduction}

HPJava \cite{HPJava} is a Java language binding of a programming model
we call the {\em HPspmd} model.  This is a particular version of the
general SPMD programming model that adds special support for
distributed arrays of the kind defined by the HPF standard.  The arrays
are bound to the base language through a series of syntax extensions.
The assumption is that the syntax extensions will be handled by a
relatively simple preprocessor which emits an SPMD program in the base
language.

Of course this implementation strategy has been followed with varying
degrees of success in many translation systems for HPF and similar
languages \cite{HPFStandard,HPFBook}.  The difference in the HPspmd
approach is that it is assumed the source code is already written in an
explicitly MIMD style.  The HPspmd syntax provides only a thin veneer
on low-level SPMD programming, and the transformations applied by the
translator are relatively simple and direct---no non-trivial analysis
should be needed to obtain good parallel performance.  Meanwhile
the language model provides a uniform model of a distributed
array, which can be targetted by libraries for parallel communication
and arithmetic.

Of course SPMD programming has been very successful.  There are
countless applications written in the most basic SPMD style, using
direct message-passing through MPI \cite{MPIStandard} or similar
low-level packages.  Many higher-level parallel programming
environments and libraries assume the SPMD style as their basic model.
Examples include ScaLAPACK \cite{ScaLAPACK}, 
%PetSc \cite{PETSc1},
DAGH \cite{HDDA_DAGH},
Kelp \cite{Block_structured} and the Global Array
Toolkit \cite{Global_Arrays}.  While there remains a prejudice that HPF
is best suited for problems with very regular data structures and
regular data access patterns, SPMD frameworks like DAGH and Kelp have
been designed to deal directly with irregularly distributed data, and
other libraries like CHAOS/PARTI \cite{CHAOS} and Global Arrays support
unstructured access to distributed arrays.

These successes aside, the library-based SPMD
approach to data-parallel programming lacks the uniformity
and elegance of HPF.  All the environments referred to above have some
idea of a distributed array, but they all describe those arrays
differently.  Compared with HPF, creating distributed arrays and
accessing their local and remote elements is clumsy and error-prone.
Because the arrays are managed entirely in libraries, the compiler
offers little support and no safety net of compile-time or
compiler-generated run-time checking.  These observations
motivate our introduction of the HPspmd model.

This article concentrates in particular on the issue of semantic
checking in HPspmd languages.  The basic features of the HPspmd
language model are introduced in the Java context in section
\ref{sec:HPJava}.  Section \ref{sec:sections} adds a discussion
of {\em array sections}.  This discussion serves to introduce
the ideas of subranges and general process groups in HPspmd.  
Building on these ideas, section \ref{sec:usage} describes
some general rules about access to distributed array elements, and
section \ref{sec:compiling} describes how a translator can
ensure these conditions are met. 

\section{HPJava---an HPspmd language\label{sec:HPJava}}

HPJava \cite{HPJava} is an instance of our HPsmpd language
model.  HPJava extends its base language, Java, by adding some
predefined classes and some additional syntax for dealing with
distributed arrays.

We aim to provide a flexible hybrid of the data parallel and low-level
SPMD paradigms.  To this end HPF-like distributed arrays appear as
language primitives.  But a design decision is made that all access to {\em
non-local} array elements should go through library functions---either
calls to a collective communication library, or simply {\tt
get} and {\tt put} functions for access to remote blocks of a
distributed array\footnote{The distributed arrays are orthogonal
to the sequential arrays of the base language---we deliberately
keep them completely separate.}.

A subscripting syntax is applied to distributed arrays to reference
elements.  But an array element reference must not imply access to a
value held on a different processor.  To simplify the task of the
programmer, who must be sure accessed elements are held locally, the
languages adds {\em distributed control} constructs.
These play a role something like the {\tt ON HOME} directives of HPF
2.0 and earlier data parallel languages \cite{Kali}.  One special
control construct---a distributed parallel loop---facilitates traversal
of locally held elements from a group of aligned arrays.

Array mapping is described in terms of a slightly different set of
basic concepts from HPF.  {\em Process group} objects generalize the
processor arrangements of HPF.  {\em Distributed range} objects are
used instead HPF templates.  A distributed range is comparable with a
single dimension of an HPF template.  These substitutions are a change
of parametrization only.  Groups and ranges fit better with our
distributed control constructs.

\begin{figure*}[tbp]
\small
\begin{verbatim}
  Procs2 p = new Procs2(P, P) ;
  on(p) {
    Range x = new BlockRange(M, p.dim(0)) ;
    Range y = new BlockRange(N, p.dim(1)) ;

    float [[,]] a = new float [[x, y]], b = new float [[x, y]],
                c = new float [[x, y]] ;

    ... initialize values in `a', `b'

    overall(i = x for :)
      overall(j = y for :)
        c [i, j] = a [i, j] + b [i, j] ;
  }
\end{verbatim}
\normalsize
\caption{A parallel matrix addition.\label{fig:addition}}
\end{figure*}

Figure \ref{fig:addition} is a simple example of an HPJava program.  It
illustrates creation of distributed arrays, and access to their
elements.  The class {\tt Procs2} is a standard library class derived
from the special base class {\tt Group}.  It represents a
two-dimensional grid of processes.  Similarly the distributed range class
{\tt BlockRange} is a library class derived from the special class {\tt
Range}; it denotes a range of subscripts distributed with
BLOCK distribution format over a specific process dimension.  Process
dimensions associated with a grid are returned by the {\tt dim()}
inquiry.
The {\tt on(p)} construct is a new control construct specifying that
the enclosed actions are performed only by processes in group {\tt p}.

The variables {\tt a}, {\tt b} and {\tt c} are all distributed array
objects.  The type signature of an $r$-dimensional distributed array
involves double brackets surrounding $r$ comma-separated slots.  The
constructors specify that these all have ranges {\tt x} and {\tt y}---they
are all {\tt M} by {\tt N} arrays, block-distributed over {\tt p}.

A second new control construct, {\em overall}, implements a distributed
parallel loop.  The
constructs here iterate over all locations (selected by the degenerate
interval ``\verb$ : $'') of ranges \verb$x$ and \verb$y$.  The symbols
{\tt i} and {\tt j} scoped by these constructs are {\em bound
locations}.  In HPF, a distributed array element is referenced using
integer subscripts, like an ordinary array.  In HPJava, with a couple
of exceptions noted below, the subscripts in element references must be
bound locations, and these must be locations in the range associated
with the array dimension.  This rather drastic restriction is a
principal means of ensuring that referenced array elements are held
locally.

\begin{figure}[btp]

\small
\begin{verbatim}
  Procs2 p = new Procs2(P, P) ;
  on(p) {
    Range x = new ExtBlockRange(N, p.dim(0), 1, 1) ;
    Range y = new ExtBlockRange(N, p.dim(1), 1, 1) ;

    float [[,]] u = new float [[x, y]] ;

    ... some code to initialise `u'

    for(int iter = 0 ; iter < NITER ; iter++) {

      Adlib.writeHalo(u) ;

      overall(i = x for 1 : N - 2)
        overall(j = y for 1 + (i` + iter) % 2 : N - 2 : 2)
          u [i, j] = 0.25 * (u [i - 1, j] + u [i + 1, j] +
                             u [i, j - 1] + u [i, j + 1]) ;
    }
  }
\end{verbatim}
\normalsize

\caption{\label{fig:redBlack}Red-black iteration.}
\end{figure}

The general policy is relaxed slightly to simplify coding of stencil
updates.  A subscript can be a {\em shifted location}.  Usually
this is only legal if the subscripted array is declared with suitable {\em
ghost regions} \cite{Ghost}.  Figure \ref{fig:redBlack} illustrates the
use of the standard library class {\tt ExtBlockRange} to create arrays with
ghost extensions (in this case, extensions of width 1 on either side of
the locally held ``physical'' segment).  A function, {\tt writeHalo}, from
the communication library Adlib updates the ghost region.  This example
also illustrates application of a postfix backquote operator to a bound
location.  The expression {\tt i`} (read ``i-primed'') yields the integer
global loop index.

Distributed arrays can be defined with some sequential dimensions.  The
sequential attribute of an array dimension is flagged by an asterisk in
the type signature.  As illustrated in Figure \ref{fig:pipedMatmul},
element reference subscripts in sequential dimensions can be ordinary
integer expressions.

% (Original position of Figure \ref{fig:pipedMatmul})

\begin{figure}[tbp]
\small
\begin{verbatim}
  Procs1 p = new Procs1(P) ;
  on(p) {
    Range x = new BlockRange(N, p.dim(0)) ;

    float [[,*]] a = new float [[x, N]], c = new float [[x, N]] ;
    float [[*,]] b = new float [[N, x]], tmp = new float [[N, x]] ;

    ... initialize `a', `b'

    for(int s = 0 ; s < N ; s++) {

      overall(i = x for :) {

        float sum = 0 ;
        for(int j = 0 ; j < N ; j++)
          sum += a [i, j] * b [j, i] ;

        c [i, (i` + s) % N] = sum ;
      }

      // cyclically shift `b' (by amount 1 in x dim)...

      Adlib.cshift(tmp, b, 1, 1) ;
      HPspmd.copy(b, tmp) ;
    }
  }
\end{verbatim}
\normalsize
\caption{A pipelined matrix multiplication program.\label{fig:pipedMatmul}}
\end{figure}


The short examples here have covered the basic syntax of HPJava.  The
language itself is relatively simple.  Complexities associated with
varied and irregular patterns of communication are dealt with in
libraries, which can implement many richer operations than the {\tt
writeHalo} and {\tt cshift} functions of the examples.  The remaining
important extensions to the language itself can be motivated most
easily by considering the need to support Fortran 90 style {\em
sections} of distributed arrays.

\section{Array sections\label{sec:sections}}

% (Move Figure \ref{fig:pipedMatmul} here to avoid it appearing on a
% separate page)

HPJava has a syntax for representing subarrays.  An {\em array
section expression} has a similar syntax to a distributed array element
reference, but uses double brackets.  Whereas an element reference is a
variable, an array section is an expression representing a new
distributed array object.  The new array contains a subset of the
elements of the parent array.  Those elements can subsequently be
accessed either through the parent array {\em or} through the array
section.  The HPJava implementation of an array section is closely
related to the Fortran 90 idea of an array pointer---in Fortran an
array pointer can reference an arbitrary regular section of a target
array.

In the previous section it was stated that the subscripts in a
distributed array element reference are either {\em locations} or
(restrictedly) integer expressions.  Options for subscripts in array
section expressions are freer.  For example, a
section subscript is allowed be a triplet.
In the simplest kinds of array section
the rank of the result is equal to the number of triplet
subscripts.  If the section also has some scalar subscripts,
this will be lower than the rank of the parent array.
We would like to be able define the mapping of an arbitrary array
section.

The mapping of a distributed array is defined
by its {\em distribution group} and its list of ranges.
In earlier examples the distribution group of arrays defaulted to the
process group specified by the surrounding on construct.
In general the distribution group can be specified explicitly
by appending an ``on'' clause to constructor itself:
\small
\begin{tabbing}
\verb$    new ${\em type }\verb$[[$$x_0$\verb$, $$x_1$\verb$, ...]] on $$p$
\end{tabbing}
\normalsize
Here each of $x_0, x_1, \ldots$ is a range object or an integer
(in which case the dimension is sequential), and $p$ is a group
contained within the set of processes that create the array.
The ranges must be distributed over distinct dimensions of $p$.
If there is any dimension of $p$ which is not a distribution target
of some range from $x_0, x_1, \ldots$, the array is {\em replicated}
over that process dimension.
The inquiries {\tt grp} and {\tt rng(int r)} return the group and ranges
for any array\footnote{For a sequential dimension the result of {\tt
rng(r)} is a member of the subclass {\tt CollapsedRange}.}.

\begin{figure*}[tbp]
\centerline{\psfig{figure=section2.eps,width=4in}}
\caption{A two-dimensional section of a two-dimensional array
(shaded area).\label{fig:section2}}
\end{figure*}

Now consider array section {\tt b} defined by
\small
\begin{verbatim}
    float [[,]] a = new float [[x, y]] on p ;

    float [[,]] b = a [[0 : N - 1 : 2, 0 : N / 2 - 1]] ;
\end{verbatim}
\normalsize
(see Figure \ref{fig:section2}).  What are the ranges of {\tt b}?
In fact they are new kind of range called a {\em subrange}.
For completeness there is a special syntax for constructing subranges
directly:
\small
\begin{verbatim}
    Range u = x [0 : N - 1 : 2] ;
    Range v = y [0 : N / 2 - 1] ;
\end{verbatim}
\normalsize
The extents of both {\tt u} and {\tt v} are {\tt N / 2}.

The {\em distribution group}
of {\tt b} can be identified with the distribution
group of the parent array {\tt a}.  
But sections constructed using a scalar subscript, eg
\small
\begin{verbatim}
    float [[]] c = a [[0, :]]
\end{verbatim}
\normalsize
(see Figure \ref{fig:section1}) present another problem.  Clearly {\tt
c.rng(0)} is {\tt y}.  But identifying the distribution group of {\tt
c} with {\tt p} is not right.  It would imply that {\tt c} is
replicated over the first dimension of {\tt p}.  Where does the
information that {\tt c} is localized to the top row of processes go?

\begin{figure*}[tbp]
\centerline{\psfig{figure=section1.eps,width=4in}}
\caption{A one-dimensional section of a two-dimensional array
(shaded area).\label{fig:section1}}
\end{figure*}

We are driven to define a new kind of group: a {\em restricted group} is
the subset of processes in some parent group to which a particular
location is be mapped.  The distribution group of
{\tt c} is defined to be the subset of processes in {\tt p} to which
the location {\tt x[0]} is mapped.  It can be written explicitly as%
\footnote{The expression {\tt x[0]} is ``pure syntax'' in HPJava.  
There is no Java type associated with a location
and no way to store a location value in a variable.  The only named
locations in HPJava are the bound locations scoped by overall
(and {\em at}---see section \ref{sec:discussion}).
Besides group restrictions, expressions like {\tt x[0]} can legally appear
as array section subscripts or in the header of an at construct.}
\small
\begin{verbatim}
    p / x [0]
\end{verbatim}
\normalsize
An equivalent definition of a restricted group is as some slice of a
process grid, chosen by restricting some of the coordinates to single
values.  

The idea of a restricted group may look slightly ad hoc, but the
implementation is quite elegant.  A restricted group is uniquely
specified by its set of effective process dimensions and the identity
of the {\em lead} process in the group---the process with coordinate
zero relative to the effective dimensions.  The dimension set can be
specified as a subset of the dimensions of the parent grid using a
simple bitmask.  The identity of the lead process can be specified
through a single integer ranking the processes of the parent grid.  So
a general (restricted) HPJava group can be fully parametrized by a
reference to the parent process grid together with just two {\tt int}
fields.

Now we can formally define of the mapping
of a typical array section.  As a matter of
definition an integer subscript $n$ in dimension $r$ of
array $a$ is equivalent to a location-valued subscript
$a${\tt .rng(}$r${\tt )[}$n${\tt ]}.
By definition, a triplet subscript $l${\tt :}$u${\tt :}$s$ in the same
dimension is equivalent to range-valued subscript
$a${\tt .rng(}$r${\tt )[}$l${\tt :}$u${\tt :}$s${\tt ]}.
If all integer and triplet subscripts in a section are replaced by their
equivalent location or range subscripts, and the
location-valued subscripts are $i, j, \ldots$, then the
distribution group of the section is
\small
\begin{tabbing}
\verb$    $$a$\verb$.grp() / $$i$\verb$ / $$j$\verb$ / ...$
\end{tabbing}
\normalsize
and the $s$th range of the section is equal to the $s$th range-valued
subscript.

Subranges and restricted groups can be used in array constructors on
the same footing as the ranges and grids introduced earlier.
This enables HPJava arrays to reproduce any alignment option
allowed by the {\tt ALIGN} directive of HPF.

\section{Variable usage rules\label{sec:usage}}

A basic principle of the HPspmd model is that a program can only
directly reference locally held variables.  Therefore
we want to define the conditions under which it is legal to
access a particular variable (most especially, a particular
array element) at a particular point in a program.

In a distributed array element reference,
as explained in section \ref{sec:HPJava}, a subscript in 
a distributed dimension of an array must be a bound location,
and this location must be a location in the appropriate range of the array%
\footnote{We can relax this restriction slightly by adding rules for
identifying locations in certain closely related ranges.
For example it is very natural to regard the locations in a subrange
to be a subset of the locations in the parent range.}%
.  Hence
\begin{usageRule}
Subscripts in distributed array element references must be bound
locations (unless the relevant dimension is sequential).
\end{usageRule}
and
\begin{usageRule}
If a location appears as a subscript,
it must be a location in the relevant range of the array.
\end{usageRule}

The rules just stated go a long way towards ensuring that only local
elements are accessed, but they don't cover all cases.
Consider the example of Figure \ref{fig:accessError}.
The subscripts on the element reference {\tt c[j]} are legal---{\tt j}
is a location in {\tt c.rng(0)} (which is equal to {\tt y}).
But, referring to Figure \ref{fig:section1}, the section
{\tt c} is localized to {\tt p/x[0]}---the top row of processes
in the figure---whereas the on construct
specifies that the element assignments are performed in the group
{\tt p/x[7]}---the bottom row of processes.

\begin{figure*}[tbp]
\small
\begin{verbatim}
    Procs2 p = new Procs2(P, P) ;

    Range x = new BlockRange(N, p.dim(0)) ;
    Range y = new BlockRange(N, p.dim(1)) ;

    float [[,]] a = new float [[x, y]] on p ;

    float [[]]  c = a [[0, :]] ;

    on(p / x [N - 1])
      overall(j = y for :)
        c [j] = j` ;               // error!
\end{verbatim}
\normalsize
\caption{Erroneous access to non-local elements of a section
\label{fig:accessError}}
\end{figure*}

Before we can give a general rule about accesses to elements
of distributed arrays we need to refine the concept of the
{\em active process group}.  Suppose the set of processes currently executing
the program text is represented by the group $q$, and the construct
\small
\begin{tabbing}
\verb$    on($$p$\verb$) $$S$
\end{tabbing}
\normalsize
appears.  For the body, $S$, of this construct, the active process group is
changed to $p$.  The group $p$ must represent a subset of the processes
in $q$.
Similarly, suppose the current active process group is $q$, and the
construct 
\small
\begin{tabbing}
\verb$    overall($$i$\verb$ = $$x$\verb$ for $$l$\verb$ : $$u$\verb$ : $$s$\verb$) $$S$
\end{tabbing}
\normalsize
appears.  For $S$ the active process group is changed to $q / i$.
Unless $x$ is a collapsed range, the process dimension over which it
is distributed {\em must be a dimension of $q$}.

Applied recursively these rules define the active process group at any
point in a legal HPJava program.

Next we will define the {\em home group} of a program variable.
The home group of an ordinary variable (not a distributed array element) is
the group that was active at the point of declaration of the variable.
For a distributed array element the rule is similar to the definition of
distribution groups of section: suppose the subscripts in the reference
\small
\begin{tabbing}
\verb$    $$a$\verb$ [$$e_0$\verb$, $$e_1$\verb$, ...]$
\end{tabbing}
\normalsize
include locations $i, j, \ldots$, then the home group of the array
element is%
\footnote{As a matter of definition, for a shifted location
$\mbox{\tt p} / (\mbox{\tt i} \pm \mbox{\em expression}) \equiv
\mbox{\tt p} / \mbox{\tt i}$.
The rationale is that a shifted location is supposed to find an array
element in the same process as the original location (albeit that
the element could be in a ghost region).}
\small
\begin{tabbing}
\verb$    $$a$\verb$.grp() / $$i$\verb$ / $$j$\verb$ / ...$
\end{tabbing}
\normalsize
(if the array has a replicated distribution this group may contain
several processes; otherwise it contains a single process).

An HPspmd rule for accessing a variable can be stated as follows:
\begin{usageRule}
A variable can only be accessed when the active process
group is contained in the home group of the variable.
\end{usageRule}
This is essentially a statement that all processes executing the
current text must hold a copy of any variable accessed.
Strictly speaking it is slightly stronger than the requirement that 
the local process holds a copy of any element it accesses.
%We will consider alternative formulations in the next section.
The version given here is convenient for formal discussion of
programs.

%Note that a distributed array variable (such as $a$ itself) is regarded
%as an ordinary variable and therefore has a home group.  In simple
%cases this home group may coincide with the distribution group of the
%referenced array (see the examples of section \ref{sec:HPJava}), but
%this is by no means true in all cases---it is not true of most examples
%from section \ref{sec:sections}.

\section{Compiling usage checks\label{sec:compiling}}

It is very important for the efficiency of HPJava programs that inner loops
in the translated code---especially those derived from overall
constructs in the source program---be kept as simple as possible.  For
example, the body of the emitted loop should avoid calls to methods
associated with the runtime descriptor of the distributed array
(unless, say, the programmer forced appearance of these calls by
including them in the source).  In particular runtime checking
code associated with references to distributed array element should be
lifted out of loops wherever possible.

Rule 1 can be enforced at compile
time by a trivial extension to normal type checking.  So far as Rule 2
is concerned, the intention is that run-time checks
needed to enforce this rule should be lifted outside of the
loop body.  Consider this fragment
\small
\begin{tabbing}
\verb$    overall($$i$\verb$ = $$x$\verb$ for $$l$\verb$ : $$u$\verb$ : $$s$\verb$) {$ \\
\verb$      ...$ \\
\verb$        ... $$a$\verb$ [..., $$i$\verb$, ...] ...$ \\
\verb$    }$
\end{tabbing}
\normalsize
where $i$ is the $r$th subscript in the list.  Of course the expression
$a$ must be a distributed array.  A naive insertion of runtime checks
might lead to code something like
\small
\begin{tabbing}
\verb$    overall($$i$\verb$ = $$x$\verb$ for $$l$\verb$ : $$u$\verb$ : $$s$\verb$) {$ \\
\verb$      ...$ \\
\verb$        ASSERT($$a$\verb$.rng($$r$\verb$).containsLocation($$x$\verb$, $$i$\verb$`)) ;$ \\
\verb$        ... $$a$\verb$ [..., $$i$\verb$, ...] ...$ \\
\verb$    }$
\end{tabbing}
\normalsize
with the runtime test appearing immediately before the element reference.
What we want to achieve instead is something like
\small
\begin{tabbing}
\verb$    ASSERT($$a$\verb$.rng($$r$\verb$).containsSubrange($$x$\verb$, $$l$\verb$, $$u$\verb$, $$s$\verb$)) ;$ \\
\verb$    overall($$i$\verb$ = $$x$\verb$ for $$l$\verb$ : $$u$\verb$ : $$s$\verb$) {$ \\
\verb$      ...$ \\
\verb$        ... $$a$\verb$ [..., $$i$\verb$, ...] ...$ \\
\verb$    }$
\end{tabbing}
\normalsize
In fact we expect that in {\em typical} HPJava programs this
transformation will be legal.  In simple cases there may even be enough
static information about the ranges of $a$ to eliminate the run-time
check altogether, because the lifted assertion can be proven at compile
time.  

There are cases where lifting the usage check may be problematic.
Consider this example
\small
\begin{verbatim}
    Range y = x [1 : N - 2] ;
    float [[]] a = float [[y]] ;

    overall(i = x for 0 : N - 1) {
      ...
      if(i` > 0 && i` < N - 1)
        ... a [i] ...
    }
\end{verbatim}
\normalsize
The array is defined over some subrange of {\tt x}.  We iterate
over the whole of {\tt x}, but mask the disallowed accesses with a
nested conditional construct.  Such usage would block naive lifting of
runtime range-checks out of the loop.  The lifted range-check would
cause an exception although the code is apparently legal.
Another potential difficultly is that the expression $a$ is not a loop
invariant with respect to the overall construct scoping the subscript.
For example:
\small
\begin{verbatim}
    float [] [[]] s = new float [n] [[]] ;
    ... allocate individual distributed arrays in `s' ...

    overall(i = x for :) {
      ...
      for(int j = 0 ; j < n ; j++)
        s [j] [i] = foo(j, i`) ;
    }
\end{verbatim}
\normalsize
The variable {\tt s} is a Java array of HPJava distributed arrays.
The array expression, $a$, is now {\tt s[j]}.  This expression is
not loop invariant with respect to the overall construct.
Without detailed (and not obviously practical) compile-time analysis
the range check for the distributed array reference cannot be
lifted out of the loops%
\footnote{Both examples would probably lead to inefficient translated
code anyway, because various other computations associated with element
reference cannot be lifted out of loops.}%
.

We assume these uses are idiosynchratic.  In the first case
the desired effect could be normally achieved more directly by
changing the triplet in the overall header%
\footnote{We emphasize that the problem here is not with appearance of
conditional code in general inside an overall construct---it is only
with use of conditional code to mask element references otherwise
forbidden by range-checking considerations.}%
:
\small
\begin{verbatim}
    overall(i = x for 1 : N - 2) {
      ...
      ... a [i] ...
    }
\end{verbatim}
\normalsize
In the second example the appearance of {\tt i} as a common subscript for all
distributed arrays in {\tt s} strongly suggests that these have a
common range, probably {\tt x}.  So replacing {\tt s} with a two-dimensional
distributed array,
\small
\begin{verbatim}
    float [[*,]] s = new float [[n, x]] ;
\end{verbatim}
\normalsize
would probably be acceptable, and would remove the blockage to lifting
the range-check.

Because efficient translation is a primary concern, and the
examples above are difficult to translate efficiently, we outlaw them
by adding a new rule
\begin{usageRule}
Logically, the effects of any bound location subscripts appearing in
a program are resolved at the head of the construct that scopes the
location.
\end{usageRule}
Interpretations of this rule (which is more pragmatic than
beautiful) include
\begin{description}
\item{a)}
a range check error may be thrown by the overall construct, even if
later conditional code masks the element usage---so far as range
checks are concerned the use is assumed to be unconditional---and
\item{b)}
if a bound location subscript is applied to a distributed array
expression (in an element reference or array section), the array
expression must be invariant in the scope of the location.
\end{description}
The translator may apply suitable dataflow analysis to verify condition
b)%
\footnote{Dependence of the array expression on field variables
(effectively global variables) may frustrate this analysis, if method
calls also appear inside the loops.}%
.

Next we consider Rule 3 of section \ref{sec:usage}.
We assume that the translator makes the current state of the
active process group visible in a variable of type {\tt Group}
called {\tt apg}%
\footnote{The variable {\tt apg} is not a conventional Java global
variable.  It is more closely akin to the {\tt this} expression of
Java.  It will be passed to HPJava-aware methods through a hidden
argument.}%
.  Most naively, the range checks for the example in Figure
\ref{fig:accessError} could be inserted as follows:
\small
\begin{verbatim}
    float [[,]] a = new float [[x, y]] on p ;

    float [[]]  c = a [[0, :]] ;

    on(p / x [N - 1])
      overall(j = y for :) {
        ASSERT((c.grp() / j).contains(apg)) ;
        c [j] = j` ;
      }
\end{verbatim}
\normalsize
We note, however, that this can be optimized by lifting the assertion
through the overall scoping {\tt j}:
\small
\begin{verbatim}
    on(p / x [N - 1]) {
      ASSERT(c.grp().contains(apg)) ;
      overall(j = y for :)
        c [j] = j` ;
   }
\end{verbatim}
\normalsize
and the lifted assertion is even simpler to compute, because the
restriction by {\tt j} is no longer needed.

In general in
\small
\begin{tabbing}
\verb$    overall($$j$\verb$ = $$y$\verb$ for $$l$\verb$ : $$u$\verb$ : $$s$\verb$) {$ \\
\verb$      ...$ \\
\verb$        ASSERT(($$a$\verb$.grp() / ... / $$i$\verb$ / $$j$\verb$ / $$k$\verb$ / ... ).contains(apg)) ;$ \\
\verb$        ... $$a$\verb$ [..., $$j$\verb$, ...] ...$ \\
\verb$        ...$ \\
\verb$    }$
\end{tabbing}
\normalsize
provided the ellided code between the overall header and the assertion
contains no statements that change {\tt apg} (ie, no other
overall or on headers) the assertion can be lifted and simplified as follows:
\small
\begin{tabbing}
\verb$    ASSERT(($$a$\verb$.grp() / ... / $$i$\verb$ / $$k$\verb$ / ... ).contains(apg)) ;$ \\
\verb$    overall($$j$\verb$ = $$y$\verb$ for $$l$\verb$ : $$u$\verb$ : $$s$\verb$) {$ \\
\verb$      ...$ \\
\verb$        ... $$a$\verb$ [..., $$j$\verb$, ...] ...$ \\
\verb$        ...$ \\
\verb$    }$
\end{tabbing}
\normalsize
We can legitimize the lifting, even if the element reference appears in
conditional code inside the loop, by appealing to Rule 4.  If overall
constructs are suitably nested this transformation can be applied
recursively.

Exact computation of the group {\tt contains} operation may be
expensive---in principle it is a set containment test.  
%An approximate
%(stronger) definition which can be useful is that a process grid $p$ is
%contained by $q$ if $q$ was {\tt apg} when $p$ was created, and that
%any group $r / i$ is contained in $r$.  These two relations and their
%closure can be computed relatively cheaply.  
In the present case, where containment of the active process group is
the issue, an alternative is available.  If all members
of the active process group make the assertion
\small
\begin{tabbing}
\verb$    ASSERT($$p$\verb$.contains(apg)) ;$
\end{tabbing}
\normalsize
this is essentially equivalent to them all making the assertion
\small
\begin{tabbing}
\verb$    ASSERT($$p$\verb$.amMember()) ;$
\end{tabbing}
\normalsize
The {\tt amMember} method on a group returns true if the the local
process is a member of the group and false otherwise.  If the {\tt
contains} assertion is false, then at least one member of the active
process group will throw an exception.  If an exception is expected to
abort the program globally, this is probably good enough.  The {\tt
amMember} function can be computed very efficiently---in fact an extra
boolean field should be added to the {\tt Group} record containing this
precomputed value, so it is simply a lookup.

Whether this approximation to the exact {\tt contains} test is
acceptable in practise is a slightly open issue.  In ``normal'' styles
of HPJava programming we expect that it is essentially equivalent to
the exact set containment test.  But in principle its adoption
alters the semantics of the program.  Replacing {\tt contains} by {\tt
amMember} only weakens the requirement of Rule 3, which may not be
a problem, unless Java-like exact exception handling is expected.
In general our policy of lifting runtime checks is somewhat at odds
with the Java ethos that exceptions should be thrown and control
switched at exactly the point where the error (eg, the erroneous
subscripting operation) occurred.  In HPJava the most important thing
is to get good performance, and probably some aspects of the strict
Java exception model have to be sacrificed.

We did not discuss use of shifted locations as subscripts.  They do not
introduce any fundamental issues.  A runtime inquiry on array ranges
will be needed giving the size of the ghost extensions.  Rule 4 can be
interpretted to mean that use of non-invariant expressions for the
shift-amounts is disallowed.

\section{Other Usage Rules}

In this section we mention two usage guidelines which have some
relationship to the usage rules of the previous sections, but which
are {\em not} required to make a legal HPJava program, and are not
checked by the translator.

The first relates to a {\em coherence} property of variables.
Informally, we will call a variable coherent if, at corresponding stages
of execution of an SPMD program, processes belonging to the
variable's home group always hold identical values in their local copies
of the variable.  If the home group of a variable is a single process,
the variable is trivially coherent.

Giving a precise formal definition of coherence is quite difficult,
because it is difficult to define formally what is meant by
``corresponding stages of execution''.  What we can give is
a precise rule which will keep all variables in a program coherent.
This rule, which looks similar to usage Rule 3, is
\begin{usageRule}
A variable should only be updated when the active process
group is identical to the home group of the variable.
\end{usageRule}
Assuming that any external method calls also respect the coherence
property\footnote{Examples of methods that do {\em not} respect
coherence include functions like {\tt MPI\_Comm\_rank}, which return
results intrinsically dependent on process id.} we assert that
if this rule is applied throughout a program it will ensure all
variable stay coherent.  An HPspmd translater does not try to enforce
this rule because the associated checks can be very expensive, and
because judicious use of incoherence is often convenient.

It is often required that scalar arguments of collective operations
should be coherent.  There is another common requirement collective
operations on arrays---namely that all copies of all
elements of array arguments should be held in the group of processes
that collectively invoke the operation.  We will say that a distributed
array $a$ is {\em fully contained} (in the active process group) if
$a\mbox{\tt .grp()} \subseteq \mbox{\tt apg}$.  Therefore a natural
requirement is
\begin{usageRule}
All distributed array arguments of collective operations should
be fully contained.
\end{usageRule}
This rule is not required for all non-local operations.  For example
it is not required for {\tt get} and {\tt put} operations that
implement one-sided communication.  So it is not appropriate for
the translator to try to enforce this rule.  Instead checks such
as
\small
\begin{verbatim}
    ASSERT(apg.contains(a.grp())) ;
\end{verbatim}
\normalsize
should appear inside the implementation of collective library functions,
as required.

\section{Discussion\label{sec:discussion}}

We have outlined a set of usage rules that HPspmd programs must
follow to ensure that only locally available array elements are
accessed directly.  Special control constructs and other syntax in
HPJava are designed to ensure that these rules can be
enforced by a combination of static semantic checks, compile-time
analysis and compiler-generated run-time checks.  We showed how the
language can be set up to ensure that the cost of any run-time checks
required should not be prohibitive, even taking into account the
complex index spaces of HPF-like distributed arrays.

The discussion has been in the context of a Java-based HPspmd language
called HPJava.  In \cite{Bindings} we have outlined possible syntax
extensions to Fortran to provide similar semantics to HPJava.
One feature of the HPJava language that was not discussed in this
article is the {\em at construct}.  This is very similar to the
overall construct except that it specifies execution for a single
location.  For the purposes of this article it can be regarded as
a special case of overall.

Two recent languages that have some similarities to our HPspmd
languages are F-\,- and ZPL.  F-\,- \cite{FMM} is an extended Fortran
dialect for SPMD programming.  It does not incorporate the HPF-like
idea of a global index space, so most of the issues discussed in the
this article do not arise in F-\,-.  ZPL \cite{ZPL} is a array parallel
programming language for scientific computations.  ZPL uses pure array
syntax for accessing its parallel arrays, and does not allow them to
subscripted directly.  So the range-checking issues are unlikely to be
directly comparable to those in HPJava.

\section{Acknowledgements}

This work was supported in part by the National Science Foundation
Division of Advanced Computational Infrastructure and Research,
contract number 9872125.

\bibliographystyle{plain}
\bibliography{translation}

\end{document}

